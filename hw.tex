\documentclass[a4paper, fleqn]{article}
\usepackage[utf8]{inputenc}
\usepackage[T1, T2A]{fontenc}
\usepackage[english,bulgarian]{babel}
\usepackage{amsthm}
\usepackage{amsmath}
\usepackage{amsfonts}
\usepackage[a4paper, left=0.50in, right=0.50in, top=0.5in, bottom=1.0in]{geometry}
\usepackage{enumitem}
\usepackage{tcolorbox}
\usepackage{listings}
\usepackage{mathtools}
\usepackage{multicol}
\usepackage{amssymb}
\usepackage[hidelinks]{hyperref}
\usepackage[symbol]{footmisc}
\usepackage{tikz}
\usepackage[dvipsnames]{xcolor}

\newtheoremstyle{cooltheorem}    % name
    {\topsep}                    % Space above
    {\topsep}                    % Space below
    {}                           % Body font
    {}                           % Indent amount
    {\bfseries}                  % Theorem head font
    {.}                          % Punctuation after theorem head
    {.5em}                       % Space after theorem head
    {}  % Theorem head spec (can be left empty, meaning ‘normal’)

\theoremstyle{cooltheorem}
\newtheorem{problem}{Задача}
\theoremstyle{remark}
\newtheorem*{remark}{Забележка}

\setlength\parindent{0pt}
\setlength\mathindent{30pt}

\lstset{
xleftmargin=1cm,
numbersep=5pt,
numbers=left,
basicstyle=\normalfont,
tabsize=2,
keywordstyle=\bfseries,
morekeywords={if, else, for, return},
inputencoding=utf8,
escapechar=\%,
}

\title{Домашни по ТМ}
\author{Мартин Георгиев}
\date{02.01.2025}

\newcommand{\pow}{\mathcal{P}}
\newcommand{\union}{\bigcup}
\newcommand{\intersection}{\bigcap}
\newcommand{\pair}[1]{\langle #1 \rangle}
\renewcommand{\land}{\ \&\ }
\renewcommand{\lor}{\vee}
\newcommand{\size}[1]{\overline{\overline{#1}}}
\newcommand{\xor}{\oplus}

% Define the custom tcolorbox style
\tcbset{
  mybox/.style={
    colback=cyan!20,       % Background color
    colframe=cyan!80,     % Frame color
	coltitle=white,
    fonttitle=\bfseries,  % Title font style
    title={#1}, % Default title
    boxrule=0.5mm,          % Thickness of the frame
    arc=1mm,              % Rounded corners
    width=\linewidth,     % Box width
    left=1mm,             % Left padding
    right=1mm,            % Right padding
    top=1mm,              % Top padding
    bottom=1mm            % Bottom padding
  }
}

\begin{document}
\maketitle
% !TEX root = hw.tex

\begin{problem}
Функция $f$ се нарича \textit{двуместна}, ако Rel(Dom($f$)). Както обикновено,
$f(x, y)$ означава $f (\pair{x, y})$. Множество $M$ се нарича
\textit{затворено относно двуместна функция} $f$,
ако всеки път, когато $x_1, x_2 \in M$, следва $f(x_1, x_2) \in M$.
Нека $W$ е непразно множество и $f: \pow(\union W) \times \pow(\union W) \to \pow(\union W)$.
\begin{enumerate}
\item Да се докаже, че съществува единствено множество $V$ със свойствата:
$W \subseteq V$, $V$ е затворено относно $f$ и всеки път, когато $W \subseteq V'$
и $V'$ е затворено относно $f$, следва $V \subseteq V'$. Това множество $V$ ще
наричаме \textit{породено} от $W$ и $f$; означаваме го с $W_f$.

\textbf{Решение:}

\smallbreak
\quad
Нека разгледаме множеството $A \subseteq \pow(\pow(\union W))$ от затворени относно $f$ надмножества на $W$:
\[
	A = \{x\ |\ x \in \pow(\pow(\union W)) \land W \subseteq x \land \forall y\, \forall z\, [ y \in x \land z \in x \to f(y, z) \in x ] \}
\]
\quad
Твърдим, че $\intersection A$ е точно множеството $V$ от условието.

\begin{tcolorbox}[mybox={Доказателство:}]
\quad
Нека първо забележим, че $A$ не е празно, понеже $\pow(\union W) \in A$. Оттук следва, че и $\intersection A$ не е празно.

\quad
Щe докажем, че $W \subseteq \intersection A$. Нека $x \in W$.
От това, че $W$ е подмножество на всеки елемент на $A$, можем да заключим, че $\forall y\, [y \in A \to x \in y] \iff x \in \intersection A$.

\quad
Сега ще докажем, че $\intersection A$ е затворено относно $f$.
Нека допуснем, че това не е така, тоест съществуват $x_1, x_2 \in \intersection A$,
за които $f(x_1, x_2) \notin \intersection A$.
Нека $x_3 = f(x_1, x_2)$ и нека $B$ е произволно множество, такова че $B \in A \land x_3 \notin B$.
Тъй като $\intersection A \subseteq B$, то $x_1, x_2 \in B$, откъдето $B$ няма да е затровено, понеже
$f(x_1, x_2) = x_3 \notin B$. Щом $B$ не е затворено то със сигурност не принадлежи на $A$.
Така получихме, че $B$ едновременно принадлежи и не принадлежи на $A$, което е противоречие
$\Rightarrow$ $\intersection A$ е затворено относно $f$.

\quad
Накрая остава да съобразим, че
тъй като всяко затворено относно $f$ надмножество $V'$ на $W$ е надмножество на $\intersection A$,
$V$ е минимално по включване.

\qed
\end{tcolorbox}

\item
Да се докаже, че $X \in W_{\cup}$ точно тогава, когато съществува такова крайно и непразно $W_0 \subseteq W$, че $X = \union{W_0}$.

\textbf{Решение:}

\smallbreak
\quad
$(\Rightarrow)$
Нека $C$ е множеството от тези $X$, за които горното е изпълнено:
\[
	C = \{x\ |\ x \in W_{\cup} \land \text{съществува крайно и непразно множество $W_0 \subseteq W$, такова, че }\, x = \union W_0\}
\]
\quad
Лесно се забелязва, че $W$ е подмножество на $C$, тъй като $\forall x \in W : x = \union \{x\}$.

\quad
Твърдим, че освен това $C$ е затворено относно $\cup$.

\begin{tcolorbox}[mybox={Доказателство:}]
\quad
Да допуснем, че $C$ не е затворено.
Нека $x_1, x_2 \in C$ и $x_1 \cup x_2 \notin C$.
Нека $x_3 = x_1 \cup x_2$. Лесно се забелязва, че щом $W_{\cup}$ е затворено,
то $x_3 \in W_{\cup}$.
От друга страна $x_1, x_2 \in C \Rightarrow x_1 = \union W_1 \land x_2 = \union W_2$
за някакви крайни непразни подмножеста $W_1, W_2$ на $W$.

\quad
Нека сега разгледаме множеството $W_0 = W_1 \cup W_2$.
Ясно е, че то е крайно и непразно подмножество на $W$.
Остава да забележим, че $x_3$ е точно множеството $\union W_0$, тоест $x_3 \in C$, което е в противоречие с изведеното $x_3 \notin C$.

\qed
\end{tcolorbox}

\quad
И така, получихме, че $C$ е затворено относно $\cup$ подмножество на $W_{\cup}$.
Тъй като $W_{\cup}$ е породено от $W$ и $\cup$ и $W \subseteq C$, последното влече, че $C = W_{\cup}$.

\qed

% $(\Rightarrow)$
% Нека $C$ е множеството от тези $X$, за които горното не е изпълнено:
% \[
% 	C = \{x\ |\ x \in W_{\cup} \land \text{за всяко крайно и непразно множество $W_0 \subseteq W$},\, x \ne \union W_0\}
% \]
% \quad
% Твърдим, че $W_{\cup} \setminus C$ е затворено относно $\cup$.
%
% \bigbreak
% \textbf{Доказателство:}
%
% \quad
% Да допуснем, че $W_{\cup} \setminus C$ не е затворено.
% Нека $x_1, x_2 \in W_{\cup} \setminus C$ и $x_1 \cup x_2 \notin W_{\cup} \setminus C$.
% Нека $x_3 = x_1 \cup x_2$. Лесно се забелязва, че щом $W_{\cup}$ е затворено,
% то $x_3 \in W_{\cup}$, и щом $x_3 \notin W_{\cup} \setminus C$, то $x_3 \in C$.
% От друга страна $x_1, x_2 \in W_{\cup} \setminus C \Rightarrow x_1 = \union W_1 \land x_2 = \union W_2$
% за някакви крайни непразни подмножеста $W_1, W_2$ на $W$.
%
% \quad
% Нека сега разгледаме множеството $W_0 = W_1 \cup W_2$.
% Ясно е, че то е крайно и непразно подмножество на $W$.
% Остава да забележим, че $x_3$ е точно множеството $\union W_0$, тоест $x_3 \notin C$, което е в противоречие с изведеното $x_3 \in C$.
% \qed
%
% \quad
% И така, получихме, че $W_{\cup} \setminus C$ е затворено подмножество на $W_{\cup}$.
% Тъй като $W_{\cup}$ е породено от $W$ и $\cup$, последното влече, че $W_{\cup} = W_{\cup} \setminus C \Rightarrow C = \varnothing$.
% \qed

\bigbreak
\quad
$(\Leftarrow)$ Ще докажем обратната посока, с индукция по мощността на $W_0$:

\bigbreak
\textbf{База:} $\size{W_0} = 1$

Нека $w$ е такова, че $W_0 = \{w\}$.
Тогава $X = \union W_0 = w$, откъдето $X \in W_{\cup}$, понеже $w \in W_{0} \subseteq W \subseteq W_{\cup}$.

\textbf{Индукционна стъпка:} $\size{W_0} = k+1$

Нека $W'$ е произволно непразно строго подмножество на $W_0$.
По индукционно предположение, тъй като $\size{W'} < k+1$ и $\size{W_0 \setminus W'} < k+1$, то
$\union W' \in W_{\cup}$ и
$\union (W_0 \setminus W') \in W_{\cup}$
откъдето от затвореността на $W_{\cup}$ следва, че:
\[
\union W' \cup \union (W_0 \setminus W') \in W_{\cup} \iff \union W_0 \in W_{\cup} \iff X \in W_{\cup}.
\]

\qed


\item
Да се докаже, че $W_{\cup \cap} = W_{\cap \cup}$.

\textbf{Решение:}

\smallbreak
\quad
Нека първо забележим, че $X \in W_{\cap}$ точно тогава, когато съществува такова крайно и непразно $W_0 \subseteq W$, такова че $X = \intersection{W_0}$.
Доказателството на това твърдение е аналогично на доказателството в предната подточка. В сила е следното:
\begin{alignat*}{2}
&X \in W_{\cup \cap} & \iff & X = (x_1 \cap x_2 \cap \dots \cap x_k) \text{ за } x_1, x_2, \dots x_k \in W_{\cup} \land k \neq 0. \\
&X \in W_{\cup}      & \iff & X = (y_1 \cup y_2 \cup \dots \cup y_l) \text{ за } y_1, y_2, \dots y_l \in W \land l \neq 0. \\
\hline
\Rightarrow {} &X \in W_{\cup \cap} & \iff & X = ((y_{1,1} \cup y_{1,2} \cup \dots \cup y_{1,l_1}) \cap \dots \cap (y_{k,1} \cup y_{k,2} \cup \dots \cup y_{k,l_k}))
\text{ за } y_{i, j} \in W, k \neq 0, l_i \neq 0.
\end{alignat*}

\quad
И следното:
\begin{alignat*}{2}
&X \in W_{\cap \cup} & \iff & X = (x_1 \cup x_2 \cup \dots \cup x_k) \text{ за } x_1, x_2, \dots x_k \in W_{\cap} \land k \neq 0. \\
&X \in W_{\cap}      & \iff & X = (y_1 \cap y_2 \cap \dots \cap y_l) \text{ за } y_1, y_2, \dots y_l \in W \land l \neq 0. \\
\hline
\Rightarrow {} &X \in W_{\cap \cup} & \iff & X = ((y_{1,1} \cap y_{1,2} \cap \dots \cap y_{1,l_1}) \cup \dots \cup (y_{k,1} \cap y_{k,2} \cap \dots \cap y_{k,l_k}))
\text{ за } y_{i,j} \in W, k \neq 0, l_i \neq 0.
\end{alignat*}

\quad
Накрая остава да забележим, че новополучените условия за принадлежност към $W_{\cup \cap}$ и $W_{\cap \cup}$ са еквивалентни,
заради дистрибутивния закон над операциите $\cap$ и $\cup$. По точно, всяко множество от вида:
\[
(y_{1,1} \cup y_{1,2} \cup \dots \cup y_{1,l_1}) \cap \dots \cap (y_{k,1} \cup y_{k,2} \cup \dots \cup y_{k,l_k})
\]
може да се запише в алтернативен вид:
\[
(y_{1,1} \cap y_{2,1} \cap \dots \cap y_{k,1}) \cup \dots \cup (y_{1,l_1} \cap y_{2,l_2} \cap \dots \cap y_{k,l_k})
\]
и обратно.

\qed

\end{enumerate}
\end{problem}

\begin{problem}
\textbf{(ZF)}
Нека $I$ и $J$ са непразни множества, $\{I_j\}_{j \in J}$ е $J$-индексирана фамилия от непразни множества,
като $I = \union_{j \in J} I_j$. Нека
\[
K = \{L\ |\ L \in \pow(I) \land (\forall j \in J) (L \cap I_j \neq \varnothing)\}
\]

Да се докаже, че за всяка $I$-индексирана фамилия от множества $\{A_i\}_{i \in I}$ са в сила равенствата:
\[
\union_{j \in J} \intersection_{i \in I_j} A_i = \intersection_{L \in K} \union_{i \in L} A_i,
\]
\[
\intersection_{j \in J} \union_{i \in I_j} A_i = \union_{L \in K} \intersection_{i \in L} A_i
\]

\textbf{Решение:}
\smallbreak

\quad
Първо ще докажем, че:
\[
\union_{j \in J} \intersection_{i \in I_j} A_i \subseteq \intersection_{L \in K} \union_{i \in L} A_i
\]

\begin{tcolorbox}[mybox={Доказателство:}]
\quad
Нека \(\displaystyle x \in \union_{j \in J} \intersection_{i \in I_j} A_i\) и нека
вземем $k \in J$, такова че $\displaystyle x \in \intersection_{i \in I_k} A_i$. Тогава:
\begin{equation}
\forall i \in I_k\ [x \in A_i]
\end{equation}

\quad
Нека $L$ е произволен елемент на $K$ и нека $y \in L \cap I_k$.
От (1) $x \in A_y$, освен това знаем, че $\displaystyle y \in L \Rightarrow x \in \union_{i \in L} A_i$.

\quad
Щом $\displaystyle x \in \union_{i \in L} A_i$ за произволно $L \in K$,
то $\displaystyle x \in \intersection_{L \in K} \union_{i \in L} A_i$.

\qed
\end{tcolorbox}

\bigbreak
\quad
В обратната посока, ще докажем, че:
\[
\union_{j \in J} \intersection_{i \in I_j} A_i \supseteq \intersection_{L \in K} \union_{i \in L} A_i
\]
\begin{tcolorbox}[mybox={Доказателство:}]
\quad
Нека \(\displaystyle x \in \intersection_{L \in K} \union_{i \in L} A_i\).
Да допуснем, че
$\displaystyle x \notin \union_{j \in J} \intersection_{i \in I_j} A_i \iff
\forall j \in J\ [x \notin \intersection_{i \in I_j} A_i] \iff
\forall j \in J\ \exists k \in I_j \ [x \notin A_k]$.

\quad
Нека разгледаме множество $L$ от такива индекси $k$:
\[
L = \{k \ |\ k \in I \land x \notin A_k\}
\]

\quad
\textbf{Наблюдение 1:} $\displaystyle x \notin \union_{i \in L} A_i$

\quad
\textbf{Наблюдение 2:} $L \in K$.
Наистина, $L \in \pow(I)$ по дефиниция и освен това за произволно $j \in J$ e изпълнено, че $L \cap I_j \neq \varnothing$,
понеже по допускане $I_j$ съдържа елемент $k$, такъв че $x \notin A_k$.

\bigbreak
\quad
Щом $L \in K$ и $\displaystyle x \notin \union_{i \in L} A_i$, то
$\displaystyle x \notin \intersection_{L \in K} \union_{i \in L} A_i$.
Но това противоречи на избора на $x$, следователно допускането ни е грешно и
$\displaystyle \union_{j \in J} \intersection_{i \in I_j} A_i \supseteq \intersection_{L \in K} \union_{i \in L} A_i$.

\qed
\end{tcolorbox}

\bigbreak
\quad
Нека сега разгледаме второто равенство:
\[
\intersection_{j \in J} \union_{i \in I_j} A_i = \union_{L \in K} \intersection_{i \in L} A_i,
\]

\quad
Първо ще докажем, че:
\[
\intersection_{j \in J} \union_{i \in I_j} A_i \subseteq \union_{L \in K} \intersection_{i \in L} A_i
\]

\begin{tcolorbox}[mybox={Доказателство:}]
\quad
Нека
$\displaystyle x \in \intersection_{j \in J} \union_{i \in I_j} A_i$.
Тогава:
\begin{equation}
\forall j \in J \ [x \in \union_{i \in I_j} A_i] \iff
\forall j \in J \ \exists k \in I_j\ [x \in A_k]
\end{equation}

\quad
Нека разгледаме множество $L$ от такива индекси $k$:
\[
L = \{k \ |\ k \in I \land x \in A_k\}
\]

\quad
\textbf{Наблюдение 1:} $\displaystyle x \in \intersection_{i \in L} A_i$

\quad
\textbf{Наблюдение 2:} $L \in K$.
Наистина, $L \in \pow(I)$ по дефиниция и освен това, като следствие от (2),
за произволно $j \in J$ e изпълнено, че $L \cap I_j \neq \varnothing$.

\bigbreak
\quad
Щом $L \in K$ и $\displaystyle x \in \intersection_{i \in L} A_i$, то
$\displaystyle x \in \union_{L \in K} \intersection_{i \in L} A_i$.

\qed
\end{tcolorbox}

\bigbreak
\quad
В обратната посока, ще докажем, че:
\[
\intersection_{j \in J} \union_{i \in I_j} A_i \supseteq \union_{L \in K} \intersection_{i \in L} A_i
\]

\begin{tcolorbox}[mybox={Доказателство:}]
\quad
Нека $\displaystyle x \in \union_{L \in K} \intersection_{i \in L} A_i$ и нека вземем $L \in K$,
такова че $\displaystyle x \in \intersection_{i \in L} A_i$.
Нека $k$ е произволен елемент на $J$.
Твърдим, че $\displaystyle x \in \union_{i \in I_k} A_i$.
Като свидетел за това можем да вземем множеството $A_y$, където $y \in I_k \cap L$.

\quad
Щом $\displaystyle x \in \union_{i \in I_k} A_i$ за произволно $k \in J$,
то $\displaystyle x \in \intersection_{j \in J} \union_{i \in L} A_i$.

\qed
\end{tcolorbox}

\end{problem}

\begin{problem}
\textbf{(ZF)}
Нека $I \ne \varnothing$ и $\{A_i\}_{i \in I}$ е $I$-индексирана фамилия от множества.
Нека $\{I_j\}_{j \in J}$ е $J$-индексирана фамилия от взаимно чужди и непразни подмножества на $I$,
като $\union_{j \in J} I_j = I$.
Да се докаже, че множествата
$\prod_{i \in I} A_i$ и $\prod_{j \in J}(\prod_{i \in I_j} A_i)$ са равномощни.
\end{problem}

\textbf{Решение:}
\smallbreak
\quad
Нека функцията $r: \prod_{i \in I} A_i \to \prod_{j \in J}(\prod_{i \in I_j} A_i)$ е дефинирана по следния начин:
\[
r(f) = s \text{ за }
s(j) = t \text{ за }
t(i) = f(i)
\]
\quad
Ще покажем, че $r$ е инекция:

\begin{tcolorbox}[mybox={Доказателство:}]
\quad
Нека $r(f_1) = r(f_2)$ и нека $i$ е произволен елемент на $I$.
Нека вземем $j \in J$ такова, че $i \in I_j$.
Ясно е, че тъй като $r(f_1) = r(f_2)$, то $r(f_1)(j) = r(f_2)(j)$.
Накрая остава да съобразим, че от дефиницията на $r$,
$f_1(i) = r(f_1)(j)(i) = r(f_2)(j)(i) = f_2(i)$.
Така получихме, че за произволно $i \in I$ е в сила $f_1(i) = f_2(i)$, откъдето
$f_1 = f_2$.

\qed
\end{tcolorbox}

\quad
Ще покажем, че $r$ е сюрекция:

\begin{tcolorbox}[mybox={Доказателство:}]
\quad
Нека $s \in \prod_{j \in J}(\prod_{i \in I_j} A_i)$.
Дефинираме $f \in \prod_{i \in I} A_i$ по следния начин:
\[
f(i) = s(j)(i) \text{ за $j$ такова, че } i \in I_j
\]
\quad
Твъдим, че $r(f) = s$. Нека $j$ е произволен елемент на $J$.
Тогава $\forall i \in I_j\ [r(f)(j)(i) = f(i) = s(j)(i)]$, тоест $r(f)(j) = s(j)$.
Така получихме, че за произволно $j \in J$ е в сила
$r(f)(j) = s(j)$, откъдето $r(f) = s$.

\qed
\end{tcolorbox}


\quad
Щом $r$ е инекция и сюрекция, то $r$ е биекция, откъдето
$\size{\prod_{i \in I} A_i} = \size{\prod_{j \in J}(\prod_{i \in I_j} A_i)}$.

\qed

\begin{problem}
\textbf{(ZF)}
Нека $I \ne \varnothing$ и $\{A_i\}_{i \in I}$ е $I$-индексирана фамилия от взаимно чужди множества.
Да се докаже, че множествата $\prod_{i \in I} ({}^{A_i} B)$ и ${}^{(\union_{i \in I} A_i)} B$ са равномощни.
\end{problem}

\textbf{Решение:}

\smallbreak
\quad
Нека функцията
$r: \prod_{i \in I} ({}^{A_i} B) \to {}^{(\union_{i \in I} A_i)} B$
е дефинирана по следния начин:
\[
r(f) = s \text{ за }
s(a) = f(i)(a) \text{ за $i$ такова, че $a \in A_i$ }
\]

\quad
Ще покажем, че $r$ е инекция:

\begin{tcolorbox}[mybox={Доказателство:}]
\quad
Нека $r(f_1) = r(f_2)$ и нека $i$ е произволен елемент на $I$.
Тогава от равенството на $r(f_1)$ и $r(f_2)$ и от дефиницията на $r$ в сила ще бъде:
$\forall a \in A_i\ [ f_1(i)(a) = r(f_1)(a) = r(f_2)(a) = f_2(i)(a)]$,
откъдето $f_1(i) = f_2(i)$.
Така получихме, че за произволно $i \in I$ е в сила $f_1(i) = f_2(i)$, откъдето
$f_1 = f_2$.

\qed
\end{tcolorbox}

\quad
Ще покажем, че $r$ е сюрекция:

\begin{tcolorbox}[mybox={Доказателство:}]
\quad
Нека $s \in {}^{(\union_{i \in I} A_i)} B$.
Дефинираме $f \in \prod_{i \in I} ({}^{A_i} B)$ по следния начин:
\[
f(i) = g \text{ за }
g(a) = s(a)
\]
\quad
Твъдим, че $r(f) = s$.
Нека $a$ е произволен елемент на $\union_{i \in I} A_i$ и нека $i$ е такова, че $a \in A_i$.
Тогава от дефинициите на $r$ и $f$ ще бъде изпълнено, че $r(f)(a) = f(i)(a) = g(a) = s(a)$.
Така получихме, че за произволно $a \in \union_{i \in I} A_i$ е в сила $r(f)(a) = s(a)$, откъдето $r(f) = s$.

\qed
\end{tcolorbox}


\quad
Щом $r$ е инекция и сюрекция, то $r$ е биекция, откъдето
$\size{\prod_{i \in I} ({}^{A_i} B)} = \size{{}^{(\union_{i \in I} A_i)} B}$.

\qed

\begin{problem}
Да се докаже, че за произволно множество $A$ са в сила следните:
\begin{enumerate}
\item
$\size{A} = \size{A \cup \{A\}} \Rightarrow \size{\pow(A)} = \size{\pow(A) \cup \{\pow(A)\}}$
\item
$\size{A} = \size{A \cup \{A\}} \Rightarrow \size{\pow(\pow(A))} = \size{\pow(\pow(A)) \times \pow(\pow(A))}$
\end{enumerate}
%\textbf{Упътване.}
%\textit{За 2) използвайте, че ${}^X 2$ и $\pow(X)$ са равномощни, както и че винаги,
%когато $B \cap C = \varnothing$, множествата ${}^{B \cup C} X$ и ${}^B X \times {}^C X$ са равномощни.}
\end{problem}

\textbf{Решение:}

\smallbreak
\quad
Първо ще докажем 1). Нека $\size{A} = \size{A \cup \{A\}}$ и нека $f: A \to A \cup \{A\}$ е произволна биекция.
Дефинираме функция $g: \pow(A) \to \pow(A) \cup \{\pow(A)\}$ по следния начин:
\[
g(x) =
\begin{cases}
\pow(A) & \text{, ако $\exists y\ [x = \{y\} \land f(y) = A]$} \\
\{f(\union x)\} & \text{, иначе, ако $\exists y\ [x = \{y\}]$} \\
x & \text{, иначе}
\end{cases}
\]

\quad
Ще докажем, че $g$ е инекция:
\begin{tcolorbox}[mybox={Доказателство:}]
\quad
Нека $g(x_1) = g(x_2)$. Разглеждаме четири случая:
\begin{enumerate}[label={\arabic* сл.}]
\item
$g(x_1) = \pow(A) = g(x_2)$.
Тогава
$\exists y_1\ [x_1 = \{y_1\} \land f(y_1) = A] \land
\exists y_2\ [x_2 = \{y_2\} \land f(y_2) = A]$.
Тъй като $f$ е инекция, последното влече, че $y_1 = y_2$, откъдето $x_1 = x_2$.

\item
$g(x_1) \ne \pow(A) \ne g(x_2) \land \exists y_1\ [x_1 = \{y_1\}] \land \exists y_2\ [x_2 = \{y_2\}]$.
Тогава $g(x_1) = \{f(y_1)\}$ и $g(x_2) = \{f(y_2)\}$.
Тъй като $g(x_1) = g(x_2)$, последното влече, че
$f(y_1) = f(y_2)$, откъдето $y_1 = y_2 $ и $x_1 = \{y_1\} = \{y_2\} = x_2$

\item
$\exists y_1\ [x_1 = \{y_1\}] \xor \exists y_2\ [x_2 = \{y_2\}]$.
Но това влече, че $\size{g(x_1)} \ne \size{g(x_2)}$, което е невъзможно.

\item
$\neg \exists y_1\ [x_1 = \{y_1\}] \land \neg \exists y_2\ [x_2 = \{y_2\}]$.
Тогава $x_1 = g(x_1) = g(x_2) = x_2$.
\end{enumerate}
\qed


\end{tcolorbox}

\quad
Ще докажем, че $g$ е сюрекция:
\begin{tcolorbox}[mybox={Доказателство:}]
\quad
Нека $y \in \pow(A) \cup \{\pow(A)\}$.
Разглеждаме три случая:
\begin{enumerate}[label={\arabic* сл.}]
\item
$y = \pow(A)$. Тогава можем да вземем като първообраз на $y$ множество $x \in \pow(A)$ от вида $x = \{z\}$,
където $z \in A \land f(z) = \{A\}$.

\item
$y \in \pow(A) \land \exists z\ [y = \{z\}]$
Нека $s = f^{-1}(z)$. Тогава $g(\{s\}) = \{f(\union \{s\})\} = \{f(s)\} = \{z\} = y$.

\item
$y \in \pow(A) \land \neg \exists z\ [y = \{z\}]$.
Тогава $g(y) = y$.
\end{enumerate}
\qed
\end{tcolorbox}

\quad
Щом $g$ е инекция и сюрекция, то $g$ е биекция $\Rightarrow$
$\size{\pow(A)} = \size{\pow(A) \cup \{\pow(A)\}}$.

\qed

% \quad
% Нека $t$ е произволен елемент, такъв, че $t \notin Rng(\pow(A))$.
% За да докажем 2) е достатъчно е да построим биекция $h: \pow(\pow(A)) \to \pow(\pow(A) \cup (\pow(A) \times \{t\}))$,
% тъй като
% \begin{alignat*}{3}
% \size{\pow(\pow(A)) \times \pow(\pow(A))} &= \size{{}^{\pow(A)} 2 \times {}^{\pow(A)} 2}  && \text{ // от упътването} &\\
%                                           &= \size{{}^{\pow(A)} 2 \times {}^{\pow(A) \times \{t\}} 2}  && \text{ // от свойствата на декартовото произведение} &\\
% 										  & = \size{{}^{\pow(A) \cup (\pow(A) \times \{t\})} 2} && \text{ // от упътването} & \\
% 										  & = \size{\pow(\pow(A) \cup (\pow(A) \times \{t\}))} && \text{ // от упътването} &
% \end{alignat*}
%
% Дефинираме $h: \pow(\pow(A)) \to \pow(\pow(A) \cup (\pow(A) \times \{t\}))$ по следния начин:
% %\[
% %h(x) = \{
% %y\ |\ y \in \pow(\pow(A) \cup (\pow(A) \times \{t\})) \land
% %( \exists a \in x\ \forall v\ [v \in a \iff (f(v) \in y \xor f(v) = A)])
% %\}
% %\]
% % \begin{alignat*}{1}
% % h(x) = \{ & y\ |\ y \in \pow(\pow(A) \cup (\pow(A) \times \{t\})) \land \exists a \in x[ \\
% %           & (\forall v\ [(v \in a \Rightarrow f(v) \ne A) \land (v \in a \iff f(v) \in y)]) \lor \\
% %           & (y \subseteq \pow(A) \times \{t\} \land ( \exists w \in a\ [f(w) = A \land \forall v\ [v \in a \setminus\{w\} \iff f(v) \in Dom(y)]]))]
% % \}
% % \end{alignat*}
% \begin{alignat*}{1}
% h(x) = \{ & y\ |\ y \in \pow(\pow(A) \cup (\pow(A) \times \{t\})) \land \forall z \in y\ \exists p \in x\ [ \\
%           & (\forall v\ [v \in p \iff f(v) \in z]) \lor \\
% 		  & (Rng(z) = t \land \exists v \in p\ [f(v) = A ] \land \forall v\ [v \in p \iff f(v) \in Dom(z) \xor f(v) = A])]
% \}
% \end{alignat*}
%
% Ще докажем, че $h$ е сюрекция:
%
% \begin{tcolorbox}[mybox={Доказателство:}]
% \quad
% Нека $y \in \pow(\pow(A) \cup (\pow(A) \times \{t\}))$. Нека:
% \[
% Z_1 = \{x\ |\ x \in y \land x \in \pow(A)\}
% \]
% \[
% Z_2 = \{x\ |\ x \in y \land x \in \pow(A) \times {t}\}
% \]
% Забележете, че $\{Z_1, Z_2\}$ е разбиване на $y$.
%
% \smallbreak
% Нека $x$ е следното множество:
% \[
% x = \{a\ |\ a \in \pow(A) \land [f[a] \in y \xor (\exists b \in a\ [f(b) = A \land f[a] \setminus b \in Dom(y)])] \}
% \]
% Твърдим, че $x$ е първообраз на $y$. Наистина
%
% \bigbreak
% Разглеждаме четири случая:
% %\begin{enumerate}[label={\arabic* сл.}]
% %
% %\end{enumerate}
% \qed
%
% \end{tcolorbox}
%
% Ще докажем, че $h$ е инекция.
%
% \begin{tcolorbox}[mybox={Доказателство:}]
% \quad
% Нека $h(x_1) = h(x_2)$. Разглеждаме четири случая:
% \qed
%
% \end{tcolorbox}

% Нека сега докажем 2). Дефинираме $h: \pow(\pow(A)) \to \pow(\pow(A)) \times \pow(\pow(A))$ по следния начин:
% \[
% h(x) = \pair{y, z} \text{ за } y = \{a \in \pow(A)\ |\ f^{-1}[a] \in x\} \land z = \{b \in \pow(A) \ |\ f^{-1}[b \cup \{A\}] \in x\}
% \]

\bigbreak
\quad
Сега ще докажем 2). Нека $c = f^{-1}(A)$. Дефинираме функция $h: \pow(\pow(A)) \to \pow(\pow(A)) \times \pow(\pow(A))$ по следния начин:
\[
h(x) = \pair{y, z} \text{ за } y = \{f[a]\ |\ a \in x \land c \notin a\} \land z = \{f[a \setminus \{c\}]\ |\ a \in x \land c \in a\}
\]
\quad
Ще докажем, че $h$ е инекция:

\begin{tcolorbox}[mybox={Доказателство:}]
\quad
Нека $h(x_1) = \pair{y, z} = h(x_2)$. Тогава:
\[
\{f[a]\ |\ a \in x_1 \land c \notin a\} = y = \{f[a]\ |\ a \in x_2 \land c \notin a\}
\]
\[
\{f[a \setminus \{c\}]\ |\ a \in x_1 \land c \in a\} = z = \{f[a \setminus \{c\}]\ |\ a \in x_2 \land c \in a\}
\]
\quad
Нека $b$ е произволно множество. Разглеждаме два случая:
\begin{enumerate}[label={\arabic* сл.}]
\item
$c \notin b$. Тогава:
\[
b \in x_1 \iff f[b] \in y \iff b \in x_2
\]
\item
$c \in b$. Тогава:
\[
b \in x_1 \iff f[b \setminus \{c\}] \in z \iff b \in x_2
\]
\end{enumerate}
\quad
Така от аксиомата за обемност $x_1 = x_2$.

\qed
\end{tcolorbox}

\quad
Ще докажем, че $h$ е сюрекция:

\begin{tcolorbox}[mybox={Доказателство:}]
\quad
Нека $\pair{y, z} \in \pow(\pow(A)) \cup \pow(\pow(A))$. Дефинираме множество първообраз $x$ по следния начин:
\[
x = \underbrace{\{f^{-1}[b]\ |\ b \in y\}}_{M} \uplus \underbrace{\{f^{-1}[b] \cup \{c\}\ |\ b \in z\}}_{N}
\]
\quad
Отсега отбелязваме, че $M$ и $N$ са чужди множества, тъй като
$\forall x\ [x \in M \Rightarrow c \notin x] \land [x \in N \Rightarrow c \in x]$.
Нека $h(x) = \pair{y', z'}$.
В сила е следното:
\begin{alignat*}{2}
b \in y & \iff f^{-1}[b] \in x      & \text{ // от $f^{-1}[b] \notin N$ и дефиницията на $x$} \\
        & \iff f[f^{-1}[b]] \in y'  & \text{ // от $c \notin f^{-1}[b]$ и дефиницията на $h$} \\
		& \iff b \in y'             &
\end{alignat*}
\quad
По същия начин:
\begin{alignat*}{2}
b \in z & \iff f^{-1}[b] \cup \{c\} \in x                         & \text{ // от $f^{-1}[b] \cup \{c\} \notin M$ и дефиницията на $x$} \\
        & \iff f[(f^{-1}[b] \cup \{c\}) \setminus \{c\}] \in z'   & \text{ // от $c \in f^{-1}[b] \cup \{c\}$ и дефиницията на $h$} \\
        & \iff f[f^{-1}[b]] \in z'                                & \text{ // от $c \notin f^{-1}[b]$} \\
		& \iff b \in z'                                           &
\end{alignat*}

\quad
Така получихме, че $h(x) = \pair{y, z} \Rightarrow \forall \pair{y, z}\, \exists x\ [h(x) = \pair{y, z}]$.

\qed
\end{tcolorbox}

\quad
Щом $h$ е инекция и сюрекция, то $h$ е биекция
$\Rightarrow \size{\pow(\pow(A))} = \size{\pow(\pow(A)) \times \pow(\pow(A))}$.

\qed


%\begin{tcolorbox}[mybox={Доказателство:}]
%\quad
%Нека $h(x_1) = h(x_2)$. Разглеждаме четири случая:
%\qed

%\end{tcolorbox}


% \textbf{Въпрос с повишена трудност (неразрешим и неполуразрешим):}
% Кому е полезно каквото и да е от това? Защо този предмет е задължително избираем за Компютърни Науки?
% Не мисля, че съм писал по безсмислено домашно през живота си.
\end{document}
