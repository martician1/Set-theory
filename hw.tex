\documentclass[a4paper, fleqn]{article}
\usepackage[utf8]{inputenc}
\usepackage[T1, T2A]{fontenc}
\usepackage[english,bulgarian]{babel}
\usepackage{amsthm}
\usepackage{amsmath}
\usepackage{amsfonts}
\usepackage[a4paper, left=0.50in, right=0.50in, top=0.5in, bottom=1.0in]{geometry}
\usepackage{enumitem}
\usepackage{tcolorbox}
\usepackage{listings}
\usepackage{mathtools}
\usepackage{multicol}
\usepackage{amssymb}
\usepackage[hidelinks]{hyperref}
\usepackage[symbol]{footmisc}
\usepackage{tikz}
\usepackage[dvipsnames]{xcolor}

\newtheoremstyle{cooltheorem}    % name
    {\topsep}                    % Space above
    {\topsep}                    % Space below
    {}                           % Body font
    {}                           % Indent amount
    {\bfseries}                  % Theorem head font
    {.}                          % Punctuation after theorem head
    {.5em}                       % Space after theorem head
    {}  % Theorem head spec (can be left empty, meaning ‘normal’)

\theoremstyle{cooltheorem}
\newtheorem{problem}{Задача}
\theoremstyle{remark}
\newtheorem*{remark}{Забележка}

\setlength\parindent{0pt}
\setlength\mathindent{30pt}

\lstset{
xleftmargin=1cm,
numbersep=5pt,
numbers=left,
basicstyle=\normalfont,
tabsize=2,
keywordstyle=\bfseries,
morekeywords={if, else, for, return},
inputencoding=utf8,
escapechar=\%,
}

\title{Домашни по ТМ}
\author{Мартин Георгиев}
\date{$\backslash \{u\ _\smallsmile \  u\}/$}

\newcommand{\pow}{\mathcal{P}}
\newcommand{\union}{\bigcup}
\newcommand{\intersection}{\bigcap}
\newcommand{\pair}[1]{\langle #1 \rangle}
\renewcommand{\land}{\ \&\ }
\renewcommand{\lor}{\ \vee\ }
\newcommand{\size}[1]{\overline{\overline{#1}}}
\newcommand{\xor}{\oplus}
\newcommand{\bijarrow}{\,\mathrlap{\rightarrowtail}\twoheadrightarrow}

% Define the custom tcolorbox style
\tcbset{
  mybox/.style={
	colback=cyan!20,       % Background color
    colframe=cyan!80,     % Frame color
	coltitle=white,
    fonttitle=\bfseries,  % Title font style
    title={#1}, % Default title
    boxrule=0.5mm,          % Thickness of the frame
    arc=1mm,              % Rounded corners
    width=\linewidth,     % Box width
    left=1mm,             % Left padding
    right=1mm,            % Right padding
    top=1mm,              % Top padding
    bottom=1mm            % Bottom padding
  }
}

\begin{document}
\maketitle
% !TEX root = hw.tex

\begin{problem}
Функция $f$ се нарича \textit{двуместна}, ако Rel(Dom($f$)). Както обикновено,
$f(x, y)$ означава $f (\pair{x, y})$. Множество $M$ се нарича
\textit{затворено относно двуместна функция} $f$,
ако всеки път, когато $x_1, x_2 \in M$, следва $f(x_1, x_2) \in M$.
Нека $W$ е непразно множество и $f: \pow(\union W) \times \pow(\union W) \to \pow(\union W)$.
\begin{enumerate}
\item Да се докаже, че съществува единствено множество $V$ със свойствата:
$W \subseteq V$, $V$ е затворено относно $f$ и всеки път, когато $W \subseteq V'$
и $V'$ е затворено относно $f$, следва $V \subseteq V'$. Това множество $V$ ще
наричаме \textit{породено} от $W$ и $f$; означаваме го с $W_f$.

\textbf{Решение:}

\smallbreak
\quad
Нека разгледаме множеството $A \subseteq \pow(\pow(\union W))$ от затворени относно $f$ надмножества на $W$:
\[
	A = \{x\ |\ x \in \pow(\pow(\union W)) \land W \subseteq x \land \forall y\, \forall z\, [ y \in x \land z \in x \to f(y, z) \in x ] \}
\]
\quad
Твърдим, че $\intersection A$ е точно множеството $V$ от условието.

\begin{tcolorbox}[mybox={Доказателство:}]
\quad
Нека първо забележим, че $A$ не е празно, понеже $\pow(\union W) \in A$. Оттук следва, че и $\intersection A$ не е празно.

\quad
Щe докажем, че $W \subseteq \intersection A$. Нека $x \in W$.
От това, че $W$ е подмножество на всеки елемент на $A$, можем да заключим, че $\forall y\, [y \in A \to x \in y] \iff x \in \intersection A$.

\quad
Сега ще докажем, че $\intersection A$ е затворено относно $f$.
Нека допуснем, че това не е така, тоест съществуват $x_1, x_2 \in \intersection A$,
за които $f(x_1, x_2) \notin \intersection A$.
Нека $x_3 = f(x_1, x_2)$ и нека $B$ е произволно множество, такова че $B \in A \land x_3 \notin B$.
Тъй като $\intersection A \subseteq B$, то $x_1, x_2 \in B$, откъдето $B$ няма да е затровено, понеже
$f(x_1, x_2) = x_3 \notin B$. Щом $B$ не е затворено то със сигурност не принадлежи на $A$.
Така получихме, че $B$ едновременно принадлежи и не принадлежи на $A$, което е противоречие
$\Rightarrow$ $\intersection A$ е затворено относно $f$.

\quad
Накрая остава да съобразим, че
тъй като всяко затворено относно $f$ надмножество $V'$ на $W$ е надмножество на $\intersection A$,
$V$ е минимално по включване.

\qed
\end{tcolorbox}

\item
Да се докаже, че $X \in W_{\cup}$ точно тогава, когато съществува такова крайно и непразно $W_0 \subseteq W$, че $X = \union{W_0}$.

\textbf{Решение:}

\smallbreak
\quad
$(\Rightarrow)$
Нека $C$ е множеството от тези $X$, за които горното е изпълнено:
\[
	C = \{x\ |\ x \in W_{\cup} \land \text{съществува крайно и непразно множество $W_0 \subseteq W$, такова, че }\, x = \union W_0\}
\]
\quad
Лесно се забелязва, че $W$ е подмножество на $C$, тъй като $\forall x \in W : x = \union \{x\}$.

\quad
Твърдим, че освен това $C$ е затворено относно $\cup$.

\begin{tcolorbox}[mybox={Доказателство:}]
\quad
Да допуснем, че $C$ не е затворено.
Нека $x_1, x_2 \in C$ и $x_1 \cup x_2 \notin C$.
Нека $x_3 = x_1 \cup x_2$. Лесно се забелязва, че щом $W_{\cup}$ е затворено,
то $x_3 \in W_{\cup}$.
От друга страна $x_1, x_2 \in C \Rightarrow x_1 = \union W_1 \land x_2 = \union W_2$
за някакви крайни непразни подмножеста $W_1, W_2$ на $W$.

\quad
Нека сега разгледаме множеството $W_0 = W_1 \cup W_2$.
Ясно е, че то е крайно и непразно подмножество на $W$.
Остава да забележим, че $x_3$ е точно множеството $\union W_0$, тоест $x_3 \in C$, което е в противоречие с изведеното $x_3 \notin C$.

\qed
\end{tcolorbox}

\quad
И така, получихме, че $C$ е затворено относно $\cup$ подмножество на $W_{\cup}$.
Тъй като $W_{\cup}$ е породено от $W$ и $\cup$ и $W \subseteq C$, последното влече, че $C = W_{\cup}$.

\qed

% $(\Rightarrow)$
% Нека $C$ е множеството от тези $X$, за които горното не е изпълнено:
% \[
% 	C = \{x\ |\ x \in W_{\cup} \land \text{за всяко крайно и непразно множество $W_0 \subseteq W$},\, x \ne \union W_0\}
% \]
% \quad
% Твърдим, че $W_{\cup} \setminus C$ е затворено относно $\cup$.
%
% \bigbreak
% \textbf{Доказателство:}
%
% \quad
% Да допуснем, че $W_{\cup} \setminus C$ не е затворено.
% Нека $x_1, x_2 \in W_{\cup} \setminus C$ и $x_1 \cup x_2 \notin W_{\cup} \setminus C$.
% Нека $x_3 = x_1 \cup x_2$. Лесно се забелязва, че щом $W_{\cup}$ е затворено,
% то $x_3 \in W_{\cup}$, и щом $x_3 \notin W_{\cup} \setminus C$, то $x_3 \in C$.
% От друга страна $x_1, x_2 \in W_{\cup} \setminus C \Rightarrow x_1 = \union W_1 \land x_2 = \union W_2$
% за някакви крайни непразни подмножеста $W_1, W_2$ на $W$.
%
% \quad
% Нека сега разгледаме множеството $W_0 = W_1 \cup W_2$.
% Ясно е, че то е крайно и непразно подмножество на $W$.
% Остава да забележим, че $x_3$ е точно множеството $\union W_0$, тоест $x_3 \notin C$, което е в противоречие с изведеното $x_3 \in C$.
% \qed
%
% \quad
% И така, получихме, че $W_{\cup} \setminus C$ е затворено подмножество на $W_{\cup}$.
% Тъй като $W_{\cup}$ е породено от $W$ и $\cup$, последното влече, че $W_{\cup} = W_{\cup} \setminus C \Rightarrow C = \varnothing$.
% \qed

\bigbreak
\quad
$(\Leftarrow)$ Ще докажем обратната посока, с индукция по мощността на $W_0$:

\bigbreak
\textbf{База:} $\size{W_0} = 1$

Нека $w$ е такова, че $W_0 = \{w\}$.
Тогава $X = \union W_0 = w$, откъдето $X \in W_{\cup}$, понеже $w \in W_{0} \subseteq W \subseteq W_{\cup}$.

\textbf{Индукционна стъпка:} $\size{W_0} = k+1$

Нека $W'$ е произволно непразно строго подмножество на $W_0$.
По индукционно предположение, тъй като $\size{W'} < k+1$ и $\size{W_0 \setminus W'} < k+1$, то
$\union W' \in W_{\cup}$ и
$\union (W_0 \setminus W') \in W_{\cup}$
откъдето от затвореността на $W_{\cup}$ следва, че:
\[
\union W' \cup \union (W_0 \setminus W') \in W_{\cup} \iff \union W_0 \in W_{\cup} \iff X \in W_{\cup}.
\]

\qed


\item
Да се докаже, че $W_{\cup \cap} = W_{\cap \cup}$.

\textbf{Решение:}

\smallbreak
\quad
Нека първо забележим, че $X \in W_{\cap}$ точно тогава, когато съществува такова крайно и непразно $W_0 \subseteq W$, такова че $X = \intersection{W_0}$.
Доказателството на това твърдение е аналогично на доказателството в предната подточка. В сила е следното:
\begin{alignat*}{2}
&X \in W_{\cup \cap} & \iff & X = (x_1 \cap x_2 \cap \dots \cap x_k) \text{ за } x_1, x_2, \dots x_k \in W_{\cup} \land k \neq 0. \\
&X \in W_{\cup}      & \iff & X = (y_1 \cup y_2 \cup \dots \cup y_l) \text{ за } y_1, y_2, \dots y_l \in W \land l \neq 0. \\
\hline
\Rightarrow {} &X \in W_{\cup \cap} & \iff & X = ((y_{1,1} \cup y_{1,2} \cup \dots \cup y_{1,l_1}) \cap \dots \cap (y_{k,1} \cup y_{k,2} \cup \dots \cup y_{k,l_k}))
\text{ за } y_{i, j} \in W, k \neq 0, l_i \neq 0.
\end{alignat*}

\quad
И следното:
\begin{alignat*}{2}
&X \in W_{\cap \cup} & \iff & X = (x_1 \cup x_2 \cup \dots \cup x_k) \text{ за } x_1, x_2, \dots x_k \in W_{\cap} \land k \neq 0. \\
&X \in W_{\cap}      & \iff & X = (y_1 \cap y_2 \cap \dots \cap y_l) \text{ за } y_1, y_2, \dots y_l \in W \land l \neq 0. \\
\hline
\Rightarrow {} &X \in W_{\cap \cup} & \iff & X = ((y_{1,1} \cap y_{1,2} \cap \dots \cap y_{1,l_1}) \cup \dots \cup (y_{k,1} \cap y_{k,2} \cap \dots \cap y_{k,l_k}))
\text{ за } y_{i,j} \in W, k \neq 0, l_i \neq 0.
\end{alignat*}

\quad
Накрая остава да забележим, че новополучените условия за принадлежност към $W_{\cup \cap}$ и $W_{\cap \cup}$ са еквивалентни,
заради дистрибутивния закон над операциите $\cap$ и $\cup$. По точно, всяко множество от вида:
\[
(y_{1,1} \cup y_{1,2} \cup \dots \cup y_{1,l_1}) \cap \dots \cap (y_{k,1} \cup y_{k,2} \cup \dots \cup y_{k,l_k})
\]
може да се запише в алтернативен вид:
\[
(y_{1,1} \cap y_{2,1} \cap \dots \cap y_{k,1}) \cup \dots \cup (y_{1,l_1} \cap y_{2,l_2} \cap \dots \cap y_{k,l_k})
\]
и обратно.

\qed

\end{enumerate}
\end{problem}

\begin{problem}
\textbf{(ZF)}
Нека $I$ и $J$ са непразни множества, $\{I_j\}_{j \in J}$ е $J$-индексирана фамилия от непразни множества,
като $I = \union_{j \in J} I_j$. Нека
\[
K = \{L\ |\ L \in \pow(I) \land (\forall j \in J) (L \cap I_j \neq \varnothing)\}
\]

Да се докаже, че за всяка $I$-индексирана фамилия от множества $\{A_i\}_{i \in I}$ са в сила равенствата:
\[
\union_{j \in J} \intersection_{i \in I_j} A_i = \intersection_{L \in K} \union_{i \in L} A_i,
\]
\[
\intersection_{j \in J} \union_{i \in I_j} A_i = \union_{L \in K} \intersection_{i \in L} A_i
\]

\textbf{Решение:}
\smallbreak

\quad
Първо ще докажем, че:
\[
\union_{j \in J} \intersection_{i \in I_j} A_i \subseteq \intersection_{L \in K} \union_{i \in L} A_i
\]

\begin{tcolorbox}[mybox={Доказателство:}]
\quad
Нека \(\displaystyle x \in \union_{j \in J} \intersection_{i \in I_j} A_i\) и нека
вземем $k \in J$, такова че $\displaystyle x \in \intersection_{i \in I_k} A_i$. Тогава:
\begin{equation}
\forall i \in I_k\ [x \in A_i]
\end{equation}

\quad
Нека $L$ е произволен елемент на $K$ и нека $y \in L \cap I_k$.
От (1) $x \in A_y$.
Освен това знаем, че $\displaystyle y \in L$, откъдето $\displaystyle x \in \union_{i \in L} A_i$.

\quad
Щом $\displaystyle x \in \union_{i \in L} A_i$ за произволно $L \in K$,
то $\displaystyle x \in \intersection_{L \in K} \union_{i \in L} A_i$.

\qed
\end{tcolorbox}

\bigbreak
\quad
В обратната посока, ще докажем, че:
\[
\union_{j \in J} \intersection_{i \in I_j} A_i \supseteq \intersection_{L \in K} \union_{i \in L} A_i
\]
\begin{tcolorbox}[mybox={Доказателство:}]
\quad
Нека \(\displaystyle x \in \intersection_{L \in K} \union_{i \in L} A_i\).
Да допуснем, че
$\displaystyle x \notin \union_{j \in J} \intersection_{i \in I_j} A_i$
тоест
$\displaystyle \forall j \in J\ [x \notin \intersection_{i \in I_j} A_i]$
тоест
$\displaystyle \forall j \in J\ \exists k \in I_j \ [x \notin A_k]$.

\quad
Нека разгледаме множество $L$ от такива индекси $k$:
\[
L = \{k \ |\ k \in I \land x \notin A_k\}
\]

\quad
\textbf{Наблюдение 1:} $\displaystyle x \notin \union_{i \in L} A_i$

\quad
\textbf{Наблюдение 2:} $L \in K$.
Наистина, $L \in \pow(I)$ по дефиниция и освен това за произволно $j \in J$ e изпълнено, че $L \cap I_j \neq \varnothing$,
понеже по допускане $I_j$ съдържа елемент $k$, такъв че $x \notin A_k$.

\bigbreak
\quad
Щом $L \in K$ и $\displaystyle x \notin \union_{i \in L} A_i$, то
$\displaystyle x \notin \intersection_{L \in K} \union_{i \in L} A_i$.
Но това противоречи на избора на $x$, следователно допускането ни е грешно и
$\displaystyle \union_{j \in J} \intersection_{i \in I_j} A_i \supseteq \intersection_{L \in K} \union_{i \in L} A_i$.

\qed
\end{tcolorbox}

\bigbreak
\quad
Нека сега разгледаме второто равенство:
\[
\intersection_{j \in J} \union_{i \in I_j} A_i = \union_{L \in K} \intersection_{i \in L} A_i,
\]

\quad
Първо ще докажем, че:
\[
\intersection_{j \in J} \union_{i \in I_j} A_i \subseteq \union_{L \in K} \intersection_{i \in L} A_i
\]

\begin{tcolorbox}[mybox={Доказателство:}]
\quad
Нека
$\displaystyle x \in \intersection_{j \in J} \union_{i \in I_j} A_i$.
Тогава:
\begin{equation}
%\forall j \in J \ [x \in \union_{i \in I_j} A_i] \iff
\forall j \in J \ \exists k \in I_j\ [x \in A_k]
\end{equation}

\quad
Нека разгледаме множество $L$ от такива индекси $k$:
\[
L = \{k \ |\ k \in I \land x \in A_k\}
\]

\quad
\textbf{Наблюдение 1:} $\displaystyle x \in \intersection_{i \in L} A_i$

\quad
\textbf{Наблюдение 2:} $L \in K$.
Наистина, $L \in \pow(I)$ по дефиниция и освен това, като следствие от (2),
за произволно $j \in J$ e изпълнено, че $L \cap I_j \neq \varnothing$.

\bigbreak
\quad
Щом $L \in K$ и $\displaystyle x \in \intersection_{i \in L} A_i$, то
$\displaystyle x \in \union_{L \in K} \intersection_{i \in L} A_i$.

\qed
\end{tcolorbox}

\bigbreak
\quad
В обратната посока, ще докажем, че:
\[
\intersection_{j \in J} \union_{i \in I_j} A_i \supseteq \union_{L \in K} \intersection_{i \in L} A_i
\]

\begin{tcolorbox}[mybox={Доказателство:}]
\quad
Нека $\displaystyle x \in \union_{L \in K} \intersection_{i \in L} A_i$ и нека вземем $L \in K$,
такова че $\displaystyle x \in \intersection_{i \in L} A_i$.
Нека $k$ е произволен елемент на $J$.
Твърдим, че $\displaystyle x \in \union_{i \in I_k} A_i$.
Като свидетел за това можем да вземем множество $A_y$, където $y \in I_k \cap L$.

\quad
Щом $\displaystyle x \in \union_{i \in I_k} A_i$ за произволно $k \in J$,
то $\displaystyle x \in \intersection_{j \in J} \union_{i \in I_j} A_i$.

\qed
\end{tcolorbox}

\end{problem}

\begin{problem}
Нека $I \ne \varnothing$ и $\{A_i\}_{i \in I}$ е $I$-индексирана фамилия от множества.
Нека $\{I_j\}_{j \in J}$ е $J$-индексирана фамилия от взаимно чужди и непразни подмножества на $I$,
като $\union_{j \in J} I_j = I$.
Да се докаже, че множествата
$\prod_{i \in I} A_i$ и $\prod_{j \in J}(\prod_{i \in I_j} A_i)$ са равномощни.
\end{problem}

\textbf{Решение:}
\smallbreak
\quad
Нека функцията $r: \prod_{i \in I} A_i \to \prod_{j \in J}(\prod_{i \in I_j} A_i)$ е дефинирана по следния начин:
\[
r(f) = s \text{ за }
s(j) = t \text{ за }
t(i) = f(i)
\]
\quad
Ще покажем, че $r$ е инекция:

\begin{tcolorbox}[mybox={Доказателство:}]
\quad
Нека $r(f_1) = r(f_2)$ и нека $i$ е произволен елемент на $I$.
Нека вземем $j \in J$ такова, че $i \in I_j$.
Ясно е, че тъй като $r(f_1) = r(f_2)$, то $r(f_1)(j) = r(f_2)(j)$.
Накрая остава да съобразим, че от дефиницията на $r$,
$f_1(i) = r(f_1)(j)(i) = r(f_2)(j)(i) = f_2(i)$.
Така получихме, че за произволно $i \in I$ е в сила $f_1(i) = f_2(i)$, откъдето
$f_1 = f_2$.

\qed
\end{tcolorbox}

\quad
Ще покажем, че $r$ е сюрекция:

\begin{tcolorbox}[mybox={Доказателство:}]
\quad
Нека $s \in \prod_{j \in J}(\prod_{i \in I_j} A_i)$.
Дефинираме $f \in \prod_{i \in I} A_i$ по следния начин:
\[
f(i) = s(j)(i) \text{ за $j$ такова, че } i \in I_j
\]
\quad
Твъдим, че $r(f) = s$. Нека $j$ е произволен елемент на $J$.
Тогава $\forall i \in I_j\ [r(f)(j)(i) = f(i) = s(j)(i)]$, тоест $r(f)(j) = s(j)$.
Така получихме, че за произволно $j \in J$ е в сила
$r(f)(j) = s(j)$, откъдето $r(f) = s$.

\qed
\end{tcolorbox}


\quad
Щом $r$ е инекция и сюрекция, то $r$ е биекция
$\Rightarrow \size{\prod_{i \in I} A_i} = \size{\prod_{j \in J}(\prod_{i \in I_j} A_i)}$.

\qed

\begin{problem}
\textbf{(ZF)}
Нека $I \ne \varnothing$ и $\{A_i\}_{i \in I}$ е $I$-индексирана фамилия от взаимно чужди множества.
Да се докаже, че множествата $\prod_{i \in I} ({}^{A_i} B)$ и ${}^{(\union_{i \in I} A_i)} B$ са равномощни.
\end{problem}

\textbf{Решение:}

\smallbreak
\quad
Нека функцията
$r: \prod_{i \in I} ({}^{A_i} B) \to {}^{(\union_{i \in I} A_i)} B$
е дефинирана по следния начин:
\[
r(f) = s \text{ за }
s(a) = f(i)(a) \text{ за $i$ такова, че $a \in A_i$ }
\]

\quad
Ще покажем, че $r$ е инекция:

\begin{tcolorbox}[mybox={Доказателство:}]
\quad
Нека $r(f_1) = r(f_2)$ и нека $i$ е произволен елемент на $I$.
Тогава от равенството на $r(f_1)$ и $r(f_2)$ и от дефиницията на $r$ в сила ще бъде:
$\forall a \in A_i\ [ f_1(i)(a) = r(f_1)(a) = r(f_2)(a) = f_2(i)(a)]$,
откъдето $f_1(i) = f_2(i)$.
Така получихме, че за произволно $i \in I$ е в сила $f_1(i) = f_2(i)$, откъдето
$f_1 = f_2$.

\qed
\end{tcolorbox}

\quad
Ще покажем, че $r$ е сюрекция:

\begin{tcolorbox}[mybox={Доказателство:}]
\quad
Нека $s \in {}^{(\union_{i \in I} A_i)} B$.
Дефинираме $f \in \prod_{i \in I} ({}^{A_i} B)$ по следния начин:
\[
f(i) = g \text{ за }
g(a) = s(a)
\]
\quad
Твъдим, че $r(f) = s$.
Нека $a$ е произволен елемент на $\union_{i \in I} A_i$ и нека $i$ е такова, че $a \in A_i$.
Тогава от дефинициите на $r$ и $f$ ще бъде изпълнено, че $r(f)(a) = f(i)(a) = g(a) = s(a)$.
Така получихме, че за произволно $a \in \union_{i \in I} A_i$ е в сила $r(f)(a) = s(a)$, откъдето $r(f) = s$.

\qed
\end{tcolorbox}


\quad
Щом $r$ е инекция и сюрекция, то $r$ е биекция, откъдето
$\size{\prod_{i \in I} ({}^{A_i} B)} = \size{{}^{(\union_{i \in I} A_i)} B}$.

\qed

\begin{problem}
\textbf{(ZF)}
Да се докаже, че за произволно множество $A$ са в сила следните:
\begin{enumerate}
\item
$\size{A} = \size{A \cup \{A\}} \Rightarrow \size{\pow(A)} = \size{\pow(A) \cup \{\pow(A)\}}$
\item
$\size{A} = \size{A \cup \{A\}} \Rightarrow \size{\pow(\pow(A))} = \size{\pow(\pow(A)) \times \pow(\pow(A))}$
\end{enumerate}
%\textbf{Упътване.}
%\textit{За 2) използвайте, че ${}^X 2$ и $\pow(X)$ са равномощни, както и че винаги,
%когато $B \cap C = \varnothing$, множествата ${}^{B \cup C} X$ и ${}^B X \times {}^C X$ са равномощни.}
\end{problem}

\textbf{Решение:}

\smallbreak
\quad
Първо ще докажем 1). Нека $\size{A} = \size{A \cup \{A\}}$ и нека $f: A \to A \cup \{A\}$ е произволна биекция.
Дефинираме функция $g: \pow(A) \to \pow(A) \cup \{\pow(A)\}$ по следния начин:
\[
g(x) =
\begin{cases}
\pow(A) & \text{, ако $\exists y\ [x = \{y\} \land f(y) = A]$} \\
\{f(\union x)\} & \text{, иначе, ако $\exists y\ [x = \{y\}]$} \\
x & \text{, иначе}
\end{cases}
\]

\quad
Ще докажем, че $g$ е инекция:
\begin{tcolorbox}[mybox={Доказателство:}]
\quad
Нека $g(x_1) = g(x_2)$. Разглеждаме четири случая:
\begin{enumerate}[label={\arabic* сл.}]
\item
$g(x_1) = \pow(A) = g(x_2)$.
Тогава
$\exists y_1\ [x_1 = \{y_1\} \land f(y_1) = A] \land
\exists y_2\ [x_2 = \{y_2\} \land f(y_2) = A]$.
Тъй като $f$ е инекция, последното влече, че $y_1 = y_2$, откъдето $x_1 = x_2$.

\item
$g(x_1) \ne \pow(A) \ne g(x_2) \land \exists y_1\ [x_1 = \{y_1\}] \land \exists y_2\ [x_2 = \{y_2\}]$.
Тогава $g(x_1) = \{f(y_1)\}$ и $g(x_2) = \{f(y_2)\}$.
Тъй като $g(x_1) = g(x_2)$, последното влече, че
$f(y_1) = f(y_2)$, откъдето $y_1 = y_2 $ и $x_1 = \{y_1\} = \{y_2\} = x_2$

\item
$\exists y_1\ [x_1 = \{y_1\}] \xor \exists y_2\ [x_2 = \{y_2\}]$.
Но това влече, че $\size{g(x_1)} \ne \size{g(x_2)}$, което е невъзможно.

\item
$\neg \exists y_1\ [x_1 = \{y_1\}] \land \neg \exists y_2\ [x_2 = \{y_2\}]$.
Тогава $x_1 = g(x_1) = g(x_2) = x_2$.
\end{enumerate}
\qed


\end{tcolorbox}

\quad
Ще докажем, че $g$ е сюрекция:
\begin{tcolorbox}[mybox={Доказателство:}]
\quad
Нека $y \in \pow(A) \cup \{\pow(A)\}$.
Разглеждаме три случая:
\begin{enumerate}[label={\arabic* сл.}]
\item
$y = \pow(A)$. Тогава можем да вземем като първообраз на $y$ множество $x \in \pow(A)$ от вида $x = \{z\}$,
където $z \in A \land f(z) = \{A\}$.

\item
$y \in \pow(A) \land \exists z\ [y = \{z\}]$
Нека $s = f^{-1}(z)$. Тогава $g(\{s\}) = \{f(\union \{s\})\} = \{f(s)\} = \{z\} = y$.

\item
$y \in \pow(A) \land \neg \exists z\ [y = \{z\}]$.
Тогава $g(y) = y$.
\end{enumerate}
\qed
\end{tcolorbox}

\quad
Щом $g$ е инекция и сюрекция, то $g$ е биекция, откъдето
$\size{\pow(A)} = \size{\pow(A) \cup \{\pow(A)\}}$.

\qed

% \quad
% Нека $t$ е произволен елемент, такъв, че $t \notin Rng(\pow(A))$.
% За да докажем 2) е достатъчно е да построим биекция $h: \pow(\pow(A)) \to \pow(\pow(A) \cup (\pow(A) \times \{t\}))$,
% тъй като
% \begin{alignat*}{3}
% \size{\pow(\pow(A)) \times \pow(\pow(A))} &= \size{{}^{\pow(A)} 2 \times {}^{\pow(A)} 2}  && \text{ // от упътването} &\\
%                                           &= \size{{}^{\pow(A)} 2 \times {}^{\pow(A) \times \{t\}} 2}  && \text{ // от свойствата на декартовото произведение} &\\
% 										  & = \size{{}^{\pow(A) \cup (\pow(A) \times \{t\})} 2} && \text{ // от упътването} & \\
% 										  & = \size{\pow(\pow(A) \cup (\pow(A) \times \{t\}))} && \text{ // от упътването} &
% \end{alignat*}
%
% Дефинираме $h: \pow(\pow(A)) \to \pow(\pow(A) \cup (\pow(A) \times \{t\}))$ по следния начин:
% %\[
% %h(x) = \{
% %y\ |\ y \in \pow(\pow(A) \cup (\pow(A) \times \{t\})) \land
% %( \exists a \in x\ \forall v\ [v \in a \iff (f(v) \in y \xor f(v) = A)])
% %\}
% %\]
% % \begin{alignat*}{1}
% % h(x) = \{ & y\ |\ y \in \pow(\pow(A) \cup (\pow(A) \times \{t\})) \land \exists a \in x[ \\
% %           & (\forall v\ [(v \in a \Rightarrow f(v) \ne A) \land (v \in a \iff f(v) \in y)]) \lor \\
% %           & (y \subseteq \pow(A) \times \{t\} \land ( \exists w \in a\ [f(w) = A \land \forall v\ [v \in a \setminus\{w\} \iff f(v) \in Dom(y)]]))]
% % \}
% % \end{alignat*}
% \begin{alignat*}{1}
% h(x) = \{ & y\ |\ y \in \pow(\pow(A) \cup (\pow(A) \times \{t\})) \land \forall z \in y\ \exists p \in x\ [ \\
%           & (\forall v\ [v \in p \iff f(v) \in z]) \lor \\
% 		  & (Rng(z) = t \land \exists v \in p\ [f(v) = A ] \land \forall v\ [v \in p \iff f(v) \in Dom(z) \xor f(v) = A])]
% \}
% \end{alignat*}
%
% Ще докажем, че $h$ е сюрекция:
%
% \begin{tcolorbox}[mybox={Доказателство:}]
% \quad
% Нека $y \in \pow(\pow(A) \cup (\pow(A) \times \{t\}))$. Нека:
% \[
% Z_1 = \{x\ |\ x \in y \land x \in \pow(A)\}
% \]
% \[
% Z_2 = \{x\ |\ x \in y \land x \in \pow(A) \times {t}\}
% \]
% Забележете, че $\{Z_1, Z_2\}$ е разбиване на $y$.
%
% \smallbreak
% Нека $x$ е следното множество:
% \[
% x = \{a\ |\ a \in \pow(A) \land [f[a] \in y \xor (\exists b \in a\ [f(b) = A \land f[a] \setminus b \in Dom(y)])] \}
% \]
% Твърдим, че $x$ е първообраз на $y$. Наистина
%
% \bigbreak
% Разглеждаме четири случая:
% %\begin{enumerate}[label={\arabic* сл.}]
% %
% %\end{enumerate}
% \qed
%
% \end{tcolorbox}
%
% Ще докажем, че $h$ е инекция.
%
% \begin{tcolorbox}[mybox={Доказателство:}]
% \quad
% Нека $h(x_1) = h(x_2)$. Разглеждаме четири случая:
% \qed
%
% \end{tcolorbox}

% Нека сега докажем 2). Дефинираме $h: \pow(\pow(A)) \to \pow(\pow(A)) \times \pow(\pow(A))$ по следния начин:
% \[
% h(x) = \pair{y, z} \text{ за } y = \{a \in \pow(A)\ |\ f^{-1}[a] \in x\} \land z = \{b \in \pow(A) \ |\ f^{-1}[b \cup \{A\}] \in x\}
% \]

\bigbreak
\quad
Сега ще докажем 2). Нека $c = f^{-1}(A)$. Дефинираме функция $h: \pow(\pow(A)) \to \pow(\pow(A)) \times \pow(\pow(A))$ по следния начин:
\[
h(x) = \pair{y, z} \text{ за } y = \{f[a]\ |\ a \in x \land c \notin a\} \land z = \{f[a \setminus \{c\}]\ |\ a \in x \land c \in a\}
\]
\quad
Ще докажем, че $h$ е инекция:

\begin{tcolorbox}[mybox={Доказателство:}]
\quad
Нека $h(x_1) = \pair{y, z} = h(x_2)$. Тогава:
\[
\{f[a]\ |\ a \in x_1 \land c \notin a\} = y = \{f[a]\ |\ a \in x_2 \land c \notin a\}
\]
\[
\{f[a \setminus \{c\}]\ |\ a \in x_1 \land c \in a\} = z = \{f[a \setminus \{c\}]\ |\ a \in x_2 \land c \in a\}
\]
\quad
Нека $b$ е произволно множество. Разглеждаме два случая:
\begin{enumerate}[label={\arabic* сл.}]
\item
$c \notin b$. Тогава:
\[
b \in x_1 \iff f[b] \in y \iff b \in x_2
\]
\item
$c \in b$. Тогава:
\[
b \in x_1 \iff f[b \setminus \{c\}] \in z \iff b \in x_2
\]
\end{enumerate}
\quad
Така от аксиомата за обемност $x_1 = x_2$.

\qed
\end{tcolorbox}

\quad
Ще докажем, че $h$ е сюрекция:

\begin{tcolorbox}[mybox={Доказателство:}]
\quad
Нека $\pair{y, z} \in \pow(\pow(A)) \times \pow(\pow(A))$. Дефинираме множество първообраз $x$ по следния начин:
\[
x = \underbrace{\{f^{-1}[b]\ |\ b \in y\}}_{M} \uplus \underbrace{\{f^{-1}[b] \cup \{c\}\ |\ b \in z\}}_{N}
\]
\quad
Отсега отбелязваме, че $M$ и $N$ са чужди множества, тъй като
$\forall x\ [x \in M \Rightarrow c \notin x] \land [x \in N \Rightarrow c \in x]$.
Нека $h(x) = \pair{y', z'}$.
В сила е следното:
\begin{alignat*}{2}
b \in y & \iff f^{-1}[b] \in x      & \text{ // от $f^{-1}[b] \notin N$ и дефиницията на $x$} \\
        & \iff f[f^{-1}[b]] \in y'  & \text{ // от $c \notin f^{-1}[b]$ и дефиницията на $h$} \\
		& \iff b \in y'             &
\end{alignat*}
\quad
По същия начин:
\begin{alignat*}{2}
b \in z & \iff f^{-1}[b] \cup \{c\} \in x                         & \text{ // от $f^{-1}[b] \cup \{c\} \notin M$ и дефиницията на $x$} \\
        & \iff f[(f^{-1}[b] \cup \{c\}) \setminus \{c\}] \in z'   & \text{ // от $c \in f^{-1}[b] \cup \{c\}$ и дефиницията на $h$} \\
        & \iff f[f^{-1}[b]] \in z'                                & \text{ // от $c \notin f^{-1}[b]$} \\
		& \iff b \in z'                                           &
\end{alignat*}

\quad
Така получихме, че $h(x) = \pair{y, z}$, откъдето $\forall \pair{y, z}\, \exists x\ [h(x) = \pair{y, z}]$.

\qed
\end{tcolorbox}

\quad
Щом $h$ е инекция и сюрекция, то $h$ е биекция, откъдето
$\size{\pow(\pow(A))} = \size{\pow(\pow(A)) \times \pow(\pow(A))}$.

\qed


%\begin{tcolorbox}[mybox={Доказателство:}]
%\quad
%Нека $h(x_1) = h(x_2)$. Разглеждаме четири случая:
%\qed

%\end{tcolorbox}


\begin{problem}
\textbf{(ZF)}
Нека $A$ е множество, за което е в сила $\size{A} = \size{A \cup \{A\}}$.
Да се докаже, че всеки път, когато $B$ е множество и $f: \pow(A) \cup B \bijarrow \pow(\pow(A))$
множествата $B$ и $\pow(\pow(A))$ са равномощни.
\end{problem}

\textbf{Решение:}

\smallbreak
\quad
Нека първо съобразим, че щом $f$ е биекция от $\pow(A) \cup B$ към $\pow(\pow(A))$,
то рестрикцията $f\restriction_B$ е инекция от $B$ към $\pow(\pow(A))$.
Ще покажем, че съществува инекция и в обратната посока.

\quad
Нека $h: \pow(\pow(A)) \times \pow(\pow(A)) \bijarrow \pow(A) \cup B$ е произволна биекция\footnote[2]{Знаем, че такава биекция съществува, понеже
$\size{\pow(\pow(A)) \times \pow(\pow(A))} = \size{\pow(\pow(A))}$ от 5-та задача и $\size{\pow(\pow(A))} = \size{\pow(A) \cup B}$ от условието.}
и нека $Y$ е следното множество:
\[
Y = \{y\ |\ y \in \pow(\pow(A)) \land \forall x \in \pow(\pow(A))\ [h(x, y) \notin \pow(A)]\}
\]

\quad
Ще докажем, че $Y \ne \varnothing$:
\begin{tcolorbox}[mybox={Доказателство:}]
\quad
Нека допуснем, че $Y = \varnothing$. Тогава:
\begin{equation}\label{blasphemy}
\forall y\, \exists x\ [h(x, y) \in \pow(A)]
\end{equation}

\quad
Ще покажем, че последното е невъзможно, като построим сюрективна фунцкия от $\pow(A)$ към $\pow(\pow(A))$.
Нека $r: \pow(A) \to \pow(\pow(A))$ е дефинирана по следния начин:
\[
r(z) = y \text{ за } h(x, y) = z
\]
\quad
Твърдим, че $r$ е сюрекция.

\begin{tcolorbox}[mybox={Доказателство:}, colback=green!20, colframe=green!60]
\quad
Нека допуснем противното и нека вземем $y \in \pow(\pow(A))$ такова, че:
\begin{equation}\label{notsurjection}
\forall z \in \pow(A)\ [r(z) \ne y].
\end{equation}
\quad
От (\ref{blasphemy}) съществува $x \in \pow(\pow(A))$ такова, че $h(x, y) \in \pow(A)$.
Нека вземем $z$ = $h(x, y)$.
Така, от дефиницията на $r$, $r(z) = y$, което е противоречие с (\ref{notsurjection}).

\qed
\end{tcolorbox}

\quad
Получихме, че съществува сюрекция от $\pow(A) \to \pow(\pow(A))$,
което противоречи с теоремата на Кантор $\Rightarrow$ първоначалното ни допускане е грешно, тоест $Y \ne \varnothing$.


\qed

\end{tcolorbox}

\quad
Нека сега вземем $y$ да е произволен елемент на $Y$. Дефинираме функция $s: \pow(\pow(A)) \to B$ по следния начин:
\[
s(x) = h(x, y)
\]
\quad
Забележете, че $s$ е добре-дефинирана, тъй като от $y \in Y$ следва, че $\forall x \in \pow(\pow(A))\ [h(x, y) \in B]$.
Освен това $s$ е инекция поради инективността на $h$.

\smallbreak
\quad
Накрая остава да съобразим, че щом съществува инекция от $B$ към $\pow(\pow(A))$ и щом съществува инекция от $\pow(\pow(A))$ към $B$,
то от теоремата на Кантор-Шрьодер-Бернщайн, $B$ и $\pow(\pow(A))$ са равномощни.

\qed

\begin{problem}
\textbf{(ZF)}
A. Линейно наредено множество $\mathcal{A} = \pair{A, \le}$ е
\textit{представено като сума на $I$-индексираната фамилия $\{A_i\}_{i \in I}$}, ако:
\begin{itemize}
\item
$(\forall i \in I)(A_i \neq \varnothing)$
\item
$(\forall i \in I)(\forall j \in I)(i \neq j \Rightarrow A_i \cap A_j = \varnothing)$
\item
$\union_{i \in I} A_i = A$
\item
$(\forall i \in I)(\forall j \in I)(i \neq j \Rightarrow (\forall x \in A_i)(\forall y \in A_j)(x < y) \lor (\forall x \in A_i)(\forall y \in A_j)(y < x))$
\end{itemize}

\quad
\textbf{1.}
Нека $\mathcal{A} = \pair{A, \leq}$ е линейно наредено множество,
представено като сума на $I$-индексирана фамилия $\{A_i\}_{i \in I}$.
Бинарната релация $\prec$ е дефинирана с равенството:
\[
\prec = \{z\ |\ (\exists i \in I)(\exists j \in I)(z = \pair{i, j}) \land (\exists x \in A_i)(\exists x \in A_j)(x < y)\}
\]

\quad
Да се докаже, че $\pair{I, \preceq}$ е линейно наредено множество.
За тази наредба $\preceq$ се казва, че е \textit{породена от представянето на $A$ като сума на
$I$-индексираната фамилия $\{A_i\}_{i \in I}$}.

\bigbreak
\textbf{Решение:}

\smallbreak
\quad
Първо да забележим, че $\preceq$ е рефлексивна в $I$ поради факта, че $\preceq$ е рефлексивно затваряне на $\prec$.

\quad
За да докажем, че $\pair{I, \preceq}$ е силно антисиметрична и транзитивна ще използваме следното наблюдение:

\smallbreak
\quad
\textbf{Наблюдение 1:}
\[
	(\forall i, j \in I) [i \preceq j \land i \neq j \iff (\forall x \in A_i)(\forall y \in A_j)(x < y) \land i \neq j]
\]

\begin{tcolorbox}[mybox, title={Доказателство:}]

\quad
Нека $i, j \in I$. Тогава:
\begin{alignat*}{4}
i \preceq j \land i \neq j & \iff & ( & \exists x \in A_i)(\exists y \in A_j)(x < y) \land i \neq j        & \text{ // от дефиницията на $\prec$ } \\
                           & \iff & ( & \exists x \in A_i)(\exists y \in A_j)(x < y)) \land i \neq j \land & \text{} \\
						   &      & ( &(\forall x \in A_i)(\forall y \in A_j)(x < y) \lor                  & \text{} \\
						   &      & ( &\forall x \in A_i)(\forall y \in A_j)(y < x))                       & \text{ // от $A$ - представено като сума на } \\
						   &      &   &                                                                    & \text{ // $I$-индексирана фамилия $\{A_i\}_{i \in I}$} \\
                           & \iff & ( &\forall x \in A_i)(\forall y \in A_j)(x < y)) \land i \neq j        & \text{ // от $A_i \neq \varnothing \neq A_j$}
\end{alignat*}
\qed
\end{tcolorbox}

\quad
Ще докажем, че $\preceq$ е силно антисиметрична:

\begin{tcolorbox}[mybox, title={Доказателство:}]
\quad
Нека $i, j \in I$. Тогава:
\begin{alignat*}{2}
i \preceq j \land i \neq j & \iff (\forall x \in A_i)(\forall y \in A_j)(x < y) \land i \neq j & \text{ // от \textbf{Наблюдение 1}} \\
            & \iff \neg(\exists x \in A_j)(\exists y \in A_i)(x < y) \land i \neq j & \text{} \\
			& \iff j \npreceq i \land i \neq j & \text { // от дефиницията на $\prec$}
\end{alignat*}
\qed
\end{tcolorbox}

\quad
Ще докажем, че $\preceq$ е транзитивна.

\begin{tcolorbox}[mybox, title={Доказателство:}]
\quad
Нека $i, j, k \in I$ са такива, че $i \preceq j \land j \preceq k$.
Ако $i = j$, то имаме, че $i = j \preceq k$.
По подобен начин, ако $j = k$, то $i \preceq j = k$.
Ако $i = k$, то от рефлексивността на $\preceq$ ще бъде изпълнено $i \preceq i = k$.
В случая, когато $i \ne j \ne k \ne i$ по \textbf{Наблюдение 1} ще бъдат в сила:

\begin{itemize}
\item
$(\forall x \in A_i)(\forall y \in A_j)(x < y)$
\item
$(\forall y \in A_j)(\forall z \in A_k)(y < z)$
\end{itemize}

\quad
От транзитивна на релацията $<$ и $A_j \neq \varnothing$ горните две влекат
$(\forall x \in A_i)(\forall z \in A_k)(x < z)$,
откъдето отново от \textbf{Наблюдение 1} следва, че $i \preceq k$.

\qed
\end{tcolorbox}

\quad
Щом $\preceq$ е рефлексивна, силно антисиметрична и транзитивна, то $\preceq$ е линейна наредба
и $\pair{I, \preceq}$ е линейно наредено множество.

\qed

\quad
Б. Нека $\mathcal{C} = \pair{C, \leq}$ е линейно наредено множество.

\quad
$\mathcal{C}$ се нарича \textit{гъсто}, ако има поне два различни елемента и:
\[
(\forall x \in C)(\forall y \in C)(x < y \Rightarrow (\exists z \in C)(x < z \land z < y))
\]

\quad
$\mathcal{C}$ се нарича \textit{разредено}, ако всеки път,
когато $B \subseteq C,\, \pair{B, \leq \cap (B \times B)}$ не е гъсто.

\quad
\textbf{2.} Да се докаже, че всяко линейно наредено множество $\mathcal{A} = \pair{A, \leq}$ е
разредено или може да се представи като сума на такава $I$-индексирана
фамилия $\{A_i\}_{i \in I}$, така че всяко едно от множествата $\pair{A_i, \leq \cap (A_i \times A_i)}$
е разредено и $\pair{I, \preceq}$ е гъсто, където $\preceq$ е породената от това представяне наредба.

\bigbreak
\textbf{Решение:}
\smallbreak
\quad
Нека $\mathcal{A} = \pair{A, \le}$ е произволно неразредено линейно наредено множество.
Ще докажем, че $\mathcal{A}$ може да се представи като сума на $I$-индексирана фамилия $\{A_i\}_{i \in I}$,
така че всяко едно от множествата $\pair{A_i, \leq \cap (A_i \times A_i)}$
е разредено и $\pair{I, \preceq}$ е гъсто, където $\preceq$ е породената от това представяне наредба.

Първо нека си дефинираме няколко помощни множества и нотации:
\[
\operatorname{interval}(x, y) \coloneq \{z\ |\ z \in A \land (x \le z \le y \lor y \le z \le x)\}
\]
\[
\operatorname{dense}(B) \iff \size{B} \ge 2 \land (\forall x \in B)(\forall y \in B)(x < y \Rightarrow (\exists z \in B)(x < z < y))
\]
\[
\operatorname{diluted}(B) \iff (\forall D \subseteq B)(\neg \operatorname{dense}(D))
\]
\[
\sim\ \coloneq \{\pair{x, y} \in A^2\ |\ \operatorname{diluted}(\operatorname{interval}(x, y))\}
\]

\quad
Твърдим, че $\sim$ е релация на еквивалентност.

\quad
Ще докажем, че $\sim$ е рефлексивна.

\begin{tcolorbox}[mybox, title={Доказателство:}]
\quad
Нека $x \in A$. Не е трудно да се съборази, че $\operatorname{interval}(x, x) = \{x\}$,
откъдето $\operatorname{diluted}(\operatorname{interval}(x, x))$ и $\pair{x, x} \in A$.

\qed
\end{tcolorbox}

\quad
Ще докажем, че $\sim$ е симетрична.

\begin{tcolorbox}[mybox, title={Доказателство:}]
\quad
Нека $\pair{x, y} \in\ \sim$. Тогава $\operatorname{diluted}(\operatorname{interval}(x, y))$,
и, тъй като $\operatorname{interval}(x, y) = \operatorname{interval}(y, x)$, то в сила ще бъде и
$\operatorname{diluted}(\operatorname{interval}(y, x))$, откъдето $\pair{y, x} \in\ \sim$.

\qed
\end{tcolorbox}

\quad
За да докажем, че $\sim$ е транзитивна, ще ползваме следните наблюдения:

\smallbreak
\quad
\textbf{Наблюдение 2:}
\[
(\forall x, y, z \in A) [\operatorname{interval}(x, z) \subseteq \operatorname{interval}(x, y) \cup \operatorname{interval}(y, z)]
\]

\begin{tcolorbox}[mybox, title={Доказателство:}]
\quad
Нека $x, y, z \in A$, $a \in \operatorname{interval}(x, z)$ и нека БОО $x < z$. Тогава $x \le a \le z$.
Разглеждаме два случая:
\begin{enumerate}[label={\arabic* сл.}]
\item
$a \le y$. Тогава $x \le a \le y$, откъдето
$a \in \operatorname{interval}(x, y) \subseteq \operatorname{interval}(x, y) \cup \operatorname{interval}(y, z)$.

\item
$a > y$. Тогава $y \le a \le z$, откъдето
$a \in \operatorname{interval}(y, z) \subseteq \operatorname{interval}(x, y) \cup \operatorname{interval}(y, z)$.


\end{enumerate}

\qed
\end{tcolorbox}

\quad
\textbf{Наблюдение 3:}
\[
(\forall x \in A) [\operatorname{diluted}(x) \Rightarrow (\forall y \subseteq x) [\operatorname{diluted}(y)]
\]
\begin{tcolorbox}[mybox, title={Доказателство:}]
\quad
Нека $x \in A$, $\operatorname{diluted}(x)$, $y \subseteq x$ и нека $a \subseteq y$.
Тогава $a \subseteq x$, откъдето $\neg \operatorname{dense}(a)$.

\qed
\end{tcolorbox}


\quad
Сега ще докажем, че $\sim$ е транзитивна.

\begin{tcolorbox}[mybox, title={Доказателство:}]
\quad
Нека $x, y, z \in A$ са такива, че $\pair{x, y}, \pair{y, z} \in\ \sim$. Тогава $\operatorname{diluted}(\operatorname{interval}(x, y))$
и $\operatorname{diluted}(\operatorname{interval}(y, z))$.
Ще докажем, че $\operatorname{diluted}(\operatorname{interval}(x, y) \cup \operatorname{interval}(y, z))$.
Да допуснем противното.
Нека вземем $a$ такова, че $a \subseteq interval(x, y) \cup interval(y, z) \land \operatorname{dense}(a)$.
Нека БОО\footnote{Лесно може да се съобрази, че всяко гъсто множество има поне 3 елемента,
откъдето сечението на $a$ с един от интервалите трябва да съдържа поне 2 елемента.}
$\size{a \cap \operatorname{interval}(x, y)} \ge 2$.
Тогава от $\operatorname{dense}(a)$ и дефиницията на $\operatorname{interval}$ ще бъде изпълнено:
\[
(\forall x \in a \cap \operatorname{interval}(x, y))(\forall y \in a \cap \operatorname{interval}(x, y))(x < y \Rightarrow (\exists z \in a \cap \operatorname{interval}(x, y)\footnote{$z \in a$ от $\operatorname{dense}(a)$, а $z \in \operatorname{interval}(x,y)$ от $(x < y < z)$})(x < z < y))
\]
\quad
откъдето
$\operatorname{dense}(a \cap \operatorname{interval}(x, y))$.
Но ние знаем, че $a \cap \operatorname{interval}(x, y) \subseteq \operatorname{interval}(x, y)$,
противоречие с $\operatorname{diluted} ( \operatorname{interval} (x, y))$.

\quad
Така показахме, че $\operatorname{diluted}(\operatorname{interval}(x, y) \cup \operatorname{interval}(y, z))$.
Щом това е изпълнено,
то от \textbf{Наблюдения 2 и 3} $\operatorname{diluted}(\operatorname{interval}(x, z))$,
откъдето $\pair{x, z} \in\ \sim$.

\qed
\end{tcolorbox}

\quad
Щом $\sim$ е рефлексивна, симетрична и транзитивна, то $\sim$ е релация на еквивалентност.

\quad
Нека сега вземем $I$ да е множеството от класовете на еквивалентност на $\sim$ и нека за $i \in I$ дефинираме $A_i$ като елементите на класа $i$:
\[
A_i \coloneq \{x\ |\ x \in i\}
\]

\quad
Твърдим, че с текущите дефиниции, $\mathcal{A}$ е
представено като сума на $I$-индексираната фамилия $\{A_i\}_{i \in I}$.
Лесно се вижда, че първите три условия за това са изпълнени,
понеже $\{A_i\}_{i \in I}$ е разбиране на $A$ (от $\sim$ - ралция на еквивалентност).
Ще докажем, че последното условие:
\[
(\forall i \in I)(\forall j \in I)(i \neq j \Rightarrow (\forall x \in A_i)(\forall y \in A_j)(x < y) \lor (\forall x \in A_i)(\forall y \in A_j)(y < x))
\]
\quad
също е в сила.

\begin{tcolorbox}[mybox, title={Доказателство:}]
\quad
Нека допуснем противното, тоест нека:
\[
(\exists i \in I)(\exists j \in I)(i \neq j \land ((\exists x \in A_i)(\exists y \in A_j)(x \le y) \land (\exists x' \in A_i)(\exists y' \in A_j)(y' \le x')))
\]
\quad
Нека вземем $i, j \in I$, $x, x' \in A_i$ и $y, y' \in A_j$ такива, че $i \neq j$ и $x \le y \land x' \ge y'$. Разглеждаме два случая:
\begin{enumerate}[label={\arabic* сл.}]
\item
$x' \ge y$. Тогава $\operatorname{interval}(x, y) \subseteq \operatorname{interval}(x, x')$.
От \textbf{Наблюдение 3} това влече, че $y \in A_i$, откъдето $i = j$, противоречие.

% \item
% $y' \le x$. Тогава $\operatorname{interval}(x, y) \subseteq \operatorname{interval}(y, y')$.
% От \textbf{Наблюдение 3} това влече, че $x \in A_j$, откъдето $i = j$, противоречие.

\item
$x' < y$.
Тогава $\operatorname{interval}(x', y) \subseteq \operatorname{interval}(y, y')$.
От \textbf{Наблюдение 3} това влече, че $x' \in A_j$, откъдето $i = j$, противоречие.
\end{enumerate}

\qed
\end{tcolorbox}

\quad
Ще покажем, че също така $\forall i \in I\ (diluted(A_i))$.
\begin{tcolorbox}[mybox, title={Доказателство:}]
\quad
Да допуснем противното. Нека $i \in I$ и $C \subseteq A_i$ са такива, че $\operatorname{dense}(C)$
и нека $x$ и $y$ са произволни елементи на $C$.
Тогава $\operatorname{dense}(\operatorname{interval}(x, y) \cap C)$,
но в същото време $\operatorname{diluted}(\operatorname{interval}(x, y))$ от
$x, y \in A_i$. Противоречие.

\qed
\end{tcolorbox}
\quad
Накрая ще покажем, че $\pair{I, \preceq}$ е гъсто.
\begin{tcolorbox}[mybox, title={Доказателство:}]
\quad
Нека $i, j \in I$ са такива, че $i \prec j$ и $i \neq j$ и нека $a$ е произволен елемент на $A_i$ и $b$ е произволен елемент на $A_j$.
Тъй като $i \neq j$, то $\neg \operatorname{diluted}(\operatorname{interval}(a, b))$, тоест:
\[
(\exists C \subseteq \operatorname{interval}(a, b)) (\operatorname{dense}(C))
\]
\quad
Нека $c$ е произволен некраен елемент на $C$ и нека $k \in I$ е такова, че $c \in A_k$.
Тогава $i \prec k \land i \neq k$.
Наистина, като свидетели за $i \prec k$ можем да вземем $a$ и $c$, понеже $a < c$,
а $i \neq k$, тъй като $\neg \operatorname{diluted}(\operatorname{interval}(a, c))$.

\quad
По подобен начин може да се покаже, че $k \prec j \land k \neq j$.

\quad
Така $k \in I$ е такова, че $i \prec k \prec j$ и $i \neq k \neq j$, откъдето $\pair{I, \preceq}$ е гъсто.

\qed
\end{tcolorbox}

\qed

\end{problem}

\begin{problem}
\textbf{(ZF)}
Нека $\mathcal{S}$  е непразно подмножество на $\pow(A)$, което има следните две свойства:
\begin{enumerate}
\item
$A \in \mathcal{S}$
\item
$\mathcal{S}$ е затворено относно произволни непразни сечения, т.е. всеки път,
когато $\mathcal{X} \neq \varnothing$ и $\mathcal{X} \subseteq \mathcal{S}$,
е в сила $\intersection\mathcal{X} \in \mathcal{S}$, и $A \in \mathcal{S}$.
Да се докаже, че всяка монотонна функция $h: \mathcal{S} \to \mathcal{S}$ има неподвижна точка.
\end{enumerate}
\end{problem}

\textbf{Решение:}

\smallbreak
\quad
Ще имитираме теоремата на Тарски. Нека си дефинираме $X$ да е следното множество:
\[
X = \{ x \ |\ x \in \mathcal{S} \land h(x) \subseteq x\}
\]

\quad
Да забележим, че $A \in \mathcal{S} \land h(A) \subseteq A$, откъдето $A \in X$, тоест $X$ не е празно.

\quad
Нека сега разгледаме множеството $X' = \intersection X$.
От второто свойство на множеството $\mathcal{S}$, то е в $\mathcal{S}$.

\quad
Твърдим, че освен това $X'$ е неподвижна точка на $h$.

\begin{tcolorbox}[mybox={Доказателство:}]
\quad
Първо ще докажем, че $h(X') \subseteq X'$.

\begin{tcolorbox}[mybox={Доказателство:}, colback=green!20, colframe=green!60]
\quad
Нека $Y$ е произволен елемент на $X$. Тогава от $h$-монотонна и от $X' = \intersection X \subseteq Y$ ще бъде в сила следното:
\[
h(X') \subseteq h(Y) \subseteq Y
\]

\quad
Така полуаваме, че $h(X')$ е подмножество на всеки елемент на $X$,
откъдето $h(X')$ е подмножество на $\intersection X = X'$.

\qed
\end{tcolorbox}

\quad
Сега ще докажем, че $X' \subseteq h(X')$.

\begin{tcolorbox}[mybox={Доказателство:}, colback=green!20, colframe=green!60]
\quad
Тъй като $h(X') \subseteq X'$, то от $h$-монотонна имаме, че:
$h(h(X')) \subseteq h(X')$, откъдето $h(X') \in X$. Но $X' = \intersection X$,
следователно $X' \subseteq h(X')$.

\qed
\end{tcolorbox}

\quad
Щом $h(X') \subseteq X'$ и $X' \subseteq h(X')$, то $X' = h(X')$, откъдето $X'$ е неподвижна точка на $h$.

\qed
\end{tcolorbox}


\bigbreak
\quad
За 9-та и 10-та задача ще използваме следните три леми:

\begin{tcolorbox}[mybox={Лема 1}, colback=purple!20, colframe=purple!40]
\quad
За всяко линейно наредено множество $\pair{A, \leq}$ е в сила:
\[
\forall a, b \in A\ [ seg(a) \subseteq seg(b) \lor seg(b) \subseteq seg(a)]
\]
\end{tcolorbox}

\begin{tcolorbox}[mybox={Доказателство:}]
\quad
Нека $a$ и $b$ са произволни елементи на $A$.
Тогава:
\[
\forall c \in seg(a) \ [c < a]
\]
\[
\forall c \in seg(b) \ [c < b]
\]

\quad
Разглеждаме два случая:
\begin{enumerate}[label={\arabic* сл.}]
\item
$a<b$.
Тогава $\forall c \in seg(a) \ [c<a \land a<b]$, тоест $\forall c \in seg(a) \ [c<b]$, откъдето $seg(a) \subseteq seg(b)$.

\item
$b<a$.
Тогава $\forall c \in seg(b) \ [c<b \land b<a]$, тоест $\forall c \in seg(b) \ [c<a]$, откъдето $seg(b) \subseteq seg(a)$.

\end{enumerate}
\qed
\end{tcolorbox}

\begin{tcolorbox}[mybox={Лема 2}, colback=purple!20, colframe=purple!40]

\quad
Нека $\pair{A, \leq}$ е непразно линейно наредено множество и нека $X$ е такова, че:
\begin{itemize}
\item
$X \neq \varnothing$
\item
$(\forall y \in X)(\exists x \in A)(y = seg(x))$
\item
$(\exists x)(\intersection X = seg(x))$
\end{itemize}

\quad
Тогава $\intersection X \in X$.
\end{tcolorbox}

\begin{tcolorbox}[mybox={Доказателство:}]
\quad
Да допуснем противното.
Нека $X$ е множество изпълняващо условията на лемата, нека $x$ е такова, че $X = seg(x)$ и нека $\intersection X \notin X$.
Разглеждаме два случая:
\begin{enumerate}[label={\arabic* сл.}]
\item
$\forall a \in X\ [seg(x) \subsetneq a]$.
Тогава $\forall a \in X\ [x \in a]$, тоест $x \in \intersection X = seg(x)$. Противоречие.
\item
$\exists a \in X\ [a \subsetneq seg(x) ]$.
Тогава $a \subsetneq seg(x) = \intersection X \subseteq a$. Противоречие.
\end{enumerate}

\quad
От \textbf{Лема 1} тези 2 случая са изчерпателни.
Така допускането ни се оказа грешно, откъдето $\intersection X \in X$.

\qed


\end{tcolorbox}

\begin{tcolorbox}[mybox={Лема 3}, colback=purple!20, colframe=purple!40]

\quad
Нека $\pair{A, \leq}$ е непразно линейно наредено множество, нека $y$ е непразно подмножество на $A$
нека $X$ е следното множество:
\[
X = \{ a\ |\ a \in \pow(A) \land \exists b\ [ b \in y \land  a = seg(b)] \}
\]
\quad
и нека $\intersection X = seg(x) \in X$. Тогава $y$ има най-малък елемент $x$.
\end{tcolorbox}

\begin{tcolorbox}[mybox={Доказателство:}]
\quad
Да допуснем противното.
Нека $x' \in y$ е такова, че $x' < x$.
Тогава $x' \in seg(x) = \intersection X \subseteq seg(x')$.
Противоречие с $x' \notin seg(x')$.

\qed
\end{tcolorbox}

\begin{problem}
\textbf{(ZF)}
Нека $\pair{A, \leq}$ е линейно наредено множество.
Нека функцията $\pi: \pow(A) \to \pow(A)$ е определена чрез:
\[
\pi(X) = \{y \in A\ | \ seg(y) \subseteq X\}
\]

Покажете, че $\pi$ е монотонна и, че ако $A^*$ е най-малката неподвижна точка на $\pi$,
то за всяко $x \in A$
\[
x \in A^* \iff \pair{seg(x), \leq \cap (seg(x) \times seg(x))} \text{ е добре наредено множество.}
\]
\end{problem}


\textbf{Решение:}

\smallbreak
\quad
Първо ще направим няколко наблюдения.

\quad
\textbf{Наблюдение 1:}
За всяко подмножество $X$ на $A$, $\pi(X)$ е начален сегмент на $A$.
\begin{tcolorbox}[mybox={Доказателство:}]
\quad
Нека $X \in \pow(A)$, $a \in \pi(X)$ и нека $b \in A$ е такова, че $b < a$.
Тогава $seg(a) \subseteq X$ и $b \in seg(a)$, откъдето $seg(b) \subseteq seg(a) \subseteq X$,
следователно $b \in \pi(X)$.

\qed
\end{tcolorbox}

\quad
\textbf{Наблюдение 2:}
$\forall x \in A\ [A^* \neq seg(x)]$.
\begin{tcolorbox}[mybox={Доказателство:}]
\quad
Да допуснем, че същесвува $x \in A$ такова, че $A^* = seg(x)$ и нека фиксираме това $x$.
Тогава от дефиницията на $\pi$, $x \in \pi(A^*) = A^* = seg(x)$.
Противоречие.

\qed
\end{tcolorbox}

\quad
\textbf{Наблюдение 3:}
За всеки собствен начален сегмент $X$ на $A^*$ е изпълнено $\exists x \in A\ [X = seg(x)]$.
\begin{tcolorbox}[mybox={Доказателство:}]
\quad
Да допуснем противното. Нека $X$ е собствен начален сегмент на $A^*$, за който
$\forall x \in A\ [X \neq seg(x)]$.
Тогава $\pi(X) = X$.
\begin{tcolorbox}[mybox={Доказателство:}, colback=green!20, colframe=green!60]
\quad
$(\Rightarrow)$
Нека $x \in \pi(X)$. Тогава $seg(x) \subseteq X$.
Тъй като $X$ е начален сегмент и $X \neq seg(x)$,
последното влече, че $x \in \pi(X).$

\quad
$(\Leftarrow)$
Нека сега $x \in X$.
От това, че $X$ е начален сегмент имаме, че $seg(x) \subseteq X$,
откъдето $x \in \pi(X)$.

\qed

\end{tcolorbox}

\quad
Така получихме, че $X \subsetneq A^*$ е неподвижна точка. Противоречие с дефиницията на $A^*$.

\qed
\end{tcolorbox}


\quad
$(\Rightarrow)$
Нека сега вземем $x$ да е произволен елемент на $A^*$
и нека вземем $y$ да е произволно непразно подмножество на $seg(x)$.
Ще докажем, че $y$ има най-малък елемент относно $\leq$.
За целта си дефинираме следното множество:
\[
X = \{ a\ |\ a \in \pow(seg(x)) \land \exists b\ [ b \in y \land  a = seg(b)] \}
\]
\quad
Тъй като $X$ е множество от собствени начални сегменти на $A^*$, то $\intersection X$ е собствен начален сегмент на $A^*$,
откъдето по \textbf{Наблюдение 3}
$(\exists a \in A)(\intersection X = seg(a))$.
Така по \textbf{Лема 2} и \textbf{Лема 3} $y$ има най-малък елемент $a$.

\qed

\quad
$(\Leftarrow)$ Нека сега
$\pair{seg(x), \leq \cap (seg(x) \times seg(x))}$
е добре наредено множество.
Да допуснем, че $x \notin A^*$.
Така от $A^*$ - начален сегмент, следва че $A^* \subsetneq seg(x)$.

\quad
Нека сега разгледаме множеството $seg(x) \setminus A^* \subseteq seg(x)$.
Тъй като
$\pair{seg(x), \leq \cap (seg(x) \times seg(x))}$
е добре наредено множество, то $seg(x) \setminus A^*$ има нак-малък елемент $z$.
Това обаче би означавало, че $A^* = seg(z)$, противоречие с \textbf{Наблюдение 2}.

\quad
Така допускането ни се оказа грешно, откъдето $x \in A^*$.

\qed

\begin{problem}
Нека $\pair{A, R}$ е линейно наредено множество.
Нека всеки път, когато $\omega$ е начален сегмент на $\pair{A, R}$, е в сила
$\omega = A$ или $(\exists x \in A)(\omega = seg(x))$.
Докажете, че $\pair{A, R}$ е добре наредено множество.
\end{problem}

\textbf{Решение:}

\smallbreak
\quad
Нека вземем $y$ да е произволно непразно подмножество на $A$.
Ще докажем, че $y$ има най-малък елемент относно $R$.
За целта си дефинираме следното множество:
\[
z = \{ a\ |\ a \in \pow(A) \land \exists x\ [ x \in y \land  a = seg(x)] \}
\]
\quad
Да забележим, че $z$ не е празно, тъй като за произволен елемент $x$ на $y$ е изпълнено, че $seg(x) \in z$.
Освен това $\intersection z$ е начален сегмент на $A$, тъй като сечението на начални сегменти е начален сегмент.

\quad
Щом $\intersection z$ е начален сегмент на $A$, то от условието на задачата имаме две възможности:
\begin{enumerate}[label={\arabic* сл.}]
\item
$\intersection z = A$.
Този случай е невъзможен, тъй като $\forall x \in A\ [A \neq seg(x)]$.
\item
$\intersection z = seg(x)$ за някакво $x \in A$.
Твърдим, че в този случай $x \in y$.
За да докажем това ще използваме следното помощно наблюдение:

\quad
\textbf{Наблюдение 1:}
\[
\forall a, b \in A\ [ seg(a) \subseteq seg(b) \lor seg(b) \subseteq seg(a)]
\]
\begin{tcolorbox}[mybox={Доказателство:}]
\quad
Нека $a$ и $b$ са произволни елементи на $A$.
Тогава:
\[
\forall c \in seg(a) \ [cRa]
\]
\[
\forall c \in seg(b) \ [cRb]
\]

\quad
Разглеждаме два случая:
\begin{enumerate}[label={\arabic* сл.}]
\item
$aRb$.
Тогава $\forall c \in seg(a) \ [cRa \land aRb]$, тоест $\forall c \in seg(a) \ [cRb]$, откъдето $seg(a) \subseteq seg(b)$.

\item
$bRa$.
Тогава $\forall c \in seg(b) \ [cRb \land bRa]$, тоест $\forall c \in seg(b) \ [cRa]$, откъдето $seg(b) \subseteq seg(a)$.

\end{enumerate}
\qed
\end{tcolorbox}

\quad
Сега ще докажем, че $x \in y$.
\begin{tcolorbox}[mybox={Доказателство:}]
\quad
Да допуснем противното.
Нека $x \notin y$.
Тогава $seg(x) \notin z$.
Разглеждаме два случая:
\begin{enumerate}[label={\arabic* сл.}]
\item
$\forall a \in z\ [seg(x) \subsetneq a]$.
Тогава $\forall a \in z\ [x \in a]$, тоест $x \in \intersection z = seg(x)$. Противоречие.
\item
$\exists a \in z\ [a \subsetneq seg(x) ]$.
Тогава $a \subsetneq seg(x) = \intersection z \subseteq a$. Противоречие.
\end{enumerate}

\quad
От \textbf{Наблюдение 1} тези 2 случая са изчерпателни.
Така допускането не се оказва грешно, откъдето $x \in y$.

\qed
\end{tcolorbox}

\quad
Накрая остава да съобразим, че щом $z = seg(x) \land x \in y$, то $x$ е най-малкият елемент на $y$.

\begin{tcolorbox}[mybox={Доказателство:}]
\quad
Да допуснем противното.
Нека $x' \in y$ е такова, че $x' \neq x \land x'Rx$.
Тогава $x' \in seg(x) = \intersection z \subseteq seg(x')$.
Противоречие с $x' \notin seg(x')$.

\qed
\end{tcolorbox}
\end{enumerate}

\quad
Показахме, че произволно подмножество на $A$ има най-малък елемент относно $R$,
следователно $\pair{A, R}$ е добре наредено множество.

\qed


\begin{problem}
\textbf{(ZF)}
Нека $A \neq \varnothing$ и $f: \pow(A) \setminus \{\varnothing \} \Rightarrow A$ е функция на избора за $A$.
Да се докаже, че следните две условия са еквивалентни:
\begin{enumerate}
\item
$\forall x \forall y (x, y \in Dom(f) \Rightarrow f(x \cup y) = f(\{ f(x), f(y) \}) )$

\item
съществува такава добра наредба $\leq$ в $A$, че за всяко непразно подмножество $y$ на $A$ е в сила $f(y) = min_{\leq} y$.
\end{enumerate}

\end{problem}

\textbf{Решение:}

\smallbreak
\quad
$(\Rightarrow)$ Нека първо приемем, че
$\forall x \forall y (x, y \in Dom(f) \Rightarrow f(x \cup y) = f(\{ f(x), f(y) \}) )$
Дефинираме си релацията $\leq$ по следния начин:
\[
\leq \coloneq \{ (a, b)\ |\ (a, b) \in A^2 \land f(\{a, b\}) = a \}
\]

\quad
Ясно е, че $\leq$ е рефлексивна, тъй като $(\forall a \in A)(f(\{a, a\}) = f(\{a\}) = a)$.

\quad
Ще докажем, че  $\leq$ е силно антисиметрична:

\begin{tcolorbox}[mybox, title={Доказателство:}]
\quad
Нека $a, b \in A$. Тогава:
\begin{alignat*}{2}
a < b & \iff f(\{a, b\}) = a        & \text{ // от дефиницията на $<$ } \\
      & \iff f(\{a, b\}) \neq b     & \text{ // от $f$ - функция на избора} \\
      & \iff b \not< a              & \text{ // от дефиницията на $<$}
\end{alignat*}
\qed
\end{tcolorbox}

\quad
Ще докажем, че  $\leq$ е транзитивна:

\begin{tcolorbox}[mybox, title={Доказателство:}]
\quad
Нека $a, b, c \in A$ са такива, че $a \leq b \land b \leq c$.
Тогава $f(\{a, b\}) = a$ и $f(\{b, c\}) = b$, откъдето по 1.
$f(\{a, b, c\}) = f(\{a, b\} \cup \{b, c\}) = f(\{f(\{a, b\}), f(\{b,c\})\}) = f(\{a, b\}) = a$.
Същевременно
$f(\{a, b, c\}) = f(\{a, c\} \cup \{b\}) = f(\{f(\{a, c\}), f(\{b\})\}) = f(\{f(\{a, c\}), b\})$
Така $a = f(\{f(a, c), b\})$.
От $f$-функция на избора, последното влече, че $a \in \{f(\{a, c\}), b\}$,
откъдето $f(\{a, c\}) = a$ и $a < c$.

\qed
\end{tcolorbox}

\quad
Щом $\leq$ е рефлексивна, силно антисиметрична и транзитивна, то $\leq$ е линейна наредба

\quad
Ще докажем, че освен това $\leq$ е добра наредба.

\quad
Нека $y$ е произволно непразно подмножество на $A$. Твърдим, че $y$ има най-малък елемент $f(y)$.

\begin{tcolorbox}[mybox, title={Доказателство:}]
\quad
Да забележим, че от $f$ - функция на избора следва, че $f(y) \in y$.
Нека сега допуснем, че $f(y)$ не е най-малък елемент на $y$. Тогава
$\exists z \in y\ [z < f(y)]$.
Нека си фиксираме такова $z$.
От $z < f(y)$ ще бъде изпълнено, че $f(\{z, f(y)\}) = z$.
Но тогава от 1.
$f(y) = f(y \cup \{z\}) = f(\{f(y), f(\{z\})\}) = f(\{f(y), z\}) = z$.
Противоречие с $z < f(y)$.

\qed
\end{tcolorbox}

\bigbreak
\quad
$(\Leftarrow)$
Нека сега съществува такава добра наредба $\leq$ в $A$, че за всяко непразно подмножество $y$ на $A$ е в сила $f(y) = min_{\leq} y$
и нека фиксираме тази наредба.
Ще докажем, че:
\[
\forall x \forall y (x, y \in Dom(f) \Rightarrow f(x \cup y) = f(\{ f(x), f(y) \}) )
\]

\begin{tcolorbox}[mybox, title={Доказателство:}]
\quad
Нека $x, y \in Dom(f)$.
От 2. $f(x) = min_{\leq} x$ и $f(y) = min_{\leq} y$.
Нека също така забележим, че минимумът на $x \cup y$ може да е само $f(x)$ или $f(y)$.
БОО нека минимумът е $f(x)$.
Тогава $f(x \cup y) = min_{\leq} (x \cup y) = f(x) = f(\{f(x), f(y)\})$.

\qed
\end{tcolorbox}

\begin{problem}
\textbf{(ZF)}
Нека $\pair{A, \leq}$ е добре наредено множество.
В $A \times A$ дефинираме бинарната релация $\leq^{can}$ така:

\smallbreak
\quad
За произволни $a_1, a_2, b_1$ и $b_2$ от $A$, $\pair{\pair{a_1, b_1}, \pair{a_2, b_2}} \in \leq^{can}$
точно тогава, когато:
\begin{align*}
\operatorname{max}_{\leq}\{a_1, b_1\} & < \operatorname{max}_{\leq}\{a_2, b_2\} \text{ или} \\
\operatorname{max}_{\leq}\{a_1, b_1\} & = \operatorname{max}_{\leq}\{a_2, b_2\} \land a_1 < a_2 \text{ или} \\
\operatorname{max}_{\leq}\{a_1, b_1\} & = \operatorname{max}_{\leq}\{a_2, b_2\} \land a_1 = a_2 \land b_1 \leq b_2
\end{align*}
\quad
Да се докаже, че $\pair{A \times A, \leq^{can}}$ е добре наредено множество.
\end{problem}

\textbf{Решение:}

\smallbreak
\quad
Нека $X$ е непразно подмножество на $A \times A$. Ще докажем, че $X$ има най-малък елемент относно дефинираната в условието релация.
За целта ще използваме шест функции: $\operatorname{max}, \operatorname{fst}, \operatorname{snd:} A \times A \to A;$
$f, g, h: \pow(X) \to \pow(X)$, дефинирани по следните начини:
\[
\operatorname{max}(a, b) = \operatorname{max}_{\leq}\{a, b\}
\]
\[
\operatorname{fst}(a, b) = a
\]
\[
\operatorname{snd}(a, b) = b
\]
\[
f(B) = \{b\ |\ b \in B \land \operatorname{max}(b) = \operatorname{min}_{\leq}(\operatorname{max}[B])\}
\]
\[
g(B) = \{b\ |\ b \in B \land \operatorname{fst}(b) = \operatorname{min}_{\leq}(\operatorname{fst}[B])\}
\]
\[
h(B) = \{b\ |\ b \in B \land \operatorname{snd}(b) = \operatorname{min}_{\leq}(\operatorname{snd}[B])\}
\]

\quad
Нека сега разгледаме множеството $C \coloneq h(g(f(X)))$.
Да забележим, че от дефиницциите на $f, g$ и $h$ и от $X \neq \varnothing$ следва, че то не е празно.
Нека тогава $c = \pair{c_1, c_2}$ е произволен елемент на $C$.
Твърдим, че $c$ е най-малкият елемент на $X$.

\begin{tcolorbox}[mybox={Доказателство:}]
\quad
Нека $a = \pair{a_1, a_2} \in X$.
Да допуснем, че $a < c$
Тогава е изпълнено едно от трите:
\begin{enumerate}[label={\arabic* сл.}]
\item
$\operatorname{max}_{\leq}\{a_1, a_2\} < \operatorname{max}_{\leq}\{c_1, c_2\}$.
Тогава $c \notin f(X)$, откъдето
$c \notin g(f(X))$ и $c \notin h(g(f(X))) = C$.
Противоречие с $c \in C$.

\item
$\operatorname{max}_{\leq}\{a_1, a_2\} = \operatorname{max}_{\leq}\{c_1, c_2\} \land a_1 < c_1$.
Тогава $c \notin g(f(X))$,
откъдето $c \notin h(g(f(X))) = C$.
Противоречие с $c \in C$.

\item
$\operatorname{max}_{\leq}\{a_1, a_2\} = \operatorname{max}_{\leq}\{c_1, c_2\} \land a_1 = c_1 \land a_2 \leq c_2$.
Ако $a_2 < c_2$, то $c \notin h(g(f(X))) = C$. Противоречие с $c \in C$.
Така $a_2 = c_2$, откъдето $a = c$.
Противоречие с $a < c$.

\end{enumerate}

\quad
Така допускането ни се оказа грешно, откъдето $c \leq a$.

\qed
\end{tcolorbox}

\quad
Показахме, че произволно подмножество на $A \times A$ има най-малък елемент. Така $A \times A$ е добре наредено.

\qed

\begin{problem}
Нека $\mathcal{A} = \pair{A, \leq}$ е линейно наредено множество.
\begin{enumerate}
\item
\textbf{(ZF)} Да се докаже, че ако $A$ е добре наредено множество, то
няма функция $f: \omega \to A$, такава че за всяко естествено число $n$ да
е в сила $f (n + 1) < f (n)$.

\textbf{Решение:}

\smallbreak
\quad
Нека допуснем противното и
нека фиксираме една фунцкия $f: \omega \to A$ такава, че $\forall n \in \omega\ [f(n+1) < f(n)]$.
Нека също така $a$ е най-малкият елемент на $\operatorname{Rng}(f)$
(тъй като $\operatorname{Rng}(f) \subseteq A$, то от добрата нареденост на $\mathcal{A}$, ще следва, че такъв елемент същесвува).
и нека $b \in \omega$ е такова, че $f(b) = a$.
Тогава $f(b+1) \in \operatorname{Rng}(f)$ и $f(b+1) < f(b) = a$.
Противоречие с избора на $a$.

\qed

\item
\textbf{(ZF)+(DC)}
Да се докаже, че ако няма функция $f: \omega \to A$
удовлетворяваща условието $(\forall n \in \omega)(f(n+1) < f(n))$, то $\mathcal{A}$ е добре наредено множество.

\textbf{Забележка. (DC)} е така наречената \textit{аксиома за зависимия избор},
която следва от аксиомата за избора в \textbf{(ZF)}:
\begin{quote}
\textit{
Всеки път, когато $B$ е непразно множество и $R$ е бинарна
релация със свойството $(\forall x \in B)(\exists y \in B)(xRy)$,
за произволно $b_0 \in B$ има такава функция $g: \omega \to B$, че $g(0) = b_0$ и
$(\forall n \in \omega)(g(n)Rg(n + 1))$.
}
\end{quote}

\textbf{Решение:}

\smallbreak
\quad
Нека няма функция $f: \omega \to A$ удовлетворяваща условието $(\forall n \in \omega)(f(n+1) < f(n))$.
Да допуснем, че $\mathcal{A}$ не е добре наредено множество.
Тогава съществува непразно подмножество на $A$,
което няма най-малък елемент.
Нека $B \subseteq A$ е едно такова множество.
Тогава:
\[
(\forall x \in B)(\exists y \in B)(x > y)
\]

\quad
От аксиомата за зависимия избор, последното влече, че
за произволно $b_0 \in B$ има такава функция $g: \omega \to B$, че $g(0) = b_0$ и
$(\forall n \in \omega)(g(n) > g(n + 1))$.
Разширявайки кодомейна на $g$ до $A$, получаваме функция $g': \omega \to A$,
за която $(\forall n \in \omega)(g'(n+1) < g'(n))$.
Противоречие.

\quad
Така допускането ни се оказа грешно, откъдето $\mathcal{A}$ е добре наредено множество.

\qed

\end{enumerate}
\end{problem}

\begin{problem}
Нека $A$ и $B$ са множества.
Ще казваме, че $A \leq B$, ако съществува инекция
$f : A \rightarrowtail B$.
Ще казваме, че $A \leq^* B$, ако съществува сюрекция
$f : B \twoheadrightarrow A$.
Докажете, че:
\begin{enumerate}
\item
\textbf{(ZF)}
За произволни множества $A \neq \varnothing$ и $B$ е в сила:
$A \leq B \Rightarrow A \leq^* B$.

\textbf{Решение:}

\smallbreak
\quad
Нека $f: A \rightarrowtail B$ е произволна инекция и нека $a$ е произволен елемент на $A$.
Дефинираме функцията $g: B \rightarrow A$ по следния начин:
\[
g(b) = \begin{cases}
       f^{-1}(b) & \text{, ако $b \in \operatorname{Rng}(f)$} \\
	   a       & \text{, иначе}
       \end{cases}
\]

\quad
Твърдим, че $g$ е сюрекция.

\begin{tcolorbox}[mybox={Доказателство:}]
\quad
Нека $c \in A$. Тогава $g(f(c)) = f^{-1}(f(c)) = c$, тъй като $f(c) \in \operatorname{Rng(f)}$.

\qed
\end{tcolorbox}

\item
\textbf{(ZF)}
За произволни множества $A$ и $B$ е в сила:
$A \leq^* B \Rightarrow \pow(A) \leq \pow(B)$.

\textbf{Решение:}

\smallbreak
\quad
Нека $f: B \twoheadrightarrow A$ е произволна сюрекция.
Дефинираме функцията $g: \pow(A) \rightarrow \pow(B)$ по следния начин:
\[
g(X) = \{b\ |\ b \in B \land f(b) \in X \}
\]

\quad
Твърдим, че $g$ е инекция.

\begin{tcolorbox}[mybox={Доказателство:}]
\quad
Нека $X, X' \in \pow(A)$ са такива, че $g(X) = g(X')$.
Да допуснем, че $X \neq X'$. Нека БОО вземем $a \in A$ такова, че $a \in X \land a \notin X'$
и нека $b$ е такова, че $f(b) = a$. Тогава $b \in g(X) \land b \notin g(X')$,
откъдето $g(X) \neq g(X')$. Противоречие.

\quad
Така допускането ни се оказа грешно, следователно $X = X'$.

\qed
\end{tcolorbox}

\item
\textbf{(ZF)}
За произволни множества $A$ и $B$, ако $B$ е добре наредимо, то е в сила:
$A \leq^* B \Rightarrow A \leq B$.

\textbf{Решение:}

\smallbreak
\quad
Нека $f: B \twoheadrightarrow A$ е произволна сюрекция и нека $\pair{B, \leq}$ е произволна добра наредба.
Дефинираме функцията $g: A \rightarrow B$ по следния начин:
\[
g(a) = min_{\leq}(\{c\ |\ c \in B \land f(c) = a\})
\]

\quad
Твърдим, че $g$ е инекция.

\begin{tcolorbox}[mybox={Доказателство:}]
\quad
Нека $a, a' \in A$ са такива, че $g(a) = g(a')$.
Тогава от дефиницията на $g$ имаме, чe:
\[
a = f(g(a)) = f(g(a')) = a'
\]
\qed
\end{tcolorbox}


\item
\textbf{(ZFC)}
За произволни множества $A$ и $B$ е в сила:
$A \leq^* B \Rightarrow A \leq B$.

\textbf{Решение:}

\smallbreak
\quad
Нека $f: B \twoheadrightarrow A$ е произволна сюрекция
и нека $h: \pow(B) \setminus \varnothing \to B$ е произволна функция на избора за $B$.
Дефинираме функцията $g: A \rightarrow B$ по следния начин:
\[
g(a) = h(\{c\ |\ c \in B \land f(c) = a\})
\]

\quad
Твърдим, че $g$ е инекция.

\begin{tcolorbox}[mybox={Доказателство:}]
\quad
Нека $a, a' \in A$ са такива, че $g(a) = g(a')$.
Тогава от дефиницията на $g$ имаме, чe:
\[
a = f(g(a)) = f(g(a')) = a'
\]
\qed
\end{tcolorbox}


\end{enumerate}
\end{problem}

\begin{problem}
\textbf{(ZF)}
Нека $B \subseteq A$ и $\pair{A, \leq}$ е добре наредено множество.
Да се докаже, че е в сила точно една от следните две възможности:
\begin{enumerate}
\item
$\pair{B, \leq \cap (B \times B)} \cong \pair{A, \leq}$.

\item
$\pair{B, \leq \cap (B \times B)}$ е изоморфно със собствен начален сегмент на $\pair{A, \leq}$.
\end{enumerate}
\end{problem}

\textbf{Решение:}

\smallbreak
\quad
Първо ще докажем, че 1. и 2. не може да бъдат изпълнени едновременно.

\begin{tcolorbox}[mybox={Доказателство:}]
\quad
Да допуснем, че $\pair{B, \leq \cap (B \times B)}$ е едновременно изоморфно на $\pair{A, \leq}$
и на собствен начален сегмент на $\pair{A, \leq}$.
Тогава от транзитивността на $\cong$ ще бъде вярно, че
$\pair{A, \leq}$ е изоморфно на собствен начален сегмент на $\pair{A, \leq}$.
Нека тогава вземем такова $x \in A$, че $\varphi$ е изоморфизъм между $A$ и $seg(x)$.\footnote{Тъй като $\pair{A, \leq}$
е добре наредено множество, то всички собствени начални сегменти на $A$ са от вида $seg(x)$ за $x \in A$.}
Тъй като $\varphi$ е биекция между $A$ и подмножество на $A$, то
$\varphi$ е инекция между $A$ и $A$, откъдето следва, че $\varphi$ е разширяваща \textit{(expansive)} функция,
тоест $\forall x \in A\ [x \leq \varphi(x)]$. В същото време $\varphi(x) \in seg(x)$, тоест $\varphi(x) < x$. Противоречие.
\end{tcolorbox}

\quad
Сега ще докажем, че винаги е изпълнено 1. или 2.
За целта нека разгледаме релацията $f \subseteq B \times A$ дефинирана по следния начин:
\[
f = \{\pair{b, a}\ |\ \pair{b, a} \in B \times A \land
                      \pair{seg_B(b), \leq \cap (seg_B(b) \times seg_B(b))} \cong \pair{seg_A(a), \leq \cap (seg_A(a) \times seg_A(a))}\}
\]

\quad
Твърдим, че $f$ е функционална релация.
\begin{tcolorbox}[mybox={Доказателство:}]
\quad
Нека $\pair{b, a_1}, \pair{b, a_2} \in f$.
Тогава от транзитивността на $\cong$ ще е изпълнено:
\[\pair{seg_A(a_1), \leq \cap (seg_A(a_1) \times seg_A(a_1))} \cong \pair{seg_A(a_2), \leq \cap (seg_A(a_2) \times seg_A(a_2))}\]

\quad
Нека допуснем, че $a_1 < a_2$.
Тогава $seg_A(a_1) \subsetneq seg_A(a_2)$, откъдето съществува изоморфизъм
между добре наредено множество и негов собствен начален сегмент.
Но ние вече доказахме, че това е невъзможно.
Така $a_1 \not < a_2$.
По същия начин можем да покажем, че $a_1 \not > a_2$, откъдето $a_1 = a_2$
Така:
\[
\forall b \in \operatorname{Dom}(f) \, \exists! a \in A
\ [
\pair{seg_B(b), \leq \cap (seg_B(b) \times seg_B(b))} \cong \pair{seg_A(a), \leq \cap (seg_A(a) \times seg_A(a))}
]
\]
\quad
откъдето $f$ е функционална релация.

\qed
\end{tcolorbox}

\quad
Твърдим, че освен това $f$ запазва наредбата и $\operatorname{Dom}(f)$ е начален сегмент на $B$.

\begin{tcolorbox}[mybox={Доказателство:}]
\quad
Нека $b \in \operatorname{Dom}(f)$ и $b' \in B$ са такива, че $b' < b$
и нека $f(b) = a$.
Така от дефиницията на $f$ съществува изоморфизъм $\varphi$ между $seg_B(b)$ и  $seg_A(a)$.
Тогава $\varphi \restriction seg_B(b')$ е изоморфизъм между $seg_B(b')$ и някакво подмножество на $seg_A(a)$.
Твърдим, че това подмножество е собствен начален сегмент на $seg_A(a)$.
\begin{tcolorbox}[mybox={Доказателство:}, colback=green!20, colframe=green!60]
\quad
Да допуснем противното.
Нека $a_1, a_2 \in seg_A(a)$ са такива, че
$b_1 = \varphi^{-1}(a_1)$,
$b_2 = \varphi^{-1}(a_2)$,
$a_1 > a_2$, $a_1 \in \operatorname{Rng}(\varphi \restriction seg_B(b'))$ и
$a_2 \notin \operatorname{Rng}(\varphi \restriction seg_B(b'))$.
Тогава $b_2 \ge b' > b_1$.
Противоречие с факта, че като изоморфно изображение, $\varphi$ запазва наредбата.

\qed
\end{tcolorbox}

\quad
Така получихме, че $\varphi \restriction seg_B(b)$ е изоморфизъм между $seg_B(b')$ и собствен начален сегмент на $seg_A(a)$,
откъдето $b' \in \operatorname{Dom}(f)$ и $f(b') < a$.

\qed
\end{tcolorbox}

\quad
По аналогичен начин може да се докаже, че $f^{-1}$ е запазваща наредбата функционална релация с
домейн начален сегмент на $A$.
Взимайки предвид тези съображения, следните четири случая са изчерпателни:

\begin{enumerate}[label={\arabic* сл.}]
\item
$\operatorname{Dom}(f) = B \land \operatorname{Rng}(f) = A$.
Тогава $f$ е свидетел за
$\pair{B, \leq \cap (B \times B)} \cong \pair{A, \leq}$.

\item
$\operatorname{Dom}(f) = B \land \operatorname{Rng}(f) = seg_A(a)$ за $a \in A$.
Тогава $f$ е изоморфизъм между $\pair{B, \leq \cap (B \times B)}$ и собствен начален сегмент на $\pair{A, \leq}$.

\item
$\operatorname{Dom}(f) = seg_B(b) \land \operatorname{Rng}(f) = A$ за $b \in B$.
Тогава $f^{-1}$ е запазваща наредбата инекция между $A$ и $A$, което влече, че $f^{-1}$ е разширяваща
тоест $\forall x \in A\ [x \leq f^{-1}(x)]$. В същото време $f^{-1}(b) \in seg_B(b)$, тоест $f^{-1}(b) < b$. Противоречие.

\item
$\operatorname{Dom}(f) = seg_B(b) \land \operatorname{Rng}(f) = seg_A(a)$ за $a \in A$ и $b \in B$.
Тогава $f$ е изоморфизъм между
$\pair{seg_B(b), \leq \cap (seg_B(b) \times seg_B(b))}$ и
$\pair{seg_A(a), \leq \cap (seg_A(a) \times seg_A(a))}$,
откъдето $\pair{b, a} \in f$. Противоречие с $\operatorname{Dom}(f) = seg_B(b)$.
\end{enumerate}

\qed

\begin{problem}
\textbf{(ZF)}
Нека $x$ е множество. Тогава $x$ е ординал точно тогава,
когато всяко транзитивно собствено подмножество на $x$ е елемент на $x$.
\end{problem}

\textbf{Решение:}

\smallbreak
\quad
$(\Rightarrow)$
В едната посока нека $x$ е ординал и нека $A$ е транзитивно собствено подмножество на $x$.
Да разгледаме множеството $A' = x \setminus A$.
Тъй като $A$ е собствено подмножество на $x$, то $A' \neq \varnothing$,
откъдето от добрата наредба на $x$ следва, че съществува елемент $y \in A'$, такъв че $y \cap A' = \varnothing$.
Нека фиксираме такъв елемент $y$.
Твърдим, че $y = A$.

\begin{tcolorbox}[mybox={Доказателство:}]
\quad
$(\Rightarrow)$ В едната посока, ще докажем, че всеки елемент на $y$ е елемент на $A$.
Нека $a \in y$. От $trans(x)$ това влече, че $a \in x$,
Но, тъй като $y \cap A' = \varnothing$, то $a \notin A'$, откъдето $a \in A$.

\quad
$(\Leftarrow)$ В обратната посока, ще докажем, че всеки елемент на $A$ е елемент на $y$.
Нека $a \in A$. Тогава като следствие от добрата нареденост на $x$ има три възможности:
$a \in y$, $y \in a$ или $a = y$.
Ако $y \in a$, то $y \in A$ от $trans(A)$.
Но $y \in A'$, откъдето този случай е невъзможен.
Ако $a = y$, то отново $y$ е едновременно в $A$ и $A'$, което е невъзможно.
Така единственият възможен случай е $a \in y$.

\quad
Показахме, че $y \subseteq A$ и $A \subseteq y$, следователно $y = A$.

\qed
\end{tcolorbox}

\quad
Накрая остава да съобразим, че $A = y \in A' \subseteq x$, откъдето $A \in x$.

\qed


\quad
$(\Leftarrow)$ В обратната посока.
Нека всяко транзитивно подмножество на $x$ е елемент на $x$
и нека $\alpha$ е най-малкият ординал, който не е елемент на $x$.
Тъй като всеки елемент $\beta$ на $\alpha$ е ординал, който е по-малък от $\alpha$,
то $x$ ще съдържа всеки елемент на $\alpha$, тоест $\alpha \subseteq x$.
Разглеждаме два случая:
\begin{enumerate}
\item
$\alpha \subsetneq x$
Тогава $\alpha$ е транзитивно подмножество на $x$, откъдето от условието $\alpha \in x$.
Противоречие с избора на $\alpha$.

\item
$\alpha = x$. Тогава $x$ е ординал.
\end{enumerate}

\qed


\begin{problem}
\textbf{(ZFC)}
Нека $\pair{A, \leq}$ е линейно наредено множество, което няма най-голям елемент.
За едно множество $X \subseteq A$, се казва, че е кофинално с $A$, ако:
\[
(\forall a \in A)(\exists x \in X)(a \leq x)
\]

\quad
Докажете, че съществува $B \subseteq A$, което е кофинално с $A$ и $\pair{B, \leq}$ е добре наредено множество.
\end{problem}

\textbf{Решение:}

\smallbreak
\quad
Нека $h: \pow(A) \setminus \varnothing \to A$ е произволна функция на избора,
нека $*$ е такова множество, че $* \notin A$,
нека $r: \pow(A) \to A \cup \{*\}$ е дефинирана като $r \coloneq h \cup {\pair{\varnothing, *}}$ и
нека $G$ е операцията определена от следното свойство:
\[
\varphi(x, y) \coloneq y = r( \{b\ |\ b \in A \land \forall c \in \operatorname{Rng}(x) \cap A \ [c < b] \})
\]

\quad
Тогава от теоремата за трансфинитна рекурсия, съществува определима операция
$F$ такава, че:
\[
(\forall \alpha)[F(\alpha) = G(F \restriction \alpha)]
\]

\quad
Нека $\alpha$ е произволен ординал.

\quad
Твърдим, че ако $* \notin \operatorname{Rng}(F \restriction \alpha)$
(тоест, ако $\operatorname{Rng}(F \restriction \alpha) \subseteq A$), то $(F \restriction \alpha)$ е инективна.
\begin{tcolorbox}[mybox={Доказателство:}]
\begin{remark}
За компактност на записа в рамките на доказателството ще бележим $(F \restriction \alpha)$ с $f$.
\end{remark}

\smallbreak
\quad
Нека $x_1, x_2 \in \alpha$ са такива, че $f(x_1) = f(x_2)$.
Да допуснем, че $x_1 < x_2$.
Тогава:
\begin{alignat*}{2}
f(x_2) & = G(F \restriction x_2)                                                                              & \text{ // от дефиницията на $F$ } \\
       & = r( \{b\ |\ b \in A \land \forall c \in \operatorname{Rng}(F \restriction x_2) \cap A \ [c < b] \}) & \text{ // от дефиницията на $G$ } \\
       & = r( \{b\ |\ b \in A \land \forall c \in \operatorname{Rng}(F \restriction x_2) \ [c < b] \})        & \text{ // от $\operatorname{Rng}(F \restriction x_2) \subseteq \operatorname{Rng}(F \restriction \alpha) \subseteq A$ } \\
       & = h( \{b\ |\ b \in A \land \forall c \in \operatorname{Rng}(F \restriction x_2) \ [c < b] \})        & \text{ // от $f(x_2) \in \operatorname{Rng}(f) \not\ni *$ }
\end{alignat*}

\quad
Тъй като $f(x_1) \in \operatorname{Rng}(F \restriction x_2)$ и $\forall c \in \operatorname{Rng}(F \restriction x_2)\ [f(x_2) > c]$,
то $f(x_1) < f(x_2)$. Противоречие с $f(x_1) = f(x_2)$.
Така $x_1 \not < x_2$.
По аналогичен начин може да се докаже, че $x_1 \not > x_2$, откъдето $x_1 = x_2$.

\quad
Щом за произволни $x_1, x_2 \in \alpha$, $f(x_1) = f(x_2)$ влече, че $x_1 = x_2$, то $f$ е инекция.

\qed
\end{tcolorbox}

\quad
Твърдим, че освен това ако $* \notin \operatorname{Rng}(F \restriction \alpha)$, то
$(F \restriction \alpha)$ запазва наредбата на $\alpha$.
\begin{tcolorbox}[mybox={Доказателство:}]
\quad
В доказателството на предишното твърдение изведохме, че за произволни $x_1, x_2 \in \alpha$,
$x_1 < x_2$ влече, че $f(x_1) < f(x_2)$.

\qed
\end{tcolorbox}


\quad
Твърдим, че също така съществува ординал $\alpha$, за който $F(\alpha) = *$.
\begin{tcolorbox}[mybox={Доказателство:}]
\quad
Нека $\beta$ е произволен ординал, за който $\size{\beta} > \size{A}$.
Ако допуснем, че $*$ не принадлежи на никой ординал, то
от първото твърдение, $F \restriction \beta$ ще е инективна функция от $\beta$ в $A$.
Така $\size{\beta} > \size{A} \land \size{\beta} \leq \size{A}$. Противоречие.
%откъдето по теоремата Кантор-Шрьодер-Бернщайн $\size{A} = \size{\beta}$,
%което противоречи на избора на $\beta$.

\qed
\end{tcolorbox}

\quad
Нека сега вземем $\alpha$ да е най-малкият ординал, за който $F(\alpha) = *$.
Така от свойствата на ординалите ще бъде изпълнено, че:
\[
(\forall \beta)(F(\beta) = * \iff \beta \ge \alpha)
\]

\quad
Нека $B \coloneq \operatorname{Rng}(F \restriction \alpha)$.
Да забележим, че $B$ е подмножество на $A$, тъй като $* \notin \operatorname{Rng}(F \restriction \alpha)$.

\quad
Твърдим, че освен това $B$ е кофинално с $A$.
\begin{tcolorbox}[mybox={Доказателство:}]
\quad
Да допуснем противното.
Нека $a \in A$ е такова, че $\forall b \in B$ $(a > b)$.
Тогава $\{b\ |\ b \in A \land \forall c \in B \ [c < b] \} \neq \varnothing$,
откъдето $r(\{b\ |\ b \in A \land \forall c \in B \ [c < b] \}) = h(\{b\ |\ b \in A \land \forall c \in B \ [c < b] \}) \neq *$.
Но тогава $F(\alpha) \neq *$. Противоречие.

\qed
\end{tcolorbox}

\quad
Накрая остава да съобразим, че тъй като $F \restriction \alpha$ е запазваща наредбата биекция между $\alpha$ и $B$,
и тъй като като ординал $\alpha$ е добре наредено множество, то $B$ също е добре наредено множество.

\qed

\begin{problem}
\textbf{(ZF)}
За произволно множество $A$ от реални числа и за произволно реално число $r$ с $r + A$ означаваме множеството
\[
\{x\ |\ x \in \mathbb{R} \land (\exists y \in A)(x = r + y)\}.
\]

\quad
Нека $A$ е най-много изброимо множество от реални числа.
Да се докаже, че има такова реално число $r$, че $A \cap (r+A) = \varnothing$.
\end{problem}

\textbf{Решение:}

\smallbreak
\quad
Нека фиксираме една инекция от $f$ от $A$ в $\omega$.
Ще извършим няколко наблюдения:

\smallbreak

\quad
\textbf{Наблюдение 1:}
$A$ е добре наредимо множество.
\begin{tcolorbox}[mybox={Доказателство:}]
\quad
Нека $\leq$ е релация над $A$ дефинирана по следния начин:
\[
\leq = \{\pair{a, b}\ |\ \pair{a, b} \in A \times A \land f(a) \leq f(b) \}
\]
\quad
С тази дефиниция добрата нареденост на $\pair{A, \leq}$ следва директно от добрата нареденост на $\omega$.

\qed
\end{tcolorbox}

\quad
Отсега нататък нека си фиксираме една добра наредба $\leq_A$ над $A$.

\smallbreak

\quad
\textbf{Наблюдение 2:}
$A \times A$ е най-много изброимо.
\begin{tcolorbox}[mybox={Доказателство:}]
\quad
Тъй като $\omega^2$ е изброимо, ще бъде достатъчно да покажем съществуването на инекция от $A \times A$ в $\omega^2$.
Дефинираме $g: A \times A \to \omega^2$ по следния начин:
\[
g(a, b) = \pair{f(a), f(b)}
\]

\quad
Инективността на $g$ следва директно от инективността на $f$.

\qed
\end{tcolorbox}

\quad
\textbf{Наблюдение 3:}
$A \times A$ е добре наредимо множество. (директно следствие от \textbf{Задача 12}).

\quad
Отсега нататък нека си фиксираме една добре наредба $\leq_{A \times A}$ над $A \times A$.

\quad
Да разгледаме множеството $S$ от разликите на елементи в $A$:
\[
S = \{r\ |\ r \in \mathbb{R} \land \exists \pair{s, t} \in A \times A\ [s - t = r] \}
\]

\quad
Твърдим, че също като $A \times A$, $S$ е най-много изброимо.

\begin{tcolorbox}[mybox={Доказателство:}]
\quad
Нека функцията $h: S \to A \times A$ е дефинирана по следния начин:
\[
h(r) = \pair{a, b} \text{ за $\pair{a, b} = min_{\leq}\{\pair{c, d}\ |\ \pair{c, d} \in A \times A \land |c - d| = r \}$}
\]

\quad
Ще докажем, че $h$ е инекция:
\begin{tcolorbox}[mybox={Доказателство:}, colback=green!20, colframe=green!60]
\quad
Нека $\pair{a_1, b_1} = h(r_1) = h(r_2) = \pair{a_2, b_2}$.
Ако $r_1 \neq r_2$, то от дефиницията на $h$ имаме, че $a_1 - b_1 = r_1 \neq r_2 = a_2 - b_2$,
откъдето $\pair{a_1, b_1} \neq \pair{a_2, b_2}$. Противоречие. Така $r_1 = r_2$.

\qed
\end{tcolorbox}

\quad
Накрая, от транзитивността на $\leq$, щом
$\size{S} \leq \size{A \times A}$
и
$\size{A \times A} \leq \size{\omega}$,
то $\size{S} \leq \size{\omega}$

\qed
\end{tcolorbox}

\quad
Щом $S$ е най-много изброимо множество от реални числа, то съществува
реално число, което не е в $S$.
Нека $r$ е едно такова число.
Твъдим, че $A \cap (r + A) = \varnothing$.
\begin{tcolorbox}[mybox={Доказателство:}]
\quad
Да допуснем противното. Нека $s \in A \cap (r + A)$.
От дефиницията на $r + A$ последното влече, че $(s - r) \in A$.
Тогава за $\pair{s, s-r} \in A \times A$ е изпълнено, че $s - (s-r) = r$,
откъдето $r \in S$. Противоречие с $r \notin S$.

\qed
\end{tcolorbox}

\qed



% Needs ZFC
%
%Нека допуснем противното, тоест нека за всяко реално число $A \cap (r + A) \neq \varnothing$.
%
%\quad
%Тъй като $A$ e най-много изброимо множество, то може да бъде добре наредено.
%Нека вземем произволна добра наредба $\pair{A, \leq}$ над $A$ и нека
%$f: \mathbb{R} \to A$ е следната функция:
%\[
%f(r) = min_{\leq}(A \cap (r + A))
%\]
%
%\quad
%Нека сега за всяко $a \in A$ си дефинираме $R_a$ като множеството от реални числа първообрази на $a$ относно $f$:
%\[
%R_a = \{r\ |\ r \in \mathbb{R} \land f(r) = a \}
%\]
%
%\quad
%Да съобразим, че индексираната фамилия $\{R_a\}_{a \in A}$ е разбиване (с възможни празни множества) на $\mathbb{R}$,
%тъй като $f$ е тотална функция.
%Освен това, тъй като $A$ е най-много изброимо, а $\mathbb{R}$ не е изброимо,
%то съществува $a \in A$ такова, че $R_a$ е неизброимо.
%Нека фиксираме едно такова $a$. Тогава:
%\[
%\forall r \in R_a\ [a \in r + A]
%\]
%
%\quad
%откъдето:
%\[
%\forall r \in R_a\ [a - r  \in A]
%\]
%
%\quad
%Така получихме, че $A$ има неизброимо много елементи, което е противоречие.
%
%\qed
%
%


\end{document}
