\begin{problem}
\textbf{(ZF)}
Нека $B \subseteq A$ и $\pair{A, \leq}$ е добре наредено множество.
Да се докаже, че е в сила точно една от следните две възможности:
\begin{enumerate}
\item
$\pair{B, \leq \cap (B \times B)} \cong \pair{A, \leq}$.

\item
$\pair{B, \leq \cap (B \times B)}$ е изоморфно със собствен начален сегмент на $\pair{A, \leq}$.
\end{enumerate}
\end{problem}

\textbf{Решение:}

\smallbreak
\quad
Първо ще докажем, че 1. и 2. не може да бъдат изпълнени едновременно.

\begin{tcolorbox}[mybox={Доказателство:}]
\quad
Да допуснем, че $\pair{B, \leq \cap (B \times B)}$ е едновременно изоморфно на $\pair{A, \leq}$
и на собствен начален сегмент на $\pair{A, \leq}$.
Тогава от транзитивността на $\cong$ ще бъде вярно, че
$\pair{A, \leq}$ е изоморфно на собствен начален сегмент на $\pair{A, \leq}$.
Нека тогава вземем такова $x \in A$, че $\varphi$ е изоморфизъм между $A$ и $seg(x)$.\footnote{Тъй като $\pair{A, \leq}$
е добре наредено множество, то всички собствени начални сегменти на $A$ са от вида $seg(x)$ за $x \in A$.}
Тъй като $\varphi$ е биекция между $A$ и подмножество на $A$, то
$\varphi$ е инекция между $A$ и $A$, откъдето следва, че $\varphi$ е разширяваща \textit{(expansive)} функция,
тоест $\forall x \in A\ [x \leq \varphi(x)]$. В същото време $\varphi(x) \in seg(x)$, тоест $\varphi(x) < x$. Противоречие.
\end{tcolorbox}

\quad
Сега ще докажем, че винаги е изпълнено 1. или 2.
За целта нека разгледаме релацията $f \subseteq B \times A$ дефинирана по следния начин:
\[
f = \{\pair{b, a}\ |\ \pair{b, a} \in B \times A \land
                      \pair{seg_B(b), \leq \cap (seg_B(b) \times seg_B(b))} \cong \pair{seg_A(a), \leq \cap (seg_A(a) \times seg_A(a))}\}
\]

\quad
Твърдим, че $f$ е функционална релация.
\begin{tcolorbox}[mybox={Доказателство:}]
\quad
Нека $\pair{b, a_1}, \pair{b, a_2} \in f$.
Тогава от транзитивността на $\cong$ ще е изпълнено:
\[\pair{seg_A(a_1), \leq \cap (seg_A(a_1) \times seg_A(a_1))} \cong \pair{seg_A(a_2), \leq \cap (seg_A(a_2) \times seg_A(a_2))}\]

\quad
Нека допуснем, че $a_1 < a_2$.
Тогава $seg_A(a_1) \subsetneq seg_A(a_2)$, откъдето съществува изоморфизъм
между добре наредено множество и негов собствен начален сегмент.
Но ние вече доказахме, че това е невъзможно.
Така $a_1 \not < a_2$.
По същия начин можем да покажем, че $a_1 \not > a_2$, откъдето $a_1 = a_2$
Така:
\[
\forall b \in \operatorname{Dom}(f) \, \exists! a \in A
\ [
\pair{seg_B(b), \leq \cap (seg_B(b) \times seg_B(b))} \cong \pair{seg_A(a), \leq \cap (seg_A(a) \times seg_A(a))}
]
\]
\quad
откъдето $f$ е функционална релация.

\qed
\end{tcolorbox}

\quad
Твърдим, че освен това $f$ запазва наредбата и $\operatorname{Dom}(f)$ е начален сегмент на $B$.

\begin{tcolorbox}[mybox={Доказателство:}]
\quad
Нека $b \in \operatorname{Dom}(f)$ и $b' \in B$ са такива, че $b' < b$
и нека $f(b) = a$.
Така от дефиницията на $f$ съществува изоморфизъм $\varphi$ между $seg_B(b)$ и  $seg_A(a)$.
Тогава $\varphi \restriction seg_B(b')$ е изоморфизъм между $seg_B(b')$ и някакво подмножество на $seg_A(a)$.
Твърдим, че това подмножество е собствен начален сегмент на $seg_A(a)$.
\begin{tcolorbox}[mybox={Доказателство:}, colback=green!20, colframe=green!60]
\quad
Да допуснем противното.
Нека $a_1, a_2 \in seg_A(a)$ са такива, че
$b_1 = \varphi^{-1}(a_1)$,
$b_2 = \varphi^{-1}(a_2)$,
$a_1 > a_2$, $a_1 \in \operatorname{Rng}(\varphi \restriction seg_B(b'))$ и
$a_2 \notin \operatorname{Rng}(\varphi \restriction seg_B(b'))$.
Тогава $b_2 \ge b' > b_1$.
Противоречие с факта, че като изоморфно изображение, $\varphi$ запазва наредбата.

\qed
\end{tcolorbox}

\quad
Така получихме, че $\varphi \restriction seg_B(b)$ е изоморфизъм между $seg_B(b')$ и собствен начален сегмент на $seg_A(a)$,
откъдето $b' \in \operatorname{Dom}(f)$ и $f(b') < a$.

\qed
\end{tcolorbox}

\quad
По аналогичен начин може да се докаже, че $f^{-1}$ е запазваща наредбата функционална релация с
домейн начален сегмент на $A$.
Взимайки предвид тези съображения, следните четири случая са изчерпателни:

\begin{enumerate}[label={\arabic* сл.}]
\item
$\operatorname{Dom}(f) = B \land \operatorname{Rng}(f) = A$.
Тогава $f$ е свидетел за
$\pair{B, \leq \cap (B \times B)} \cong \pair{A, \leq}$.

\item
$\operatorname{Dom}(f) = B \land \operatorname{Rng}(f) = seg_A(a)$ за $a \in A$.
Тогава $f$ е изоморфизъм между $\pair{B, \leq \cap (B \times B)}$ и собствен начален сегмент на $\pair{A, \leq}$.

\item
$\operatorname{Dom}(f) = seg_B(b) \land \operatorname{Rng}(f) = A$ за $b \in B$.
Тогава $f^{-1}$ е запазваща наредбата инекция между $A$ и $A$, което влече, че $f^{-1}$ е разширяваща
тоест $\forall x \in A\ [x \leq f^{-1}(x)]$. В същото време $f^{-1}(b) \in seg_B(b)$, тоест $f^{-1}(b) < b$. Противоречие.

\item
$\operatorname{Dom}(f) = seg_B(b) \land \operatorname{Rng}(f) = seg_A(a)$ за $a \in A$ и $b \in B$.
Тогава $f$ е изоморфизъм между
$\pair{seg_B(b), \leq \cap (seg_B(b) \times seg_B(b))}$ и
$\pair{seg_A(a), \leq \cap (seg_A(a) \times seg_A(a))}$,
откъдето $\pair{b, a} \in f$. Противоречие с $\operatorname{Dom}(f) = seg_B(b)$.
\end{enumerate}

\qed
