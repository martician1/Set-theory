\begin{problem}
Да се докаже, че за произволно множество $A$ са в сила следните:
\begin{enumerate}
\item
$\size{A} = \size{A \cup \{A\}} \Rightarrow \size{\pow(A)} = \size{\pow(A) \cup \{\pow(A)\}}$
\item
$\size{A} = \size{A \cup \{A\}} \Rightarrow \size{\pow(\pow(A))} = \size{\pow(\pow(A)) \times \pow(\pow(A))}$
\end{enumerate}
%\textbf{Упътване.}
%\textit{За 2) използвайте, че ${}^X 2$ и $\pow(X)$ са равномощни, както и че винаги,
%когато $B \cap C = \varnothing$, множествата ${}^{B \cup C} X$ и ${}^B X \times {}^C X$ са равномощни.}
\end{problem}

\textbf{Решение:}

\smallbreak
\quad
Първо ще докажем 1). Нека $\size{A} = \size{A \cup \{A\}}$ и нека $f: A \to A \cup \{A\}$ е произволна биекция.
Дефинираме функция $g: \pow(A) \to \pow(A) \cup \{\pow(A)\}$ по следния начин:
\[
g(x) =
\begin{cases}
\pow(A) & \text{, ако $\exists y\ [x = \{y\} \land f(y) = A]$} \\
\{f(\union x)\} & \text{, иначе, ако $\exists y\ [x = \{y\}]$} \\
x & \text{, иначе}
\end{cases}
\]

\quad
Ще докажем, че $g$ е инекция:
\begin{tcolorbox}[mybox={Доказателство:}]
\quad
Нека $g(x_1) = g(x_2)$. Разглеждаме четири случая:
\begin{enumerate}[label={\arabic* сл.}]
\item
$g(x_1) = \pow(A) = g(x_2)$.
Тогава
$\exists y_1\ [x_1 = \{y_1\} \land f(y_1) = A] \land
\exists y_2\ [x_2 = \{y_2\} \land f(y_2) = A]$.
Тъй като $f$ е инекция, последното влече, че $y_1 = y_2$, откъдето $x_1 = x_2$.

\item
$g(x_1) \ne \pow(A) \ne g(x_2) \land \exists y_1\ [x_1 = \{y_1\}] \land \exists y_2\ [x_2 = \{y_2\}]$.
Тогава $g(x_1) = \{f(y_1)\}$ и $g(x_2) = \{f(y_2)\}$.
Тъй като $g(x_1) = g(x_2)$, последното влече, че
$f(y_1) = f(y_2)$, откъдето $y_1 = y_2 $ и $x_1 = \{y_1\} = \{y_2\} = x_2$

\item
$\exists y_1\ [x_1 = \{y_1\}] \xor \exists y_2\ [x_2 = \{y_2\}]$.
Но това влече, че $\size{g(x_1)} \ne \size{g(x_2)}$, което е невъзможно.

\item
$\neg \exists y_1\ [x_1 = \{y_1\}] \land \neg \exists y_2\ [x_2 = \{y_2\}]$.
Тогава $x_1 = g(x_1) = g(x_2) = x_2$.
\end{enumerate}
\qed


\end{tcolorbox}

\quad
Ще докажем, че $g$ е сюрекция:
\begin{tcolorbox}[mybox={Доказателство:}]
\quad
Нека $y \in \pow(A) \cup \{\pow(A)\}$.
Разглеждаме три случая:
\begin{enumerate}[label={\arabic* сл.}]
\item
$y = \pow(A)$. Тогава можем да вземем като първообраз на $y$ множество $x \in \pow(A)$ от вида $x = \{z\}$,
където $z \in A \land f(z) = \{A\}$.

\item
$y \in \pow(A) \land \exists z\ [y = \{z\}]$
Нека $s = f^{-1}(z)$. Тогава $g(\{s\}) = \{f(\union \{s\})\} = \{f(s)\} = \{z\} = y$.

\item
$y \in \pow(A) \land \neg \exists z\ [y = \{z\}]$.
Тогава $g(y) = y$.
\end{enumerate}
\qed
\end{tcolorbox}

\quad
Щом $g$ е инекция и сюрекция, то $g$ е биекция $\Rightarrow$
$\size{\pow(A)} = \size{\pow(A) \cup \{\pow(A)\}}$.

\qed

% \quad
% Нека $t$ е произволен елемент, такъв, че $t \notin Rng(\pow(A))$.
% За да докажем 2) е достатъчно е да построим биекция $h: \pow(\pow(A)) \to \pow(\pow(A) \cup (\pow(A) \times \{t\}))$,
% тъй като
% \begin{alignat*}{3}
% \size{\pow(\pow(A)) \times \pow(\pow(A))} &= \size{{}^{\pow(A)} 2 \times {}^{\pow(A)} 2}  && \text{ // от упътването} &\\
%                                           &= \size{{}^{\pow(A)} 2 \times {}^{\pow(A) \times \{t\}} 2}  && \text{ // от свойствата на декартовото произведение} &\\
% 										  & = \size{{}^{\pow(A) \cup (\pow(A) \times \{t\})} 2} && \text{ // от упътването} & \\
% 										  & = \size{\pow(\pow(A) \cup (\pow(A) \times \{t\}))} && \text{ // от упътването} &
% \end{alignat*}
%
% Дефинираме $h: \pow(\pow(A)) \to \pow(\pow(A) \cup (\pow(A) \times \{t\}))$ по следния начин:
% %\[
% %h(x) = \{
% %y\ |\ y \in \pow(\pow(A) \cup (\pow(A) \times \{t\})) \land
% %( \exists a \in x\ \forall v\ [v \in a \iff (f(v) \in y \xor f(v) = A)])
% %\}
% %\]
% % \begin{alignat*}{1}
% % h(x) = \{ & y\ |\ y \in \pow(\pow(A) \cup (\pow(A) \times \{t\})) \land \exists a \in x[ \\
% %           & (\forall v\ [(v \in a \Rightarrow f(v) \ne A) \land (v \in a \iff f(v) \in y)]) \lor \\
% %           & (y \subseteq \pow(A) \times \{t\} \land ( \exists w \in a\ [f(w) = A \land \forall v\ [v \in a \setminus\{w\} \iff f(v) \in Dom(y)]]))]
% % \}
% % \end{alignat*}
% \begin{alignat*}{1}
% h(x) = \{ & y\ |\ y \in \pow(\pow(A) \cup (\pow(A) \times \{t\})) \land \forall z \in y\ \exists p \in x\ [ \\
%           & (\forall v\ [v \in p \iff f(v) \in z]) \lor \\
% 		  & (Rng(z) = t \land \exists v \in p\ [f(v) = A ] \land \forall v\ [v \in p \iff f(v) \in Dom(z) \xor f(v) = A])]
% \}
% \end{alignat*}
%
% Ще докажем, че $h$ е сюрекция:
%
% \begin{tcolorbox}[mybox={Доказателство:}]
% \quad
% Нека $y \in \pow(\pow(A) \cup (\pow(A) \times \{t\}))$. Нека:
% \[
% Z_1 = \{x\ |\ x \in y \land x \in \pow(A)\}
% \]
% \[
% Z_2 = \{x\ |\ x \in y \land x \in \pow(A) \times {t}\}
% \]
% Забележете, че $\{Z_1, Z_2\}$ е разбиване на $y$.
%
% \smallbreak
% Нека $x$ е следното множество:
% \[
% x = \{a\ |\ a \in \pow(A) \land [f[a] \in y \xor (\exists b \in a\ [f(b) = A \land f[a] \setminus b \in Dom(y)])] \}
% \]
% Твърдим, че $x$ е първообраз на $y$. Наистина
%
% \bigbreak
% Разглеждаме четири случая:
% %\begin{enumerate}[label={\arabic* сл.}]
% %
% %\end{enumerate}
% \qed
%
% \end{tcolorbox}
%
% Ще докажем, че $h$ е инекция.
%
% \begin{tcolorbox}[mybox={Доказателство:}]
% \quad
% Нека $h(x_1) = h(x_2)$. Разглеждаме четири случая:
% \qed
%
% \end{tcolorbox}

% Нека сега докажем 2). Дефинираме $h: \pow(\pow(A)) \to \pow(\pow(A)) \times \pow(\pow(A))$ по следния начин:
% \[
% h(x) = \pair{y, z} \text{ за } y = \{a \in \pow(A)\ |\ f^{-1}[a] \in x\} \land z = \{b \in \pow(A) \ |\ f^{-1}[b \cup \{A\}] \in x\}
% \]

\bigbreak
\quad
Сега ще докажем 2). Нека $c = f^{-1}(A)$. Дефинираме функция $h: \pow(\pow(A)) \to \pow(\pow(A)) \times \pow(\pow(A))$ по следния начин:
\[
h(x) = \pair{y, z} \text{ за } y = \{f[a]\ |\ a \in x \land c \notin a\} \land z = \{f[a \setminus \{c\}]\ |\ a \in x \land c \in a\}
\]
\quad
Ще докажем, че $h$ е инекция:

\begin{tcolorbox}[mybox={Доказателство:}]
\quad
Нека $h(x_1) = \pair{y, z} = h(x_2)$. Тогава:
\[
\{f[a]\ |\ a \in x_1 \land c \notin a\} = y = \{f[a]\ |\ a \in x_2 \land c \notin a\}
\]
\[
\{f[a \setminus \{c\}]\ |\ a \in x_1 \land c \in a\} = z = \{f[a \setminus \{c\}]\ |\ a \in x_2 \land c \in a\}
\]
\quad
Нека $b$ е произволно множество. Разглеждаме два случая:
\begin{enumerate}[label={\arabic* сл.}]
\item
$c \notin b$. Тогава:
\[
b \in x_1 \iff f[b] \in y \iff b \in x_2
\]
\item
$c \in b$. Тогава:
\[
b \in x_1 \iff f[b \setminus \{c\}] \in z \iff b \in x_2
\]
\end{enumerate}
\quad
Така от аксиомата за обемност $x_1 = x_2$.

\qed
\end{tcolorbox}

\quad
Ще докажем, че $h$ е сюрекция:

\begin{tcolorbox}[mybox={Доказателство:}]
\quad
Нека $\pair{y, z} \in \pow(\pow(A)) \cup \pow(\pow(A))$. Дефинираме множество първообраз $x$ по следния начин:
\[
x = \underbrace{\{f^{-1}[b]\ |\ b \in y\}}_{M} \uplus \underbrace{\{f^{-1}[b] \cup \{c\}\ |\ b \in z\}}_{N}
\]
\quad
Отсега отбелязваме, че $M$ и $N$ са чужди множества, тъй като
$\forall x\ [x \in M \Rightarrow c \notin x] \land [x \in N \Rightarrow c \in x]$.
Нека $h(x) = \pair{y', z'}$.
В сила е следното:
\begin{alignat*}{2}
b \in y & \iff f^{-1}[b] \in x      & \text{ // от $f^{-1}[b] \notin N$ и дефиницията на $x$} \\
        & \iff f[f^{-1}[b]] \in y'  & \text{ // от $c \notin f^{-1}[b]$ и дефиницията на $h$} \\
		& \iff b \in y'             &
\end{alignat*}
\quad
По същия начин:
\begin{alignat*}{2}
b \in z & \iff f^{-1}[b] \cup \{c\} \in x                         & \text{ // от $f^{-1}[b] \cup \{c\} \notin M$ и дефиницията на $x$} \\
        & \iff f[(f^{-1}[b] \cup \{c\}) \setminus \{c\}] \in z'   & \text{ // от $c \in f^{-1}[b] \cup \{c\}$ и дефиницията на $h$} \\
        & \iff f[f^{-1}[b]] \in z'                                & \text{ // от $c \notin f^{-1}[b]$} \\
		& \iff b \in z'                                           &
\end{alignat*}

\quad
Така получихме, че $h(x) = \pair{y, z} \Rightarrow \forall \pair{y, z}\, \exists x\ [h(x) = \pair{y, z}]$.

\qed
\end{tcolorbox}

\quad
Щом $h$ е инекция и сюрекция, то $h$ е биекция
$\Rightarrow \size{\pow(\pow(A))} = \size{\pow(\pow(A)) \times \pow(\pow(A))}$.

\qed


%\begin{tcolorbox}[mybox={Доказателство:}]
%\quad
%Нека $h(x_1) = h(x_2)$. Разглеждаме четири случая:
%\qed

%\end{tcolorbox}

