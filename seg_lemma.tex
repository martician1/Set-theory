\quad
За 9-та и 10-та задача ще използваме следните три леми:

\begin{tcolorbox}[mybox={Лема 1}, colback=purple!20, colframe=purple!40]
\quad
За всяко линейно наредено множество $\pair{A, \leq}$ е в сила:
\[
\forall a, b \in A\ [ seg(a) \subseteq seg(b) \lor seg(b) \subseteq seg(a)]
\]
\end{tcolorbox}

\begin{tcolorbox}[mybox={Доказателство:}]
\quad
Нека $a$ и $b$ са произволни елементи на $A$.
Тогава:
\[
\forall c \in seg(a) \ [c < a]
\]
\[
\forall c \in seg(b) \ [c < b]
\]

\quad
Разглеждаме два случая:
\begin{enumerate}[label={\arabic* сл.}]
\item
$a<b$.
Тогава $\forall c \in seg(a) \ [c<a \land a<b]$, тоест $\forall c \in seg(a) \ [c<b]$, откъдето $seg(a) \subseteq seg(b)$.

\item
$b<a$.
Тогава $\forall c \in seg(b) \ [c<b \land b<a]$, тоест $\forall c \in seg(b) \ [c<a]$, откъдето $seg(b) \subseteq seg(a)$.

\end{enumerate}
\qed
\end{tcolorbox}

\begin{tcolorbox}[mybox={Лема 2}, colback=purple!20, colframe=purple!40]

\quad
Нека $\pair{A, \leq}$ е непразно линейно наредено множество и нека $X$ е такова, че:
\begin{itemize}
\item
$X \neq \varnothing$
\item
$(\forall y \in X)(\exists x \in A)(y = seg(x))$
\item
$(\exists x)(\intersection X = seg(x))$
\end{itemize}

\quad
Тогава $\intersection X \in X$.
\end{tcolorbox}

\begin{tcolorbox}[mybox={Доказателство:}]
\quad
Да допуснем противното.
Нека $X$ е множество изпълняващо условията на лемата, нека $x$ е такова, че $X = seg(x)$ и нека $\intersection X \notin X$.
Разглеждаме два случая:
\begin{enumerate}[label={\arabic* сл.}]
\item
$\forall a \in X\ [seg(x) \subsetneq a]$.
Тогава $\forall a \in X\ [x \in a]$, тоест $x \in \intersection X = seg(x)$. Противоречие.
\item
$\exists a \in X\ [a \subsetneq seg(x) ]$.
Тогава $a \subsetneq seg(x) = \intersection X \subseteq a$. Противоречие.
\end{enumerate}

\quad
От \textbf{Лема 1} тези 2 случая са изчерпателни.
Така допускането ни се оказа грешно, откъдето $\intersection X \in X$.

\qed


\end{tcolorbox}

\begin{tcolorbox}[mybox={Лема 3}, colback=purple!20, colframe=purple!40]

\quad
Нека $\pair{A, \leq}$ е непразно линейно наредено множество, нека $y$ е непразно подмножество на $A$
нека $X$ е следното множество:
\[
X = \{ a\ |\ a \in \pow(A) \land \exists b\ [ b \in y \land  a = seg(b)] \}
\]
\quad
и нека $\intersection X = seg(x) \in X$. Тогава $y$ има най-малък елемент $x$.
\end{tcolorbox}

\begin{tcolorbox}[mybox={Доказателство:}]
\quad
Да допуснем противното.
Нека $x' \in y$ е такова, че $x' < x$.
Тогава $x' \in seg(x) = \intersection X \subseteq seg(x')$.
Противоречие с $x' \notin seg(x')$.

\qed
\end{tcolorbox}
