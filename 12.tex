\begin{problem}
\textbf{(ZF)}
Нека $\pair{A, \leq}$ е добре наредено множество.
В $A \times A$ дефинираме бинарната релация $\leq^{can}$ така:

\smallbreak
\quad
За произволни $a_1, a_2, b_1$ и $b_2$ от $A$, $\pair{\pair{a_1, b_1}, \pair{a_2, b_2}} \in \leq^{can}$
точно тогава, когато:
\begin{align*}
\operatorname{max}_{\leq}\{a_1, b_1\} & < \operatorname{max}_{\leq}\{a_2, b_2\} \text{ или} \\
\operatorname{max}_{\leq}\{a_1, b_1\} & = \operatorname{max}_{\leq}\{a_2, b_2\} \land a_1 < a_2 \text{ или} \\
\operatorname{max}_{\leq}\{a_1, b_1\} & = \operatorname{max}_{\leq}\{a_2, b_2\} \land a_1 = a_2 \land b_1 \leq b_2
\end{align*}
\quad
Да се докаже, че $\pair{A \times A, \leq^{can}}$ е добре наредено множество.
\end{problem}

\textbf{Решение:}

\smallbreak
\quad
Нека $X$ е непразно подмножество на $A \times A$. Ще докажем, че $X$ има най-малък елемент относно дефинираната в условието релация.
За целта ще използваме шест функции: $\operatorname{max}, \operatorname{fst}, \operatorname{snd:} A \times A \to A;$
$f, g, h: \pow(X) \to \pow(X)$, дефинирани по следните начини:
\[
\operatorname{max}(a, b) = \operatorname{max}_{\leq}\{a, b\}
\]
\[
\operatorname{fst}(a, b) = a
\]
\[
\operatorname{snd}(a, b) = b
\]
\[
f(B) = \{b\ |\ b \in B \land \operatorname{max}(b) = \operatorname{min}_{\leq}(\operatorname{max}[B])\}
\]
\[
g(B) = \{b\ |\ b \in B \land \operatorname{fst}(b) = \operatorname{min}_{\leq}(\operatorname{fst}[B])\}
\]
\[
h(B) = \{b\ |\ b \in B \land \operatorname{snd}(b) = \operatorname{min}_{\leq}(\operatorname{snd}[B])\}
\]

\quad
Нека сега разгледаме множеството $C \coloneq h(g(f(X)))$.
Да забележим, че от дефиницциите на $f, g$ и $h$ и от $X \neq \varnothing$ следва, че то не е празно.
Нека тогава $c = \pair{c_1, c_2}$ е произволен елемент на $C$.
Твърдим, че $c$ е най-малкият елемент на $X$.

\begin{tcolorbox}[mybox={Доказателство:}]
\quad
Нека $a = \pair{a_1, a_2} \in X$.
Да допуснем, че $a < c$
Тогава е изпълнено едно от трите:
\begin{enumerate}[label={\arabic* сл.}]
\item
$\operatorname{max}_{\leq}\{a_1, a_2\} < \operatorname{max}_{\leq}\{c_1, c_2\}$.
Тогава $c \notin f(X)$, откъдето
$c \notin g(f(X))$ и $c \notin h(g(f(X))) = C$.
Противоречие с $c \in C$.

\item
$\operatorname{max}_{\leq}\{a_1, a_2\} = \operatorname{max}_{\leq}\{c_1, c_2\} \land a_1 < c_1$.
Тогава $c \notin g(f(X))$,
откъдето $c \notin h(g(f(X))) = C$.
Противоречие с $c \in C$.

\item
$\operatorname{max}_{\leq}\{a_1, a_2\} = \operatorname{max}_{\leq}\{c_1, c_2\} \land a_1 = c_1 \land a_2 \leq c_2$.
Ако $a_2 < c_2$, то $c \notin h(g(f(X))) = C$. Противоречие с $c \in C$.
Така $a_2 = c_2$, откъдето $a = c$.
Противоречие с $a < c$.

\end{enumerate}

\quad
Така допускането ни се оказа грешно, откъдето $c \leq a$.

\qed
\end{tcolorbox}

\quad
Показахме, че произволно подмножество на $A \times A$ има най-малък елемент. Така $A \times A$ е добре наредено.

\qed
