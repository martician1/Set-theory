\begin{problem}
\textbf{(ZF)}
Нека $x$ е множество. Тогава $x$ е ординал точно тогава,
когато всяко транзитивно собствено подмножество на $x$ е елемент на $x$.
\end{problem}

\textbf{Решение:}

\smallbreak
\quad
$(\Rightarrow)$
В едната посока нека $x$ е ординал и нека $A$ е транзитивно собствено подмножество на $x$.
Да разгледаме множеството $A' = x \setminus A$.
Тъй като $A$ е собствено подмножество на $x$, то $A' \neq \varnothing$,
откъдето от добрата наредба на $x$ следва, че съществува елемент $y \in A'$, такъв че $y \cap A' = \varnothing$.
Нека фиксираме такъв елемент $y$.
Твърдим, че $y = A$.

\begin{tcolorbox}[mybox={Доказателство:}]
\quad
$(\Rightarrow)$ В едната посока, ще докажем, че всеки елемент на $y$ е елемент на $A$.
Нека $a \in y$. От $trans(x)$ това влече, че $a \in x$,
Но, тъй като $y \cap A' = \varnothing$, то $a \notin A'$, откъдето $a \in A$.

\quad
$(\Leftarrow)$ В обратната посока, ще докажем, че всеки елемент на $A$ е елемент на $y$.
Нека $a \in A$. Тогава като следствие от добрата нареденост на $x$ има три възможности:
$a \in y$, $y \in a$ или $a = y$.
Ако $y \in a$, то $y \in A$ от $trans(A)$.
Но $y \in A'$, откъдето този случай е невъзможен.
Ако $a = y$, то отново $y$ е едновременно в $A$ и $A'$, което е невъзможно.
Така единственият възможен случай е $a \in y$.

\quad
Показахме, че $y \subseteq A$ и $A \subseteq y$, следователно $y = A$.

\qed
\end{tcolorbox}

\quad
Накрая остава да съобразим, че $A = y \in A' \subseteq x$, откъдето $A \in x$.

\qed


\quad
$(\Leftarrow)$ В обратната посока.
Нека всяко транзитивно подмножество на $x$ е елемент на $x$
и нека $\alpha$ е най-малкият ординал, който не е елемент на $x$.
Тъй като всеки елемент $\beta$ на $\alpha$ е ординал, който е по-малък от $\alpha$,
то $x$ ще съдържа всеки елемент на $\alpha$, тоест $\alpha \subseteq x$.
Разглеждаме два случая:
\begin{enumerate}
\item
$\alpha \subsetneq x$
Тогава $\alpha$ е транзитивно подмножество на $x$, откъдето от условието $\alpha \in x$.
Противоречие с избора на $\alpha$.

\item
$\alpha = x$. Тогава $x$ е ординал.
\end{enumerate}

\qed

