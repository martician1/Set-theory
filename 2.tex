\begin{problem}
\textbf{(ZF)}
Нека $I$ и $J$ са непразни множества, $\{I_j\}_{j \in J}$ е $J$-индексирана фамилия от непразни множества,
като $I = \union_{j \in J} I_j$. Нека
\[
K = \{L\ |\ L \in \pow(I) \land (\forall j \in J) (L \cap I_j \neq \varnothing)\}
\]

Да се докаже, че за всяка $I$-индексирана фамилия от множества $\{A_i\}_{i \in I}$ са в сила равенствата:
\[
\union_{j \in J} \intersection_{i \in I_j} A_i = \intersection_{L \in K} \union_{i \in L} A_i,
\]
\[
\intersection_{j \in J} \union_{i \in I_j} A_i = \union_{L \in K} \intersection_{i \in L} A_i
\]

\textbf{Решение:}
\smallbreak

\quad
Първо ще докажем, че:
\[
\union_{j \in J} \intersection_{i \in I_j} A_i \subseteq \intersection_{L \in K} \union_{i \in L} A_i
\]

\begin{tcolorbox}[mybox={Доказателство:}]
\quad
Нека \(\displaystyle x \in \union_{j \in J} \intersection_{i \in I_j} A_i\) и нека
вземем $k \in J$, такова че $\displaystyle x \in \intersection_{i \in I_k} A_i$. Тогава:
\begin{equation}
\forall i \in I_k\ [x \in A_i]
\end{equation}

\quad
Нека $L$ е произволен елемент на $K$ и нека $y \in L \cap I_k$.
От (1) $x \in A_y$.
Освен това знаем, че $\displaystyle y \in L$, откъдето $\displaystyle x \in \union_{i \in L} A_i$.

\quad
Щом $\displaystyle x \in \union_{i \in L} A_i$ за произволно $L \in K$,
то $\displaystyle x \in \intersection_{L \in K} \union_{i \in L} A_i$.

\qed
\end{tcolorbox}

\bigbreak
\quad
В обратната посока, ще докажем, че:
\[
\union_{j \in J} \intersection_{i \in I_j} A_i \supseteq \intersection_{L \in K} \union_{i \in L} A_i
\]
\begin{tcolorbox}[mybox={Доказателство:}]
\quad
Нека \(\displaystyle x \in \intersection_{L \in K} \union_{i \in L} A_i\).
Да допуснем, че
$\displaystyle x \notin \union_{j \in J} \intersection_{i \in I_j} A_i$
тоест
$\displaystyle \forall j \in J\ [x \notin \intersection_{i \in I_j} A_i]$
тоест
$\displaystyle \forall j \in J\ \exists k \in I_j \ [x \notin A_k]$.

\quad
Нека разгледаме множество $L$ от такива индекси $k$:
\[
L = \{k \ |\ k \in I \land x \notin A_k\}
\]

\quad
\textbf{Наблюдение 1:} $\displaystyle x \notin \union_{i \in L} A_i$

\quad
\textbf{Наблюдение 2:} $L \in K$.
Наистина, $L \in \pow(I)$ по дефиниция и освен това за произволно $j \in J$ e изпълнено, че $L \cap I_j \neq \varnothing$,
понеже по допускане $I_j$ съдържа елемент $k$, такъв че $x \notin A_k$.

\bigbreak
\quad
Щом $L \in K$ и $\displaystyle x \notin \union_{i \in L} A_i$, то
$\displaystyle x \notin \intersection_{L \in K} \union_{i \in L} A_i$.
Но това противоречи на избора на $x$, следователно допускането ни е грешно и
$\displaystyle \union_{j \in J} \intersection_{i \in I_j} A_i \supseteq \intersection_{L \in K} \union_{i \in L} A_i$.

\qed
\end{tcolorbox}

\bigbreak
\quad
Нека сега разгледаме второто равенство:
\[
\intersection_{j \in J} \union_{i \in I_j} A_i = \union_{L \in K} \intersection_{i \in L} A_i,
\]

\quad
Първо ще докажем, че:
\[
\intersection_{j \in J} \union_{i \in I_j} A_i \subseteq \union_{L \in K} \intersection_{i \in L} A_i
\]

\begin{tcolorbox}[mybox={Доказателство:}]
\quad
Нека
$\displaystyle x \in \intersection_{j \in J} \union_{i \in I_j} A_i$.
Тогава:
\begin{equation}
%\forall j \in J \ [x \in \union_{i \in I_j} A_i] \iff
\forall j \in J \ \exists k \in I_j\ [x \in A_k]
\end{equation}

\quad
Нека разгледаме множество $L$ от такива индекси $k$:
\[
L = \{k \ |\ k \in I \land x \in A_k\}
\]

\quad
\textbf{Наблюдение 1:} $\displaystyle x \in \intersection_{i \in L} A_i$

\quad
\textbf{Наблюдение 2:} $L \in K$.
Наистина, $L \in \pow(I)$ по дефиниция и освен това, като следствие от (2),
за произволно $j \in J$ e изпълнено, че $L \cap I_j \neq \varnothing$.

\bigbreak
\quad
Щом $L \in K$ и $\displaystyle x \in \intersection_{i \in L} A_i$, то
$\displaystyle x \in \union_{L \in K} \intersection_{i \in L} A_i$.

\qed
\end{tcolorbox}

\bigbreak
\quad
В обратната посока, ще докажем, че:
\[
\intersection_{j \in J} \union_{i \in I_j} A_i \supseteq \union_{L \in K} \intersection_{i \in L} A_i
\]

\begin{tcolorbox}[mybox={Доказателство:}]
\quad
Нека $\displaystyle x \in \union_{L \in K} \intersection_{i \in L} A_i$ и нека вземем $L \in K$,
такова че $\displaystyle x \in \intersection_{i \in L} A_i$.
Нека $k$ е произволен елемент на $J$.
Твърдим, че $\displaystyle x \in \union_{i \in I_k} A_i$.
Като свидетел за това можем да вземем множество $A_y$, където $y \in I_k \cap L$.

\quad
Щом $\displaystyle x \in \union_{i \in I_k} A_i$ за произволно $k \in J$,
то $\displaystyle x \in \intersection_{j \in J} \union_{i \in I_j} A_i$.

\qed
\end{tcolorbox}

\end{problem}
