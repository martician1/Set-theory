\begin{problem}
\textbf{(ZF)}
Нека $I \ne \varnothing$ и $\{A_i\}_{i \in I}$ е $I$-индексирана фамилия от множества.
Нека $\{I_j\}_{j \in J}$ е $J$-индексирана фамилия от взаимно чужди и непразни подмножества на $I$,
като $\union_{j \in J} I_j = I$.
Да се докаже, че множествата
$\prod_{i \in I} A_i$ и $\prod_{j \in J}(\prod_{i \in I_j} A_i)$ са равномощни.
\end{problem}

\textbf{Решение:}
\smallbreak
\quad
Нека функцията $r: \prod_{i \in I} A_i \to \prod_{j \in J}(\prod_{i \in I_j} A_i)$ е дефинирана по следния начин:
\[
r(f) = s \text{ за }
s(j) = t \text{ за }
t(i) = f(i)
\]
\quad
Ще покажем, че $r$ е инекция:

\begin{tcolorbox}[mybox={Доказателство:}]
\quad
Нека $r(f_1) = r(f_2)$ и нека $i$ е произволен елемент на $I$.
Нека вземем $j \in J$ такова, че $i \in I_j$.
Ясно е, че тъй като $r(f_1) = r(f_2)$, то $r(f_1)(j) = r(f_2)(j)$.
Накрая остава да съобразим, че от дефиницията на $r$,
$f_1(i) = r(f_1)(j)(i) = r(f_2)(j)(i) = f_2(i)$.
Така получихме, че за произволно $i \in I$ е в сила $f_1(i) = f_2(i)$, откъдето
$f_1 = f_2$.

\qed
\end{tcolorbox}

\quad
Ще покажем, че $r$ е сюрекция:

\begin{tcolorbox}[mybox={Доказателство:}]
\quad
Нека $s \in \prod_{j \in J}(\prod_{i \in I_j} A_i)$.
Дефинираме $f \in \prod_{i \in I} A_i$ по следния начин:
\[
f(i) = s(j)(i) \text{ за $j$ такова, че } i \in I_j
\]
\quad
Твъдим, че $r(f) = s$. Нека $j$ е произволен елемент на $J$.
Тогава $\forall i \in I_j\ [r(f)(j)(i) = f(i) = s(j)(i)]$, тоест $r(f)(j) = s(j)$.
Така получихме, че за произволно $j \in J$ е в сила
$r(f)(j) = s(j)$, откъдето $r(f) = s$.

\qed
\end{tcolorbox}


\quad
Щом $r$ е инекция и сюрекция, то $r$ е биекция
$\Rightarrow \size{\prod_{i \in I} A_i} = \size{\prod_{j \in J}(\prod_{i \in I_j} A_i)}$.

\qed
