\begin{problem}
\textbf{(ZF)}
A. Линейно наредено множество $\mathcal{A} = \pair{A, \le}$ е
\textit{представено като сума на $I$-индексираната фамилия $\{A_i\}_{i \in I}$}, ако:
\begin{itemize}
\item
$(\forall i \in I)(A_i \neq \varnothing)$
\item
$(\forall i \in I)(\forall j \in I)(i \neq j \Rightarrow A_i \cap A_j = \varnothing)$
\item
$\union_{i \in I} A_i = A$
\item
$(\forall i \in I)(\forall j \in I)(i \neq j \Rightarrow (\forall x \in A_i)(\forall y \in A_j)(x < y) \lor (\forall x \in A_i)(\forall y \in A_j)(y < x))$
\end{itemize}

\quad
\textbf{1.}
Нека $\mathcal{A} = \pair{A, \leq}$ е линейно наредено множество,
представено като сума на $I$-индексирана фамилия $\{A_i\}_{i \in I}$.
Бинарната релация $\prec$ е дефинирана с равенството:
\[
\prec = \{z\ |\ (\exists i \in I)(\exists j \in I)(z = \pair{i, j}) \land (\exists x \in A_i)(\exists y \in A_j)(x < y)\}
\]

\quad
Да се докаже, че $\pair{I, \preceq}$ е линейно наредено множество.
За тази наредба $\preceq$ се казва, че е \textit{породена от представянето на $A$ като сума на
$I$-индексираната фамилия $\{A_i\}_{i \in I}$}.

\bigbreak
\textbf{Решение:}

\smallbreak
\quad
Първо да забележим, че $\preceq$ е рефлексивна в $I$ поради факта, че $\preceq$ е рефлексивно затваряне на $\prec$.

\quad
За да докажем, че $\pair{I, \preceq}$ е силно антисиметрична и транзитивна ще използваме следното наблюдение:

\smallbreak
\quad
\textbf{Наблюдение 1:}
\[
	(\forall i, j \in I) [i \preceq j \land i \neq j \iff (\forall x \in A_i)(\forall y \in A_j)(x < y) \land i \neq j]
\]

\begin{tcolorbox}[mybox, title={Доказателство:}]

\quad
Нека $i, j \in I$. Тогава:
\begin{alignat*}{4}
i \preceq j \land i \neq j & \iff & ( & \exists x \in A_i)(\exists y \in A_j)(x < y) \land i \neq j        & \text{ // от дефиницията на $\prec$ } \\
                           & \iff & ( & \exists x \in A_i)(\exists y \in A_j)(x < y)) \land i \neq j \land & \text{} \\
						   &      & ( &(\forall x \in A_i)(\forall y \in A_j)(x < y) \lor                  & \text{} \\
						   &      & ( &\forall x \in A_i)(\forall y \in A_j)(y < x))                       & \text{ // от $A$ - представено като сума на } \\
						   &      &   &                                                                    & \text{ // $I$-индексирана фамилия $\{A_i\}_{i \in I}$} \\
                           & \iff & ( &\forall x \in A_i)(\forall y \in A_j)(x < y)) \land i \neq j        & \text{ // от $A_i \neq \varnothing \neq A_j$}
\end{alignat*}
\qed
\end{tcolorbox}

\quad
Ще докажем, че $\preceq$ е силно антисиметрична:

\begin{tcolorbox}[mybox, title={Доказателство:}]
\quad
Нека $i, j \in I$. Тогава:
\begin{alignat*}{2}
i \preceq j \land i \neq j & \iff (\forall x \in A_i)(\forall y \in A_j)(x < y) \land i \neq j & \text{ // от \textbf{Наблюдение 1}} \\
            & \iff \neg(\exists x \in A_j)(\exists y \in A_i)(x < y) \land i \neq j & \text{} \\
			& \iff j \npreceq i \land i \neq j & \text { // от дефиницията на $\prec$}
\end{alignat*}
\qed
\end{tcolorbox}

\quad
Ще докажем, че $\preceq$ е транзитивна.

\begin{tcolorbox}[mybox, title={Доказателство:}]
\quad
Нека $i, j, k \in I$ са такива, че $i \preceq j \land j \preceq k$.
Ако $i = j$, то имаме, че $i = j \preceq k$.
По подобен начин, ако $j = k$, то $i \preceq j = k$.
Ако $i = k$, то от рефлексивността на $\preceq$ ще бъде изпълнено $i \preceq i = k$.
В случая, когато $i \ne j \ne k \ne i$ по \textbf{Наблюдение 1} ще бъдат в сила:

\begin{itemize}
\item
$(\forall x \in A_i)(\forall y \in A_j)(x < y)$
\item
$(\forall y \in A_j)(\forall z \in A_k)(y < z)$
\end{itemize}

\quad
От транзитивна на релацията $<$ и $A_j \neq \varnothing$ горните две влекат
$(\forall x \in A_i)(\forall z \in A_k)(x < z)$,
откъдето отново от \textbf{Наблюдение 1} следва, че $i \preceq k$.

\qed
\end{tcolorbox}

\quad
Щом $\preceq$ е рефлексивна, силно антисиметрична и транзитивна, то $\preceq$ е линейна наредба
и $\pair{I, \preceq}$ е линейно наредено множество.

\qed

\quad
Б. Нека $\mathcal{C} = \pair{C, \leq}$ е линейно наредено множество.

\quad
$\mathcal{C}$ се нарича \textit{гъсто}, ако има поне два различни елемента и:
\[
(\forall x \in C)(\forall y \in C)(x < y \Rightarrow (\exists z \in C)(x < z \land z < y))
\]

\quad
$\mathcal{C}$ се нарича \textit{разредено}, ако всеки път,
когато $B \subseteq C,\, \pair{B, \leq \cap (B \times B)}$ не е гъсто.

\quad
\textbf{2.} Да се докаже, че всяко линейно наредено множество $\mathcal{A} = \pair{A, \leq}$ е
разредено или може да се представи като сума на такава $I$-индексирана
фамилия $\{A_i\}_{i \in I}$, така че всяко едно от множествата $\pair{A_i, \leq \cap (A_i \times A_i)}$
е разредено и $\pair{I, \preceq}$ е гъсто, където $\preceq$ е породената от това представяне наредба.

\bigbreak
\textbf{Решение:}
\smallbreak
\quad
Нека $\mathcal{A} = \pair{A, \le}$ е произволно неразредено линейно наредено множество.
Ще докажем, че $\mathcal{A}$ може да се представи като сума на $I$-индексирана фамилия $\{A_i\}_{i \in I}$,
така че всяко едно от множествата $\pair{A_i, \leq \cap (A_i \times A_i)}$
е разредено и $\pair{I, \preceq}$ е гъсто, където $\preceq$ е породената от това представяне наредба.

\quad
Първо нека си дефинираме няколко помощни множества и нотации:
\[
\operatorname{interval}(x, y) \coloneq \{z\ |\ z \in A \land (x \le z \le y \lor y \le z \le x)\}
\]
\[
\operatorname{dense}(B) \iff \size{B} \ge 2 \land (\forall x \in B)(\forall y \in B)(x < y \Rightarrow (\exists z \in B)(x < z < y))
\]
\[
\operatorname{diluted}(B) \iff (\forall D \subseteq B)(\neg \operatorname{dense}(D))
\]
\[
\sim\ \coloneq \{\pair{x, y} \in A^2\ |\ \operatorname{diluted}(\operatorname{interval}(x, y))\}
\]

\quad
Твърдим, че $\sim$ е релация на еквивалентност.

\quad
Ще докажем, че $\sim$ е рефлексивна.

\begin{tcolorbox}[mybox, title={Доказателство:}]
\quad
Нека $x \in A$. Не е трудно да се съборази, че $\operatorname{interval}(x, x) = \{x\}$,
откъдето $\operatorname{diluted}(\operatorname{interval}(x, x))$ и $\pair{x, x} \in \sim$.

\qed
\end{tcolorbox}

\quad
Ще докажем, че $\sim$ е симетрична.

\begin{tcolorbox}[mybox, title={Доказателство:}]
\quad
Нека $\pair{x, y} \in\ \sim$. Тогава $\operatorname{diluted}(\operatorname{interval}(x, y))$,
и, тъй като $\operatorname{interval}(x, y) = \operatorname{interval}(y, x)$, то в сила ще бъде и
$\operatorname{diluted}(\operatorname{interval}(y, x))$, откъдето $\pair{y, x} \in\ \sim$.

\qed
\end{tcolorbox}

\quad
За да докажем, че $\sim$ е транзитивна, ще ползваме следните наблюдения:

\smallbreak
\quad
\textbf{Наблюдение 2:}
\[
(\forall x, y, z \in A) [\operatorname{interval}(x, z) \subseteq \operatorname{interval}(x, y) \cup \operatorname{interval}(y, z)]
\]

\begin{tcolorbox}[mybox, title={Доказателство:}]
\quad
Нека $x, y, z \in A$, $a \in \operatorname{interval}(x, z)$ и нека БОО $x < z$. Тогава $x \le a \le z$.
Разглеждаме два случая:
\begin{enumerate}[label={\arabic* сл.}]
\item
$a \le y$. Тогава $x \le a \le y$, откъдето
$a \in \operatorname{interval}(x, y) \subseteq \operatorname{interval}(x, y) \cup \operatorname{interval}(y, z)$.

\item
$a > y$. Тогава $y \le a \le z$, откъдето
$a \in \operatorname{interval}(y, z) \subseteq \operatorname{interval}(x, y) \cup \operatorname{interval}(y, z)$.


\end{enumerate}

\qed
\end{tcolorbox}

\quad
\textbf{Наблюдение 3:}
\[
(\forall x \subseteq A) [\operatorname{diluted}(x) \Rightarrow (\forall y \subseteq x) [\operatorname{diluted}(y)]
\]
\begin{tcolorbox}[mybox, title={Доказателство:}]
\quad
Нека $x \subseteq A$, $\operatorname{diluted}(x)$, $y \subseteq x$ и нека $a \subseteq y$.
Тогава $a \subseteq x$, откъдето $\neg \operatorname{dense}(a)$.

\qed
\end{tcolorbox}


\quad
Сега ще докажем, че $\sim$ е транзитивна.

\begin{tcolorbox}[mybox, title={Доказателство:}]
\quad
Нека $x, y, z \in A$ са такива, че $\pair{x, y}, \pair{y, z} \in\ \sim$. Тогава $\operatorname{diluted}(\operatorname{interval}(x, y))$
и $\operatorname{diluted}(\operatorname{interval}(y, z))$.
Ще докажем, че $\operatorname{diluted}(\operatorname{interval}(x, y) \cup \operatorname{interval}(y, z))$.
Да допуснем противното.
Нека вземем $a$ такова, че $a \subseteq \operatorname{interval}(x, y) \cup \operatorname{interval}(y, z) \land \operatorname{dense}(a)$.
Нека БОО\footnote{Лесно може да се съобрази, че всяко гъсто множество има поне 3 елемента,
откъдето сечението на $a$ с един от интервалите трябва да съдържа поне 2 елемента.}
$\size{a \cap \operatorname{interval}(x, y)} \ge 2$.
Тогава от $\operatorname{dense}(a)$ и дефиницията на $\operatorname{interval}$ ще бъде изпълнено:
\[
(\forall v \in a \cap \operatorname{interval}(x, y))(\forall u \in a \cap \operatorname{interval}(x, y))(v < u \Rightarrow (\exists w \in a \cap \operatorname{interval}(x, y)\footnote{$w \in a$ от $\operatorname{dense}(a)$, а $w \in \operatorname{interval}(x,y)$ (от $x \leq v < w < u \leq y$)})(v < w < u))
\]
\quad
откъдето
$\operatorname{dense}(a \cap \operatorname{interval}(x, y))$.
Но ние знаем, че $a \cap \operatorname{interval}(x, y) \subseteq \operatorname{interval}(x, y)$,
противоречие с $\operatorname{diluted} ( \operatorname{interval} (x, y))$.

\quad
Така показахме, че $\operatorname{diluted}(\operatorname{interval}(x, y) \cup \operatorname{interval}(y, z))$.
Щом това е изпълнено,
то от \textbf{Наблюдения 2 и 3} $\operatorname{diluted}(\operatorname{interval}(x, z))$,
откъдето $\pair{x, z} \in\ \sim$.

\qed
\end{tcolorbox}

\quad
Щом $\sim$ е рефлексивна, симетрична и транзитивна, то $\sim$ е релация на еквивалентност.

\quad
Нека сега вземем $I$ да е множеството от класовете на еквивалентност на $\sim$ и нека за $i \in I$ дефинираме $A_i$ като елементите на класа $i$:
\[
A_i \coloneq \{x\ |\ x \in i\}
\]

\quad
Твърдим, че с текущите дефиниции, $\mathcal{A}$ е
представено като сума на $I$-индексираната фамилия $\{A_i\}_{i \in I}$.
Лесно се вижда, че първите три условия за това са изпълнени,
понеже $\{A_i\}_{i \in I}$ е разбиране на $A$ (от $\sim$ - ралция на еквивалентност).
Ще докажем, че последното условие:
\[
(\forall i \in I)(\forall j \in I)(i \neq j \Rightarrow (\forall x \in A_i)(\forall y \in A_j)(x < y) \lor (\forall x \in A_i)(\forall y \in A_j)(y < x))
\]
\quad
също е в сила.

\begin{tcolorbox}[mybox, title={Доказателство:}]
\quad
Нека допуснем противното, тоест нека:
\[
(\exists i \in I)(\exists j \in I)(i \neq j \land ((\exists x \in A_i)(\exists y \in A_j)(x \le y) \land (\exists x' \in A_i)(\exists y' \in A_j)(y' \le x')))
\]
\quad
Нека вземем $i, j \in I$, $x, x' \in A_i$ и $y, y' \in A_j$ такива, че $i \neq j$ и $x \le y \land x' \ge y'$. Разглеждаме два случая:
\begin{enumerate}[label={\arabic* сл.}]
\item
$x' \ge y$. Тогава $\operatorname{interval}(x, y) \subseteq \operatorname{interval}(x, x')$.
От \textbf{Наблюдение 3} това влече, че $y \in A_i$, откъдето $i = j$, противоречие.

% \item
% $y' \le x$. Тогава $\operatorname{interval}(x, y) \subseteq \operatorname{interval}(y, y')$.
% От \textbf{Наблюдение 3} това влече, че $x \in A_j$, откъдето $i = j$, противоречие.

\item
$x' < y$.
Тогава $\operatorname{interval}(x', y) \subseteq \operatorname{interval}(y, y')$.
От \textbf{Наблюдение 3} това влече, че $x' \in A_j$, откъдето $i = j$, противоречие.
\end{enumerate}

\qed
\end{tcolorbox}

\quad
Ще покажем, че също така $\forall i \in I\ (\operatorname{diluted}(A_i))$.
\begin{tcolorbox}[mybox, title={Доказателство:}]
\quad
Да допуснем противното. Нека $i \in I$ и $C \subseteq A_i$ са такива, че $\operatorname{dense}(C)$
и нека $x$ и $y$ са два различни елемента на $C$.
Тогава $\operatorname{dense}(\operatorname{interval}(x, y) \cap C)$,
но в същото време $\operatorname{diluted}(\operatorname{interval}(x, y))$ от
$x, y \in A_i$. Противоречие.

\qed
\end{tcolorbox}
\quad
Накрая ще покажем, че $\pair{I, \preceq}$ е гъсто.
\begin{tcolorbox}[mybox, title={Доказателство:}]
\quad
Нека $i, j \in I$ са такива, че $i \prec j$ и $i \neq j$ и нека $a$ е произволен елемент на $A_i$ и $b$ е произволен елемент на $A_j$,
като БОО нека $a < b$.
Тъй като $i \neq j$, то $\neg \operatorname{diluted}(\operatorname{interval}(a, b))$, тоест:
\[
(\exists C \subseteq \operatorname{interval}(a, b)) (\operatorname{dense}(C))
\]
\quad
Нека $c$ е произволен некраен елемент на $C$ и нека $k \in I$ е такова, че $c \in A_k$.
Тогава $i \prec k \land i \neq k$.
Наистина, като свидетели за $i \prec k$ можем да вземем $a$ и $c$, понеже $a < c$,
а $i \neq k$, тъй като $\neg \operatorname{diluted}(\operatorname{interval}(a, c))$.

\quad
По подобен начин може да се покаже, че $k \prec j \land k \neq j$.

\quad
Така $k \in I$ е такова, че $i \prec k \prec j$ и $i \neq k \neq j$, откъдето $\pair{I, \preceq}$ е гъсто.

\qed
\end{tcolorbox}

\qed

\end{problem}
