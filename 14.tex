\begin{problem}
Нека $A$ и $B$ са множества.
Ще казваме, че $A \leq B$, ако съществува инекция
$f : A \rightarrowtail B$.
Ще казваме, че $A \leq^* B$, ако съществува сюрекция
$f : B \twoheadrightarrow A$.
Докажете, че:
\begin{enumerate}
\item
\textbf{(ZF)}
За произволни множества $A \neq \varnothing$ и $B$ е в сила:
$A \leq B \Rightarrow A \leq^* B$.

\textbf{Решение:}

\smallbreak
\quad
Нека $f: A \rightarrowtail B$ е произволна инекция и нека $a$ е произволен елемент на $A$.
Дефинираме функцията $g: B \rightarrow A$ по следния начин:
\[
g(b) = \begin{cases}
       f^{-1}(b) & \text{, ако $b \in \operatorname{Rng}(f)$} \\
	   a       & \text{, иначе}
       \end{cases}
\]

\quad
Твърдим, че $g$ е сюрекция.

\begin{tcolorbox}[mybox={Доказателство:}]
\quad
Нека $c \in A$. Тогава $g(f(c)) = f^{-1}(f(c)) = c$, тъй като $f(c) \in \operatorname{Rng(f)}$.

\qed
\end{tcolorbox}

\item
\textbf{(ZF)}
За произволни множества $A$ и $B$ е в сила:
$A \leq^* B \Rightarrow \pow(A) \leq \pow(B)$.

\textbf{Решение:}

\smallbreak
\quad
Нека $f: B \twoheadrightarrow A$ е произволна сюрекция.
Дефинираме функцията $g: \pow(A) \rightarrow \pow(B)$ по следния начин:
\[
g(X) = \{b\ |\ b \in B \land f(b) \in X \}
\]

\quad
Твърдим, че $g$ е инекция.

\begin{tcolorbox}[mybox={Доказателство:}]
\quad
Нека $X, X' \in \pow(A)$ са такива, че $g(X) = g(X')$.
Да допуснем, че $X \neq X'$. Нека БОО вземем $a \in A$ такова, че $a \in X \land a \notin X'$
и нека $b$ е такова, че $f(b) = a$. Тогава $b \in g(X) \land b \notin g(X')$,
откъдето $g(X) \neq g(X')$. Противоречие.

\quad
Така допускането ни се оказа грешно, следователно $X = X'$.

\qed
\end{tcolorbox}

\item
\textbf{(ZF)}
За произволни множества $A$ и $B$, ако $B$ е добре наредимо, то е в сила:
$A \leq^* B \Rightarrow A \leq B$.

\textbf{Решение:}

\smallbreak
\quad
Нека $f: B \twoheadrightarrow A$ е произволна сюрекция и нека $\pair{B, \leq}$ е произволна добра наредба.
Дефинираме функцията $g: A \rightarrow B$ по следния начин:
\[
g(a) = min_{\leq}(\{c\ |\ c \in B \land f(c) = a\})
\]

\quad
Твърдим, че $g$ е инекция.

\begin{tcolorbox}[mybox={Доказателство:}]
\quad
Нека $a, a' \in A$ са такива, че $g(a) = g(a')$.
Тогава от дефиницията на $g$ имаме, чe:
\[
a = f(g(a)) = f(g(a')) = a'
\]
\qed
\end{tcolorbox}


\item
\textbf{(ZFC)}
За произволни множества $A$ и $B$ е в сила:
$A \leq^* B \Rightarrow A \leq B$.

\textbf{Решение:}

\smallbreak
\quad
Нека $f: B \twoheadrightarrow A$ е произволна сюрекция
и нека $h: \pow(B) \setminus \varnothing \to B$ е произволна функция на избора за $B$.
Дефинираме функцията $g: A \rightarrow B$ по следния начин:
\[
g(a) = h(\{c\ |\ c \in B \land f(c) = a\})
\]

\quad
Твърдим, че $g$ е инекция.

\begin{tcolorbox}[mybox={Доказателство:}]
\quad
Нека $a, a' \in A$ са такива, че $g(a) = g(a')$.
Тогава от дефиницията на $g$ имаме, чe:
\[
a = f(g(a)) = f(g(a')) = a'
\]
\qed
\end{tcolorbox}


\end{enumerate}
\end{problem}
