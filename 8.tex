\begin{problem}
\textbf{(ZF)}
Нека $\mathcal{S}$  е непразно подмножество на $\pow(A)$, което има следните две свойства:
\begin{enumerate}
\item
$A \in \mathcal{S}$
\item
$\mathcal{S}$ е затворено относно произволни непразни сечения, т.е. всеки път,
когато $\mathcal{X} \neq \varnothing$ и $\mathcal{X} \subseteq \mathcal{S}$,
е в сила $\intersection\mathcal{X} \in \mathcal{S}$, и $A \in \mathcal{S}$.
Да се докаже, че всяка монотонна функция $h: \mathcal{S} \to \mathcal{S}$ има неподвижна точка.
\end{enumerate}
\end{problem}

\textbf{Решение:}

\smallbreak
\quad
Ще имитираме теоремата на Тарски. Нека си дефинираме $X$ да е следното множество:
\[
X = \{ x \ |\ x \in \mathcal{S} \land h(x) \subseteq x\}
\]

\quad
Да забележим, че $A \in \mathcal{S} \land h(A) \subseteq A$, откъдето $A \in X$, тоест $X$ не е празно.

\quad
Нека сега разгледаме множеството $X' = \intersection X$.
От второто свойство на множеството $\mathcal{S}$, то е в $\mathcal{S}$.

\quad
Твърдим, че освен това $X'$ е неподвижна точка на $h$.

\begin{tcolorbox}[mybox={Доказателство:}]
\quad
Първо ще докажем, че $h(X') \subseteq X'$.

\begin{tcolorbox}[mybox={Доказателство:}, colback=green!20, colframe=green!60]
\quad
Нека $Y$ е произволен елемент на $X$. Тогава от $h$-монотонна и от $X' = \intersection X \subseteq Y$ ще бъде в сила следното:
\[
h(X') \subseteq h(Y) \subseteq Y
\]

\quad
Така полуаваме, че $h(X')$ е подмножество на всеки елемент на $X$,
откъдето $h(X')$ е подмножество на $\intersection X = X'$.

\qed
\end{tcolorbox}

\quad
Сега ще докажем, че $X' \subseteq h(X')$.

\begin{tcolorbox}[mybox={Доказателство:}, colback=green!20, colframe=green!60]
\quad
Тъй като $h(X') \subseteq X'$, то от $h$-монотонна имаме, че:
$h(h(X')) \subseteq h(X')$, откъдето $h(X') \in X$. Но $X' = \intersection X$,
следователно $X \subseteq h(X')$.

\qed
\end{tcolorbox}

\quad
Щом $h(X') \subseteq X'$ и $X' \subseteq h(X')$, то $X' = h(X')$, откъдето $X'$ е неподвижна точка на $h$.

\qed
\end{tcolorbox}

