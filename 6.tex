\begin{problem}
\textbf{(ZF)}
Нека $A$ е множество, за което е в сила $\size{A} = \size{A \cup \{A\}}$.
Да се докаже, че всеки път, когато $B$ е множество и $f: \pow(A) \cup B \bijarrow \pow(\pow(A))$
множествата $B$ и $\pow(\pow(A))$ са равномощни.
\end{problem}

\textbf{Решение:}

\smallbreak
\quad
Нека първо съобразим, че щом $f$ е биекция от $\pow(A) \cup B$ към $\pow(\pow(A))$,
то рестрикцията $f\restriction_B$ е инекция от $B$ към $\pow(\pow(A))$.
Ще покажем, че съществува инекция и в обратната посока.

\quad
Нека $h: \pow(\pow(A)) \times \pow(\pow(A)) \bijarrow \pow(A) \cup B$ е произволна биекция\footnote[2]{Знаем, че такава биекция съществува, понеже
$\size{\pow(\pow(A)) \times \pow(\pow(A))} = \size{\pow(\pow(A))}$ от 5-та задача и $\size{\pow(\pow(A))} = \size{\pow(A) \cup B}$ от условието.}
и нека $Y$ е следното множество:
\[
Y = \{y\ |\ y \in \pow(\pow(A)) \land \forall x \in \pow(\pow(A))\ [h(x, y) \notin \pow(A)]\}
\]

\quad
Ще докажем, че $Y \ne \varnothing$:
\begin{tcolorbox}[mybox={Доказателство:}]
\quad
Нека допуснем, че $Y = \varnothing$. Тогава:
\begin{equation}\label{blasphemy}
\forall y\, \exists x\ [h(x, y) \in \pow(A)]
\end{equation}

\quad
Ще покажем, че последното е невъзможно, като построим сюрективна фунцкия от $\pow(A)$ към $\pow(\pow(A))$.
Нека $r: \pow(A) \to \pow(\pow(A))$ е дефинирана по следния начин:
\[
r(z) = y \text{ за } h(x, y) = z
\]
\quad
Твърдим, че $r$ е сюрекция.

\begin{tcolorbox}[mybox={Доказателство:}, colback=green!20, colframe=green!60]
\quad
Нека допуснем противното и нека вземем $y \in \pow(\pow(A))$ такова, че:
\begin{equation}\label{notsurjection}
\forall z \in \pow(A)\ [r(z) \ne y].
\end{equation}
\quad
От (\ref{blasphemy}) съществува $x \in \pow(\pow(A))$ такова, че $h(x, y) \in \pow(A)$.
Нека вземем $z$ = $h(x, y)$.
Така, от дефиницията на $r$, $r(z) = y$, което е противоречие с (\ref{notsurjection}).

\qed
\end{tcolorbox}

\quad
Получихме, че съществува сюрекция от $\pow(A) \to \pow(\pow(A))$,
което противоречи с теоремата на Кантор.
Така първоначалното ни допускане е грешно, тоест $Y \ne \varnothing$.


\qed

\end{tcolorbox}

\quad
Нека сега вземем $y$ да е произволен елемент на $Y$. Дефинираме функция $s: \pow(\pow(A)) \to B$ по следния начин:
\[
s(x) = h(x, y)
\]
\quad
Да забележим, че $s$ е добре-дефинирана, тъй като от $y \in Y$ следва, че $\forall x \in \pow(\pow(A))\ [h(x, y) \in B]$.
Освен това $s$ е инекция поради инективността на $h$.

\smallbreak
\quad
Накрая остава да съобразим, че щом съществува инекция от $B$ към $\pow(\pow(A))$ и щом съществува инекция от $\pow(\pow(A))$ към $B$,
то от теоремата на Кантор-Шрьодер-Бернщайн, $B$ и $\pow(\pow(A))$ са равномощни.

\qed
