\begin{problem}
\textbf{(ZF)}
За произволно множество $A$ от реални числа и за произволно реално число $r$ с $r + A$ означаваме множеството
\[
\{x\ |\ x \in \mathbb{R} \land (\exists y \in A)(x = r + y)\}.
\]

\quad
Нека $A$ е най-много изброимо множество от реални числа.
Да се докаже, че има такова реално число $r$, че $A \cap (r+A) = \varnothing$.
\end{problem}

\textbf{Решение:}

\smallbreak
\quad
Нека фиксираме една инекция от $f$ от $A$ в $\omega$.
Ще извършим няколко наблюдения:

\smallbreak

\quad
\textbf{Наблюдение 1:}
$A$ е добре наредимо множество.
\begin{tcolorbox}[mybox={Доказателство:}]
\quad
Нека $\leq$ е релация над $A$ дефинирана по следния начин:
\[
\leq = \{\pair{a, b}\ |\ \pair{a, b} \in A \times A \land f(a) \leq f(b) \}
\]
\quad
С тази дефиниция добрата нареденост на $\pair{A, \leq}$ следва директно от добрата нареденост на $\omega$.

\qed
\end{tcolorbox}

\quad
Отсега нататък нека си фиксираме една добра наредба $\leq_A$ над $A$.

\smallbreak

\quad
\textbf{Наблюдение 2:}
$A \times A$ е най-много изброимо.
\begin{tcolorbox}[mybox={Доказателство:}]
\quad
Тъй като $\omega^2$ е изброимо, ще бъде достатъчно да покажем съществуването на инекция от $A \times A$ в $\omega^2$.
Дефинираме $g: A \times A \to \omega^2$ по следния начин:
\[
g(a, b) = \pair{f(a), f(b)}
\]

\quad
Инективността на $g$ следва директно от инективността на $f$.

\qed
\end{tcolorbox}

\quad
\textbf{Наблюдение 3:}
$A \times A$ е добре наредимо множество. (директно следствие от \textbf{Задача 12}).

\quad
Отсега нататък нека си фиксираме една добре наредба $\leq_{A \times A}$ над $A \times A$.

\quad
Да разгледаме множеството $S$ от разликите на елементи в $A$:
\[
S = \{r\ |\ r \in \mathbb{R} \land \exists \pair{s, t} \in A \times A\ [s - t = r] \}
\]

\quad
Твърдим, че също като $A \times A$, $S$ е най-много изброимо.

\begin{tcolorbox}[mybox={Доказателство:}]
\quad
Нека функцията $h: S \to A \times A$ е дефинирана по следния начин:
\[
h(r) = \pair{a, b} \text{ за $\pair{a, b} = min_{\leq}\{\pair{c, d}\ |\ \pair{c, d} \in A \times A \land |c - d| = r \}$}
\]

\quad
Ще докажем, че $h$ е инекция:
\begin{tcolorbox}[mybox={Доказателство:}, colback=green!20, colframe=green!60]
\quad
Нека $\pair{a_1, b_1} = h(r_1) = h(r_2) = \pair{a_2, b_2}$.
Ако $r_1 \neq r_2$, то от дефиницията на $h$ имаме, че $a_1 - b_1 = r_1 \neq r_2 = a_2 - b_2$,
откъдето $\pair{a_1, b_1} \neq \pair{a_2, b_2}$. Противоречие. Така $r_1 = r_2$.

\qed
\end{tcolorbox}

\quad
Накрая, от транзитивността на $\leq$, щом
$\size{S} \leq \size{A \times A}$
и
$\size{A \times A} \leq \size{\omega}$,
то $\size{S} \leq \size{\omega}$

\qed
\end{tcolorbox}

\quad
Щом $S$ е най-много изброимо множество от реални числа, то съществува
реално число, което не е в $S$.
Нека $r$ е едно такова число.
Твъдим, че $A \cap (r + A) = \varnothing$.
\begin{tcolorbox}[mybox={Доказателство:}]
\quad
Да допуснем противното. Нека $s \in A \cap (r + A)$.
От дефиницията на $r + A$ последното влече, че $(s - r) \in A$.
Тогава за $\pair{s, s-r} \in A \times A$ е изпълнено, че $s - (s-r) = r$,
откъдето $r \in S$. Противоречие с $r \notin S$.

\qed
\end{tcolorbox}

\qed



% Needs ZFC
%
%Нека допуснем противното, тоест нека за всяко реално число $A \cap (r + A) \neq \varnothing$.
%
%\quad
%Тъй като $A$ e най-много изброимо множество, то може да бъде добре наредено.
%Нека вземем произволна добра наредба $\pair{A, \leq}$ над $A$ и нека
%$f: \mathbb{R} \to A$ е следната функция:
%\[
%f(r) = min_{\leq}(A \cap (r + A))
%\]
%
%\quad
%Нека сега за всяко $a \in A$ си дефинираме $R_a$ като множеството от реални числа първообрази на $a$ относно $f$:
%\[
%R_a = \{r\ |\ r \in \mathbb{R} \land f(r) = a \}
%\]
%
%\quad
%Да съобразим, че индексираната фамилия $\{R_a\}_{a \in A}$ е разбиване (с възможни празни множества) на $\mathbb{R}$,
%тъй като $f$ е тотална функция.
%Освен това, тъй като $A$ е най-много изброимо, а $\mathbb{R}$ не е изброимо,
%то съществува $a \in A$ такова, че $R_a$ е неизброимо.
%Нека фиксираме едно такова $a$. Тогава:
%\[
%\forall r \in R_a\ [a \in r + A]
%\]
%
%\quad
%откъдето:
%\[
%\forall r \in R_a\ [a - r  \in A]
%\]
%
%\quad
%Така получихме, че $A$ има неизброимо много елементи, което е противоречие.
%
%\qed
%
%
