\begin{problem}
\textbf{(ZF)}
Нека $\pair{A, R}$ е линейно наредено множество.
Нека всеки път, когато $\omega$ е начален сегмент на $\pair{A, R}$, е в сила
$\omega = A$ или $(\exists x \in A)(\omega = seg(x))$.
Докажете, че $\pair{A, R}$ е добре наредено множество.
\end{problem}

\textbf{Решение:}

\smallbreak
\quad
Нека вземем $y$ да е произволно непразно подмножество на $A$.
Ще докажем, че $y$ има най-малък елемент относно $R$.
За целта си дефинираме следното множество:
\[
X = \{ a\ |\ a \in \pow(A) \land \exists b\ [ b \in y \land  a = seg(b)] \}
\]
\quad
Да забележим, че $X$ не е празно, тъй като за произволен елемент $b$ на $y$ е изпълнено, че $seg(b) \in X$.
Освен това $\intersection X$ е начален сегмент на $A$, тъй като сечението на начални сегменти е начален сегмент.

\quad
Щом $\intersection X$ е начален сегмент на $A$, то от условието на задачата имаме две възможности:
\begin{enumerate}[label={\arabic* сл.}]
\item
$\intersection X = A$.
Този случай е невъзможен, тъй като $\forall x \in A\ [A \nsubseteq seg(x)]$.
\item
$\intersection X = seg(x)$ за някакво $x \in A$.
Тогава директно по \textbf{Лема 2} и \textbf{Лема 3}, $x$ е най-малкият елемент на $y$ относно $R$.

% Твърдим, че в този случай $x \in y$.
% За да докажем това ще използваме следното помощно наблюдение:
%
% \quad
% \textbf{Наблюдение 1:}
% \[
% \forall a, b \in A\ [ seg(a) \subseteq seg(b) \lor seg(b) \subseteq seg(a)]
% \]
% \begin{tcolorbox}[mybox={Доказателство:}]
% \quad
% Нека $a$ и $b$ са произволни елементи на $A$.
% Тогава:
% \[
% \forall c \in seg(a) \ [cRa]
% \]
% \[
% \forall c \in seg(b) \ [cRb]
% \]
%
% \quad
% Разглеждаме два случая:
% \begin{enumerate}[label={\arabic* сл.}]
% \item
% $aRb$.
% Тогава $\forall c \in seg(a) \ [cRa \land aRb]$, тоест $\forall c \in seg(a) \ [cRb]$, откъдето $seg(a) \subseteq seg(b)$.
%
% \item
% $bRa$.
% Тогава $\forall c \in seg(b) \ [cRb \land bRa]$, тоест $\forall c \in seg(b) \ [cRa]$, откъдето $seg(b) \subseteq seg(a)$.
%
% \end{enumerate}
% \qed
% \end{tcolorbox}
%
% \quad
% Сега ще докажем, че $x \in y$.
% \begin{tcolorbox}[mybox={Доказателство:}]
% \quad
% Да допуснем противното.
% Нека $x \notin y$.
% Тогава $seg(x) \notin X$.
% Разглеждаме два случая:
% \begin{enumerate}[label={\arabic* сл.}]
% \item
% $\forall a \in X\ [seg(x) \subsetneq a]$.
% Тогава $\forall a \in X\ [x \in a]$, тоест $x \in \intersection X = seg(x)$. Противоречие.
% \item
% $\exists a \in X\ [a \subsetneq seg(x) ]$.
% Тогава $a \subsetneq seg(x) = \intersection X \subseteq a$. Противоречие.
% \end{enumerate}
%
% \quad
% От \textbf{Наблюдение 1} тези 2 случая са изчерпателни.
% Така допускането не се оказва грешно, откъдето $x \in y$.
%
% \qed
% \end{tcolorbox}
%
% \quad
% Накрая остава да съобразим, че щом $X = seg(x) \land x \in y$, то $x$ е най-малкият елемент на $y$.
%
% \begin{tcolorbox}[mybox={Доказателство:}]
% \quad
% Да допуснем противното.
% Нека $x' \in y$ е такова, че $x' \neq x \land x'Rx$.
% Тогава $x' \in seg(x) = \intersection X \subseteq seg(x')$.
% Противоречие с $x' \notin seg(x')$.
%
% \qed
% \end{tcolorbox}
\end{enumerate}

\quad
Показахме, че произволно подмножество на $A$ има най-малък елемент относно $R$,
следователно $\pair{A, R}$ е добре наредено множество.

\qed

