\begin{problem}
\textbf{(ZF)}
Нека $\pair{A, \leq}$ е линейно наредено множество.
Нека функцията $\pi: \pow(A) \to \pow(A)$ е определена чрез:
\[
\pi(X) = \{y \in A\ | \ seg(y) \subseteq X\}
\]

Покажете, че $\pi$ е монотонна и, че ако $A^*$ е най-малката неподвижна точка на $\pi$,
то за всяко $x \in A$
\[
x \in A^* \iff \pair{seg(x), \leq \cap (seg(x) \times seg(x))} \text{ е добре наредено множество.}
\]
\end{problem}


\textbf{Решение:}

\smallbreak
\quad
Първо ще направим няколко наблюдения.

\quad
\textbf{Наблюдение 1:}
Всяко подмножество $X$ на $A$ $\pi(X)$ е начален сегмент на $A$.
\begin{tcolorbox}[mybox={Доказателство:}]
\quad
Нека $X \in \pow(A)$, $a \in \pi(X)$ и нека $b \in A$ е такова, че $b < a$.
Тогава $seg(a) \subseteq X$ и $b \in seg(a)$, откъдето $seg(b) \subseteq seg(a) \in X$,
следователно $b \in \pi(X)$.

\qed
\end{tcolorbox}

\quad
\textbf{Наблюдение 2:}
$\forall x \in A\ [A^* \neq seg(x)]$.
\begin{tcolorbox}[mybox={Доказателство:}]
\quad
Нека сега допуснем, че същесвува $x \in A$ такова, че $A^* = seg(x)$ и нека фиксираме това $x$.
Тогава от дефиницията на $\pi$, $x \in \pi(A^*) = A^* = seg(x)$.
Противоречие.

\qed
\end{tcolorbox}

\quad
\textbf{Наблюдение 3:}
$\forall X \subsetneq A^*\, \exists x \in A\ [\pi(X) = seg(x)]$.
\begin{tcolorbox}[mybox={Доказателство:}]
\quad
Да допуснем противното. Нека $X \subsetneq A^*$ е такова, че $\forall x \in A\ [\pi(X) \neq seg(x)]$.
Тогава $\pi(\pi(X)) = \pi(X)$.
\begin{tcolorbox}[mybox={Доказателство:}, colback=green!20, colframe=green!60]
\quad
Нека $x \in \pi(\pi(X))$. Тогава $seg(x) \subseteq \pi(X)$.
Тъй като $\pi$ е начален сегмент и $\pi(X) \neq seg(x)$,
последното влече, че $x \in \pi(\pi(X)).$

\quad
Нека сега $x \in \pi(X)$.
Тогава $seg(x) \subseteq X$ и $\forall y < x\ [ seg(y) \subsetneq seg(x) \subseteq X]$, тоест $\forall y < x\ [y \in \pi(X)]$.
Последното влече, че $seg(x) \subseteq \pi(X)$, откъдето $x \in \pi(\pi(X))$.

\qed

\end{tcolorbox}

\quad
Така получихме, че $\pi(X)$ е неподвижна точка. Тъй като $A^*$ е най-малката такава, то $X = A^*$.

\qed
\end{tcolorbox}


\quad
$(\Rightarrow)$ Нека сега вземем $x$ да е произволен елемент на $A^*$
и нека вземем $y$ да е произволно непразно подмножество на $seg(x)$.
Ще докажем, че $y$ има най-малък елемент относно $\leq$.
За целта си дефинираме следното множество:
\[
X = \{ a\ |\ a \in \pow(seg(x)) \land \exists b\ [ b \in y \land  a = seg(b)] \}
\]
\quad
Тъй като $\intersection X \subsetneq A^*$, то по \textbf{Наблюдение 3}
$(\exists a \in A)(\intersection X = seg(a))$.
Така по \textbf{Лема 3} $y$ има най-малък елемент.

\qed

\quad
$(\Leftarrow)$ Нека сега
$\pair{seg(x), \leq \cap (seg(x) \times seg(x))}$
е добре наредено множество.
Да допуснем, че $x \notin A^*$.
Така от $A^*$ - начален сегмент, следва че $A^* \subsetneq seg(x)$.

\quad
Нека сега разгледаме множеството $seg(x) \setminus A^* \subseteq seg(x)$.
Тъй като
$\pair{seg(x), \leq \cap (seg(x) \times seg(x))}$
е добре наредено множество, то $seg(x) \setminus A^*$ има минимален елемент $z$.
Това обаче би означавало, че $A^* = seg(z)$, противоречие с \textbf{Наблюдение 2}.

\quad
Така допускането ни се оказа грешно, откъдето $x \in A^*$.

\qed
